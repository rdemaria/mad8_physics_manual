%macropackage=LPLAIN
\documentstyle[11pt,twoside,a4p,bezier,makeidx]{report}
\def\documentlabel#1{\gdef\@documentlabel{#1}}
\gdef\@documentlabel{\tt Third draft}

% ***** varying information for partial printing and final copy *****
% *******************************************************************
\newif\ifdraft
\draftfalse
 
\ifdraft\else
  \documentlabel{CERN/SL/92-?? (AP)}
  \makeindex
\fi
 
% *******************************************************************
 
\raggedbottom %required to avoid underfull pages with equations
 
% Allow large floats to be interspersed with the text
\renewcommand{\topfraction}{1.}
\renewcommand{\bottomfraction}{1.}
\renewcommand{\textfraction}{0.1}
\setcounter{topnumber}{10}
\setcounter{bottomnumber}{10}
\setcounter{totalnumber}{20}
 
\def\mad{MAD~Version~8}
 
\pagestyle{headings}
 
% heading for the \chapter command.
\makeatletter
\def\@makechapterhead#1{{
   \parindent 0pt\raggedright\LARGE \bf
   \ifnum \c@secnumdepth >\m@ne
      \@chapapp{} \thechapter.\ \ \fi
   #1\par\nobreak\vskip 20pt
}}
 
% heading for the \chapter* command.
\def\@makeschapterhead#1{{
   \parindent 0pt\raggedright\LARGE \bf
   #1\par\nobreak\vskip 20pt
}}
 
% titles for the lists of figures and tables:
\def\listoffigures{
   \section*{List of Figures}
   \@starttoc{lof}
}
\def\listoftables{
   \section*{List of Tables}
   \@starttoc{lot}
}
 
% description lists:
\def\mylist{\list{}{
   \setlength{\labelwidth}{2.3cm}
   \setlength{\leftmargin}{2.5cm}
   \let\makelabel\mylabel}
}
 
\let\endmylist\endlist
 
\def\mylabel#1{#1\hfill}
 
% automatic indexing of keywords:
\def\keyitem#1{\item[{\tt #1}]\index{#1}}
\def\ttindex#1{{\tt #1}\index{#1}}
\def\emindex#1{{\em #1\/}\index{#1}}
\def\ttitem#1{\item[{\tt #1}]}
\def\emitem#1{\item[{\em #1}]}
\def\bfitem#1{\item[{\bf #1}]}
 
% MAD command specifications:
\newcommand{\mybox}[1]{
   \begin{quote}
      \tt
      \fbox{
         \begin{minipage}{0.95\textwidth}
            \begin{tabbing}
               #1
            \end{tabbing}
         \end{minipage}
      }
   \end{quote}
}
 
% MAD command examples:
\newcommand{\myxmp}[1]{
   \begin{quote}
      \tt
      \begin{tabbing}
         #1
      \end{tabbing}
   \end{quote}
}
 
% Some Mathematical Symbols and Operators
\def\bold#1{\hbox{\bf #1}}
\def\sume{\sum_{\hbox{elements}}}
\def\sumq{\sum_{\hbox{quadrupoles}}}
\def\sums{\sum_{\hbox{sextupoles}}}
\def\sumo{\sum_{\hbox{octupoles}}}
\def\half{\frac{1}{2}}
\def\sixth{\frac{1}{6}}
\def\sign{\mathop{\mathrm{sign}}\nolimits}
\def\Tr{\mathop{\mathrm{Tr}}\nolimits}
\def\ad{{\sl Ad\/}}
\def\adphsp{$\{{:}f{:}\}$}
\def\eqdef{\stackrel{def}{=}}
\def\gappeq{\stackrel{>}{\approx}}
\def\lieop#1{{:}{#1}{:}}
\def\lietran#1{e^{{:}{#1}{:}}}
\def\map#1{$\cal #1$}
\def\numclass#1{\mathrm{\bf#1}}
\def\order{order}
\def\phsp{\{Z\}}
\def\pbkt#1#2{\sum_{k=1}^3\left(
  {\partial #1\over\partial q_k}{\partial #2\over\partial p_k} -
  {\partial #1\over\partial p_k}{\partial #2\over\partial q_k}
\right)}
 
\def\aux{%
  \left(p_x^2+p_y^2+\frac{p_t^2}{\beta_s^2\gamma_s^2}\right)%
}
\def\solaux{%
  \left((p_x+ky)^2+(p_y-kx)^2+\frac{p_t^2}{\beta_s^2\gamma_s^2}\right)%
}
 
\def\vbar{\overline{v}}
\def\wbar{\overline{w}}
 
% `array' environment using display style
\long\def\eqarray#1{\vcenter{\let\\=\cr\openup1\jot
  \halign{&\strut$\,\displaystyle{##}\;$\hfil\crcr#1\crcr}}}
 
% `array' environment using display style for matrices
\long\def\myarray#1{\null\vcenter{\let\\=\cr\openup1\jot
  \halign{&\strut\hfil$\>\displaystyle{##}\>$\hfil\crcr#1\crcr}}}
 
\tabskip=0pt plus 1fill
\begin{document}
\setlength{\evensidemargin}{\oddsidemargin}
 
\makeatother
 
% ====================================================================
 
\ifdraft\else
  \pagenumbering{roman}
\fi
\begin{titlepage}
\begin{center}\normalsize
EUROPEAN LABORATORY FOR PARTICLE PHYSICS
\end{center}
\vskip 0.7cm
\begin{flushright}
\@documentlabel                      % document label
\end{flushright}
\vskip 2.3cm
\begin{center}\LARGE                 % document title
{\bf The MAD Program} \\
(Methodical Accelerator Design) \\
Version 8.13 \\
{\bf Physical Methods Manual}
\end{center}
\vskip 1.5em
\begin{center}                       % author
F. Christoph Iselin
\vskip 2em
 
{\large \bf Abstract}
\end{center}
\begin{quotation}
MAD is a tool for charged-particle optics in
alternating-gradient accelerators and beam lines.
It can handle from very large to very small accelerators,
and solve various problems on such machines.
 
This document outlines the physical models used in MAD.
It should help the physicist in understanding the precise function
of the program, and to appreciate possible limits of its validity.
\end{quotation}
 
\vfill
\begin{center}
Geneva, Switzerland \\
\today
\end{center}
\end{titlepage}
 
\ifdraft\else
  \tableofcontents
  \listoffigures
  \listoftables
  \cleardoublepage\pagenumbering{arabic}
\fi
 
 
% ====================================================================
 
\chapter{Basic Conventions and Notations}
\label{notation}
\index{conventions}
\index{notations}
 
 
\section{Canonical Variables}
\index{variables!canonical}
\index{canonical variables}
\index{phase space}
Many beam optics programs work on the set of variables
\begin{equation}
x,\quad x',\quad y,\quad y',\quad \Delta s,\quad \frac{\Delta p}{p_0}.
\end{equation}
These variables do not form a set of canonical pairs in
six-dimensional phase space. 
For this reason MAD uses the set of canonical variables
\begin{equation}
( x, \; p_x/p_s), \qquad ( y, \; p_y/p_s), \qquad
( - c \Delta t, \; p_t = \Delta E/p_s c).
\end{equation}
In earlier versions MAD used $p_s = p_0$, the {\em design momentum},
to normalise the momentum components.
As from Version~8.13, MAD normalises with the {\em average momentum}
$p_s = p_0 ( 1 + \delta_s) = m \beta_s \gamma_s$ of the particle.
The quantity $\delta_s = (p_s - p_0) / p_0$ can be arbitrarily large
and is imposed by the user.
\index{momentum error}
\index{energy error}
This makes the momentum dependence of the expanded Hamiltonian exact
for particles with constant momentum~$p_s$.
For particles with variable momentum,
there is an additional differential momentum error~$p_t$,
whose average over the machine will be close to zero.
This should keep the errors due to the truncation of maps down to an
acceptable level.
 
From the equation of motion in reference~\cite{ISE85} one easily
derives the relations between the slopes $x', y'$ and the normalised
canonical momenta $p_x, p_y$:
\index{slope of orbit}
\index{orbit!slope}
\begin{equation}
x' \approx p_x (1 + h x - p_t\beta_s), \qquad
y' \approx p_y (1 + h x - p_t\beta_s),
\end{equation}
where $h$ is the curvature of the reference orbit in the mid-plane.
The relative energy error $p_t$ is related to the relative momentum
error $\Delta p / p_s$ by
\begin{equation}
p_t = (E/p_s c) - (1/\beta_s) = \Delta E/p_s c
  \approx \beta_s (\Delta p/p_s).
\end{equation}
The special choice of variables affects those second-order terms
in the transfer maps which contain $p_t$.
For constant momentum calculations we have~$p_t = 0$,
and the momentum dependence is correct.
 
 
\section{Phase Space Vector}
\label{phase-space}
\index{phase space}
\index{phase space vector}
MAD normally works on the phase space vector
\begin{equation}
Z=\left(\begin{array}{c}
z_1\\ z_2\\ z_3\\ z_4\\ z_5\\ z_6
\end{array}\right)
=\left(\begin{array}{c}
x\\ p_x\\ y\\ p_y\\ t \\ p_t
\end{array}\right).
\end{equation}
\index{coupling}
and handles coupling effects by means of full $6 \times 6$~matrices.
 
 
\section{Auxiliary Functions and Their Integrals}
\label{integrals}
\index{auxiliary functions}
\index{functions!auxiliary}
\index{integrals}
To simplify notations we introduce the abbreviations
\begin{equation}
\eqarray{
c(k,l)& &                   &=&\cos(kl)&=&\cosh(ikl)\\
s(k,l)&=&\int_0^l c(k,t)\,dt&=&\sin(kl)/k&=&\sinh(ikl)/ik\\
d(k,l)&=&\int_0^l s(k,t)\,dt&=&(1-c(k,l))/k^2\\
f(k,l)&=&\int_0^l d(k,t)\,dt&=&(l-s(k,l))/k^2
}
\end{equation}
The quantities in reference~\cite{SLAC75} are related to the above
quantities as follows:
\begin{equation}\eqarray{
c_x &=& c(k_x,l), \qquad s_x &=& s(k_x,l), \qquad d_x = h d(k_x,l), \\
c_y &=& c(k_y,l), \qquad s_y &=& s(k_y,l)
}\end{equation}
For evaluating transfer maps we shall need the integrals:
\begin{equation}\eqarray{
J_1&=&\int_0^sd_x(t)  \,dt&=&(L-s_x)/k_x^2,\\
J_2&=&\int_0^sd_x^2(t)\,dt&=&(3L-4s_x+s_xc_x)/(2 k_x^4),\\
J_3&=&\int_0^sd_x^3(t)\,dt&=&(15L-22s_x+9s_xc_x-2s_xc_x^2)(6 k_x^6),\\
J_c&=&                    &=&(c(2k_y,s)-c(k_x,s))(k_x^2-4k_y^2), \\
J_s&=&\int_0^sJ_c(t)\,dt  &=&(s(2k_y,s)-s(k_x,s))(k_x^2-4k_y^2), \\
J_d&=&\int_0^sJ_s(t)\,dt  &=&(d(2k_y,s)-d(k_x,s))(k_x^2-4k_y^2), \\
J_f&=&\int_0^sJ_d(t)\,dt  &=&(f(2k_y,s)-f(k_x,s))(k_x^2-4k_y^2).
}\end{equation}
 
 
%==============================================================================
 
\chapter{Hamiltonian Representations}
\label{hamilton}
\index{Hamiltonian}
 
MAD derives most equations of motion from a Hamiltonian.
The Hamiltonian governing particle motion in magnetic elements has
been given in reference~\cite{DRA81}.
Below the reader can find the forms used in MAD for different elements.
 
 
\section{Magnetic Field and Vector Potential (Transverse Field,
  Curved Reference)}
\index{potential!vector}
\index{vector potential}
\index{magnetic field}
\index{field!magnetic}
\index{dipole}
\index{mid-plane symmetry}
\index{Taylor expansion}
For all mid-plane symmetric elements MAD defines the magnetic field on
the mid-plane of a sector dipole by its Taylor expansion:
\begin{equation}
B_x(x,0,s)=0, \qquad
B_y(x,0,s)=B_0+B_1\frac{x}{1!}+B_2\frac{x^2}{2!}+B_3\frac{x^3}{3!}+
\ldots\qquad
B_s(x,0,s)=0.
\end{equation}
For positive $x$ a positive field coefficient gives a contribution to
the field in positive $y$-direction.
The vector potential has a longitudinal component~$A_s$ depending on
\index{potential!vector}
\index{vector potential}
$x, y$ only.
Expanded to order four it takes the form
\begin{equation}\eqarray{
A_x(x,y,s) &=& 0, \\
A_y(x,y,s) &=& 0, \\
A_s(x,y,s) &=&-B_0\left(x-\frac{hx^2}{2(1+hx)}\right)
              -B_1\left(\frac{1}{2}(x^2-y^2)-\frac{h}{6}x^3+
                        \frac{h^2}{24}(4x^4-y^4)+\ldots\right)\\
           & &-B_2\left(\frac{1}{6}(x^3-3xy^2)-\frac{h}{24}(x^4-y^4)+
                        \ldots\right)
              -B_3\left(\frac{1}{24}(x^4-6x^2y^2+y^4)+\ldots\right)+\ldots
}\end{equation}
where $h$ is the curvature of the reference orbit.
\index{curvature}
\index{orbit!reference}
\index{reference orbit}
Taking the curl of $A_s$ in curvilinear coordinates the field
components are to order three
\index{magnetic field}
\index{field!magnetic}
\begin{equation}\eqarray{
B_x(x,y,s)
  &=&+B_1\left(y+\frac{h^2}{6}y^3+\ldots\right)
     +B_2\left(xy-\frac{h^3}{6}y^3+\ldots\right)\\
  & &+B_3\left(\frac{1}{6}(3x^2y-y^3)+\ldots\right)+\ldots\\
B_y(x,y,s)
  &=&+B_0
     +B_1\left(x-\frac{h}{2}y^2+\frac{h^2}{2}xy^2+\ldots\right)
     +B_2\left(\frac{1}{2}(x^2-y^2)-\frac{h}{2}xy^2+\ldots\right)\\
  & &+B_3\left(\frac{1}{6}(x^3-3xy^2)+\ldots\right)+\ldots\\
B_s(x,y,s)&=&0.
}\end{equation}
 
\section{Magnetic Field and Vector Potential (Multipole)}
\index{potential!vector}
\index{vector potential}
\index{magnetic field}
\index{field!magnetic}
\index{multipole}
\index{quadrupole}
\index{sextupole}
\index{octupole}
For zero curvature ($h = 0$) the field coefficients may be complex.
The field expansion then takes the form of a complex Taylor series:
\begin{equation}
B_y(x,y,s) + i B_x(x,y,s) = \sum_{k=0}^\infty C_k \frac{(x + i y)^k}{k!},
\qquad B_s(x,y,s) = 0.
\end{equation}
The vector potential can still be written with a longitudinal
component~$A_s$ only:
\begin{equation}\eqarray{
A_x(x,y,s)=0, \qquad A_y(x,y,s)=0, \qquad A_s(x,y,s)=
  \Re\left(\sum_{k=0}^\infty C_k\frac{(x+i y)^{k+1}}{(k+1)!}\right).
}\end{equation}
 
\section{Magnetic Field and Vector Potential (Solenoid Fields)}
\index{potential!vector}
\index{vector potential}
\index{magnetic field}
\index{field!magnetic}
\index{solenoid}
The magnetic field in a solenoid has the constant value
\begin{equation}
B_x(x,y,s) = 0, \qquad B_y(x,y,s) = 0, \qquad B_s(x,y,s) = B_0.
\end{equation}
The vector potential requires two transverse components:
\begin{equation}
A_x(x,y,s) = - \frac{1}{2} B_0 y, \qquad A_y(x,y,s) = + \frac{1}{2} B_0 x,
\qquad A_s(x,y,s) = 0.
\end{equation}
 
\section{Electric Field for an RF Cavity}
\index{potential!vector}
\index{vector potential}
\index{electric field}
\index{field!electric}
\index{RF cavity}
The voltage in a thin cavity is
\begin{equation}
V = \hat{V} \sin(\phi_s - 2\pi f_{RF}t),
\end{equation}
where $\hat{V}$ is the peak RF~voltage,
$\phi_s$ is the RF~phase relative to a time reference to be defined
below,
and~$f_{RF}$ is the RF~frequency.
The field can be derived from a vector potential
\begin{equation}
A_x(x,y,s)=0, \qquad
A_y(x,y,s)=0, \qquad
A_s(x,y,s)=c\hat{V}\sin(\phi_s-2\pi f_{RF}t)/(2\pi f_{RF}).
\end{equation}
 
\section{Electric Field for an Electrostatic Separator}
\index{potential!scalar}
\index{scalar potential}
\index{electric field}
\index{field!electric}
\index{separator}
The vertical electrostatic field in a separator has the constant value
\begin{equation}
E_x(x,y,s) = 0, \qquad E_y(x,y,s) = E_0, \qquad E_s(x,y,s) = 0.
\end{equation}
The scalar potential depends on~$y$ only:
\begin{equation}
\phi(x,y) = - E_0 y.
\end{equation}
 
\section{General Form for the Hamiltonian}
\index{Hamiltonian}
The general Hamiltonian in a curved reference system with the
curvature~$h$
\begin{equation}
H = - (1 + h x) \left( q A_s +
  \sqrt{ \frac{1}{c} ( E - q \phi )^2 - (mc)^2 - 
         (p_x - q A_x)^2 - (p_y - q A_y)^2} \right),
\end{equation}
is expressed using the arc length~$s$ as the independent variable.
The particle charge is~$q$ and the canonical pairs of variables are
\begin{equation}
(x, \; p_x), \qquad (y, \; p_y), \qquad (E, \; -t).
\end{equation}
The Hamiltonian is normalised in MAD by the following transformations:
\index{Hamiltonian!scaled}
\begin{equation}
H_1 = \frac{H}{p_s}, \qquad
(x_1 = x, \; p_{x1} = \frac{p_x}{p_s}), \qquad
(y_1 = y, \; p_{y1} = \frac{p_y}{p_s}), \qquad
(E_1 = \frac{E}{p_s c}, \; t_1 = c t),
\end{equation}
giving
\begin{equation}
H_1 = - (1 + h x_1) \left( \frac{qA_s}{p_s} +
  \sqrt{ \left( E_1 - \frac{q\phi}{p_s} \right)^2 -
         \frac{1}{\beta_s^2\gamma_s^2} - 
         \left( p_{x1} - \frac{qA_x}{p_s} \right)^2 -
         \left( p_{y1} - \frac{qA_y}{p_s} \right)^2} \right ).
\end{equation}
In the unperturbed machine, and ignoring any momentum change, a
reference particle with constant momentum~$p_s$ would travel with a
constant velocity $v_s = \beta_s c = p_s / m$ on an orbit of the
length $C_s = C (1 + \eta \delta_s)$,
where $C$~is the machine circumference and 
$\eta = \alpha - 1/\gamma_s^2$~is the momentum compaction factor.
Hence it would take the time $T_s = C_s / v_s$ to complete one
revolution.
 
The two variables~$t_1$ and~$E_s$ can take on large values.
\index{Hamiltonian!referred to ideal orbit}
To make a perturbation approach feasible,
the two variables must be replaced by their deviations from a fixed
reference.
To this effect MAD introduces a time reference frame such that
$\partial t_{\hbox{ref}} / \partial s = - T_s / C$.
The generating function
\begin{equation}
F = x_1 p_{x2} + y_1 p_{y2} -
  \left( t_1 + \frac{1 + \eta \delta_s}{\beta_s} s \right)
  \left( p_t + \frac{1}{\beta_s} \right).
\end{equation}
generates a canonical transformation with the following properties:
\begin{itemize}
\item
  The transverse variables are unchanged.
\item
  For a general particle the time difference relative to the reference
  frame is
  \begin{equation}
    t_2 = t_1 + \frac{1 + \eta \delta_s}{\beta_s} s.
  \end{equation}
\item
  The relative energy deviation is defined as
  \begin{equation}
    p_t = E_1 - E_{s1} = \frac{E}{p_s c} - \frac{1}{\beta_s}.
  \end{equation}
\item
  The closed orbit has $t = 0$ before and after one turn,
  i.~e. it also closes in the longitudinal plane.
  The average momentum on the closed orbit is approximately~$p_s$.
\end{itemize}
This canonical transformation generates the new Hamiltonian
\begin{equation}\eqarray{
H_2 &=&-\frac{\partial F}{\partial s}-(1+h x_2)\frac{qA_s}{p_s}\\
    & &-(1+h x_2)\sqrt{\left(p_t-\frac{q\phi}{p_s}+\frac{1}{\beta_s}\right)^2 -
        \frac{1}{\beta_s^2\gamma_s^2}-
        \left(p_{x2}-\frac{qA_x}{p_s}\right)^2-
        \left(p_{y2}-\frac{qA_y}{p_s}\right)^2}\\
    &=&+\frac{1+\eta\delta_s}{\beta_s}\left(p_t+\frac{1}{\beta_s}\right)
       -(1+h x_2)\frac{qA_s}{p_s}\\ 
    & &-(1+h x_2)\sqrt{\left(1+p_t-\frac{q\phi}{p_s}\right)^2-
       \left(p_x-\frac{qA_x}{p_s}\right)^2-
       \left(p_y-\frac{qA_y}{p_s}\right)^2-
       \frac{1}{\beta_s^2\gamma_s^2}\left(p_t-\frac{q\phi}{p_s}\right)^2}.
}\end{equation}
In the latter form the Hamiltonian can easily be expanded as a Taylor
series.
 
 
% ====================================================================
 
\chapter{Lie Algebraic Maps}
\label{LIE}
 
\section{Definitions}
\index{Lie transformation}
\index{map!Lie algebraic}
Let the functions $f(p,q)$ and $g(p,q)$ be differentiable functions of
the canonical variables $p$ and $q$. The Poisson bracket of $f$ and
$g$ is defined as
\begin{equation}
[f,g] = \pbkt{f}{g}.
\end{equation}
The Lie operator associated with $f$ is defined as the Poisson bracket
\begin{equation}
\lieop{f} g = [f,g],
\end{equation}
Iterated Lie operators are defined recursively:
\begin{equation}
\lieop{f}^n g = [f,\lieop{f}^{n-1}g]
\end{equation}
and we also use the abbreviations for iterated Lie operators
\begin{equation}
[f,g,h] = [f,[g,h]], \qquad
[f,g,h,i] = [f,[g,[h,i]]],\qquad
\mathrm{etc.}
\end{equation}
The Lie transformation associated with $f$ is defined as the exponential
\begin{equation}
\lietran{f} g = \sum_{k=0}^\infty \frac{\lieop{f}^k}{k!} g.
\end{equation}
An arbitrary Lie transformation acting on the components of the phase
space vector $Z$ always represents a canonical transformation,
or in other terms, a symplectic map.
In MAD, like in the program MARYLIE,
a Lie algebraic map is represented as the composition of Lie
transformations
\begin{equation}
z^{(2)}_j = \lietran{f_1} \lietran{f_2} \lietran{f_3} \lietran{f_4} \cdots 
  Z^{(1)}_j, \qquad \mathrm{for} \quad j = 1 \ldots 6.
\end{equation}
where each $f_k$ is a homogeneous polynomials of order $k$.
The polynomial $f_2$ generates the ordinary transfer matrix.
It is normally not stored, but replaced by that matrix.
For details refer to~\cite{DOU82,DRA81,HEA86}.
 
Maps for various elements have been derived in~\cite{DOU82,ISE85}.
Note that the signs of energy and time are inverted in MAD with respect
to~\cite{DOU82} and to MARYLIE.
For most elements MAD carries terms up to order four in the
Hamiltonian.
 
 
\section{Tilted Elements}
\index{Lie transformation!tilted element}
\index{tilted element}
\index{map!tilted element}
The effect of the \ttindex{TILT} parameter on an element is that the 
reference system is rotated by the angle $\psi = {\tt TILT}$ at element
entrance, and by $-\psi$ at element exit.
Such a rotation ${\cal R}$ has been described in Section~\ref{srot}.
The transfer map ${\cal F}$ for the element must be transformed to
\begin{equation}
\overline{\cal F} = {\cal R \cal F \cal R}^{-1}.
\end{equation}
 
 
\section{Map Composition}
\index{Lie transformation!composition}
\index{composition!Lie transformation} 
\index{map!composition} 
Let us assume that the two maps ${\cal F} = \{F, f_i\}$ and
${\cal G} = \{G, g_i\}$ occur in this order in a beam line.
The problem is to build their composition as follows:
\begin{equation}
\lietran{f_1} \lietran{f_2} \lietran{f_3} \lietran{f_4} \lietran{f_5}
\lietran{f_6} \ldots
\lietran{g_1} \lietran{g_2} \lietran{g_3} \lietran{g_4} \lietran{g_5}
\lietran{g_6} \ldots\approx
\lietran{h_1} \lietran{h_2} \lietran{h_3} \lietran{h_4} \lietran{h_5}
\lietran{h_6} \ldots = {\cal H} = \{H, h_i\}
\end{equation}
with truncation at a predefined order.
Formulas valid up to order~$6$ have been given in Appendix~B
of~\cite{HEA88}.
These formulas have first been implemented in MARYLIE~5.1 and have
been copied and modified for use in MAD.
The formulas are listed here for reference;
for the theory refer to~\cite{HEA88}.
 
Using the exchange formula
\begin{equation}
e^{\lieop{f}} e^{\lieop{g}} = e^{\lieop{g}}
e^{\lieop{\exp{\lieop{-g}}f}}
\end{equation}
composition is done in three steps.
First consider the problem of moving the first-order term~$g_1$ to the
left:
\begin{equation}
\lietran{f_1} \lietran{f_2} \lietran{f_3} \lietran{f_4} \lietran{f_5}
\lietran{f_6}
\lietran{g_1} \lietran{g_2} \lietran{g_3} \lietran{g_4} \lietran{g_5}
\lietran{g_6} = 
\lietran{h_1} \lietran{f_2} \lietran{t_2} \lietran{t_3} \lietran{t_4}
\lietran{t_5} \lietran{t_6}
\lietran{g_2} \lietran{g_3} \lietran{g_4} \lietran{g_5}
\lietran{g_6}\end{equation}
To move~$g_1$ successively over the polynomials~$f_i$ we define:
\begin{equation}
j_m^{(n)} = \frac{1}{(n-m)!} \lieop{-g_1}^{(n-m)} f_n,
  \qquad m = 0, 1, \ldots , n, \qquad n = 6, 5, 4, 3.
\end{equation}
and then we regroup the various Lie transformations arising:
\begin{equation}\eqarray{
k_1^{(3)} &=& j_1^{(3)}
           +  \frac{1}{2} [j_2^{(3)},j_1^{(3)}]
           -  \frac{1}{6} [j_1^{(3)},j_3^{(3)},j_1^{(3)}]
           +  \frac{1}{6} [j_2^{(3)},j_2^{(3)},j_1^{(3)}] \\
          &-& \frac{1}{8} [j_1^{(3)},j_3^{(3)},j_2^{(3)},j_1^{(3)}]
           -  \frac{1}{24}[j_2^{(3)},j_1^{(3)},j_3^{(3)},j_1^{(3)}]
           +  \frac{1}{24}[j_2^{(3)},j_2^{(3)},j_2^{(3)},j_1^{(3)}] \\
k_2^{(3)} &=& j_2^{(3)}
           +  \frac{1}{2} [j_3^{(3)},j_1^{(3)}]
           -  \frac{1}{12}[j_2^{(3)},j_3^{(3)},j_1^{(3)}]
           +  \frac{1}{6} [j_3^{(3)},j_2^{(3)},j_1^{(3)}] \\
          &-& \frac{1}{24}[j_2^{(3)},j_3^{(3)},j_2^{(3)},j_1^{(3)}]
           -  \frac{1}{24}[j_3^{(3)},j_1^{(3)},j_3^{(3)},j_1^{(3)}]
           +  \frac{1}{24}[j_3^{(3)},j_2^{(3)},j_2^{(3)},j_1^{(3)}] \\
k_3^{(3)} &=& j_3^{(3)}
           +  \frac{1}{2} [j_3^{(3)},j_2^{(3)}]
           -  \frac{1}{6} [j_2^{(3)},j_3^{(3)},j_2^{(3)}]
           +  \frac{1}{6} [j_3^{(3)},j_3^{(3)},j_1^{(3)}]
           +  \frac{1}{24}[j_2^{(3)},j_2^{(3)},j_3^{(3)},j_2^{(3)}] \\
          &-& \frac{1}{8} [j_2^{(3)},j_3^{(3)},j_3^{(3)},j_1^{(3)}]
           +  \frac{1}{24}[j_3^{(3)},j_2^{(3)},j_3^{(3)},j_1^{(3)}]
           +  \frac{1}{24}[j_3^{(3)},j_3^{(3)},j_2^{(3)},j_1^{(3)}] \\   
k_4^{(3)} &=&-\frac{1}{12}[j_3^{(3)},j_3^{(3)},j_2^{(3)}]
           +  \frac{1}{24}[j_3^{(3)},j_2^{(3)},j_3^{(3)},j_2^{(3)}]
           -  \frac{1}{24}[j_3^{(3)},j_3^{(3)},j_3^{(3)},j_1^{(3)}] \\
k_5^{(3)} &=& \frac{1}{24}[j_3^{(3)},j_3^{(3)},j_3^{(3)},j_2^{(3)}] \\
}\end{equation}
\begin{equation}\eqarray{
k_1^{(4)} &=& j_1^{(4)}
           +  \frac{1}{2} [j_2^{(4)},j_1^{(4)}] \\
k_2^{(4)} &=& j_2^{(4)}
           +  \frac{1}{2} [j_3^{(4)},j_1^{(4)}] \\
k_3^{(4)} &=& j_3^{(4)}
           +  \frac{1}{2} [j_3^{(4)},j_2^{(4)}]
           +  \frac{1}{2} [j_4^{(4)},j_1^{(4)}] \\
k_4^{(4)} &=& j_4^{(4)}
           +  \frac{1}{2} [j_4^{(4)},j_2^{(4)}] \\
k_5^{(4)} &=& \frac{1}{2} [j_4^{(4)},j_3^{(4)}] \\
}\end{equation}
\begin{equation}\eqarray{
k_m^{(5)} &=& j_m^{(5)} \\
k_m^{(6)} &=& j_m^{(6)} \\
}\end{equation}
and get the final result:
\begin{equation}\eqarray{
h_1 &=& f_1 + e^{\lieop{f_2}}
    (g_1 + k_1^{(3)} + k_1^{(4)} + k_1^{(5)} + k_1^{(6)}) \\
t_2 &=& k_2^{(3)} + k_2^{(4)} + k_2^{(5)} + k_2^{(6)}
     +  \frac{1}{2} [k_2^{(3)},k_2^{(4)} + k_2^{(5)}]
     +  \frac{1}{12}[k_2^{(3)},k_2^{(3)},k_2^{(4)}] \\
t_3 &=& k_3^{(3)} + k_3^{(4)} + k_3^{(5)} + k_3^{(6)}
     +              [k_3^{(3)},k_2^{(4)} + k_2^{(5)}] \\
t_4 &=& k_4^{(3)} + k_4^{(4)} + k_4^{(5)} + k_4^{(6)}
     +  \frac{1}{2} [k_3^{(3)},k_3^{(4)} + k_3^{(5)}] \\
t_5 &=& k_5^{(3)} + k_5^{(4)} + k_5^{(5)} + k_5^{(6)}
     -  \frac{1}{6} [k_3^{(3)},k_3^{(3)},k_3^{(4)}], \\
t_6 &=& k_6^{(6)} \\
}\end{equation}
 
The second step converts $t_2$ to a matrix and combines it with the
matrices~$F$ and~$G$.
We note that
\begin{equation}
t_2 = \frac{1}{2} Z^t W Z, \qquad W \mathrm{\ symmetric},
\end{equation}
is of small order~3, and that
\begin{equation}
\lieop{t_2} Z = J W Z.
\end{equation}
Hence, using the approximation $\tanh (x) \approx x - x^3/6$,
we define a symplectic matrix $T$ by
\begin{equation}\eqarray{
\exp(\lieop{t_2})&=&\exp(JW)Z=(I+\tanh(JW/2))(I-\tanh(JW/2))^{-1} \\
&\approx&TZ=(I+(W/2-W^3/24))(I-(W/2-W^3/24))^{-1}Z.
}\end{equation}
In the third step the remaining problem is to convert
\begin{equation}
\lietran{h_1} \lietran{h_2}
\lietran{u_3} \lietran{u_4} \lietran{u_5} \lietran{u_6}
\lietran{g_3} \lietran{g_4} \lietran{g_5} \lietran{g_6}
= \lietran{h_1} \lietran{h_2}
\lietran{h_3} \lietran{h_4} \lietran{h_5} \lietran{h_6}.
\end{equation}
Due to the exchange formula the $t$ polynomials are converted to
\begin{equation}
u_k = t_k(\lieop{-g_2} Z), \quad k = 3, 4, 5, 6.
\end{equation}
The solution is
\begin{equation}\eqarray{
h_3 &=& f_3 + u_3 \\
h_4 &=& f_4 + u_4 + \frac{1}{2} [f_3,u_3] \\
h_5 &=& f_5 + u_5 - [u_3,f_4] - \frac{1}{6} \lieop{f_3}^2 u_3
     +  \frac{1}{3} \lieop{u_3}^2 f_3 \\
h_6 &=& f_6 + u_6 - [u_3,f_5] + \frac{1}{2} \lieop{u_3}^2 f_4
     +  \frac{1}{2} [f_4,u_4] - \frac{1}{4} [f_4,f_3,u_3] \\
    &-& \frac{1}{4} [u_4,f_3,u_3] + \frac{1}{12} \lieop{f_3}^3 u_3
     -  \frac{1}{8} \lieop{u_3}^3 f_3 + \frac{1}{8} [f_3,u_3,f_3,u_3].
}\end{equation}
 
 
\section{Reverse Factorisation}
\index{Lie transformation!reverse factorisation}
\index{reverse factorisation!Lie transformation}
\index{map!reverse factorisation}
MAD requires only the case where $f_1$ vanishes and the order is four:
\begin{equation}
{\cal F}=\lietran{f_2}\lietran{f_3}\lietran{f_4}
=\lietran{g_4}\lietran{g_3}\lietran{g_2},
\end{equation}
where the second-order polynomials are represented by the
corresponding matrices.
The result is:
\begin{equation}
G = F, \qquad g_3(Z) = f_3(F Z), \qquad g_4(Z) = g_4(F Z).
\end{equation}
 
 
\section{Map Inversion}
\index{Lie transformation!inversion}
\index{inversion!Lie transformation}
\index{map!inversion}
MAD requires only the case where $f_1$ vanishes and the order is four.
The inverse of a Lie algebraic map ${\cal F}$ is found in two steps.
First the factorisation is reversed as shown in the previous subsection.
The second step uses the formula 
$\left(\lietran{f}\right)^{-1} = \lietran{-f}$
to find the inverse map:
\begin{equation}
{\cal F}^{-1}=\lietran{-g_2}\lietran{-g_3}\lietran{-g_4}
\end{equation}
The matrix of the inverse map is thus~$F^{-1}$,
and the polynomials are the negative of the ones found for the reverse
factorisation.
 
 
\section{Reflection of a Lie Algebraic Map}
\index{Lie transformation!reflection}
\index{reflection!Lie transformation}
\index{map!reflection}
Reflection of a transfer map is the transformation needed to simulate
traversal through a beam line in reverse direction.
Note that this also reverses asymmetric elements.
This transformation is equivalent to inversion of the transfer map,
followed by a change of sign for the variables $p_x, p_y$ and $t$.
 
 
\section{Fixed Points}
\index{Lie transformation!fixed point}
\index{fixed point!Lie transformation}
\index{map!fixed point}
Starting with an arbitrary initial approximation~$Z_0$,
an iterative procedure can be defined as follows:
\begin{enumerate}
\item
Define a first-order polynomial~$g_1$ such that
\begin{equation}
Z_0 = [g_1, Z].
\end{equation}
This polynomial represents a map which sends  the origin to the
initial approximation.
\item
Compose the map $\cal F$ with the ``map''~$g_1$ to get~$h_1$
according to the composition algorithm.
$h_1$ then maps the origin onto the orbit at the end of the system.
The matrix~$H$ is the Jacobian of this map and can be used to find
a new approximation:
\begin{equation}
Z_1 = Z_0 - H^{-1} (\lieop{h_1} Z - Z_0).
\end{equation}
\item
Repeat, until convergence is achieved.
\end{enumerate}
Note that MAD now finds the fixed point by tracking through each
element individually, since the above procedure may not give good
results when high-order terms resulting from concatenation become
important.
 
 
\section{Tracking}
\index{Lie transformation!tracking}
\index{tracking!Lie method}
When one of the options
\myxmp{METHOD=LIE3|LIE4}
occurs on a {\tt RUN} command, MAD uses the Lie-algebraic tracking
method up to $f_3$~or~$f_4$ terms, respectively.
For {\tt LUMP}s it always uses the order specified on their
definition.
Note that truncating the exponential series
\begin{equation}
e^{\lieop{f}} Z = \sum_{k=0}^\infty \frac{{\lieop{f}}^k}{k!}
\end{equation}
at a finite $k$ {\em does not} produce a symplectic map.
MAD therefore tracks the linear terms using the transfer matrix.
For the non-linear terms~$f_n$ it uses a generating function of the
form
\begin{equation}
G(q_1,p_2) = q_1 p_2 + \sum_{k=3}^6 g_k(q_1,p_2).
\end{equation}
This function is set up such that the resulting canonical
transformation agrees to the desired order with the mapping.
With the temporary values
\begin{equation}\eqarray{
t_4&=&\sum_{m=1}^3
    \frac{\partial f_3}{\partial q_m}
    \frac{\partial f_3}{\partial p_m}, \\
t_5&=&\sum_{m=1}^3
    \frac{\partial f_3}{\partial q_m}
    \frac{\partial t_4}{\partial p_m}
+ \sum_{l=1}^3\sum_{m=1}^3
    \frac{\partial^2f_3}{\partial q_{l}\partial q_m}
    \frac{\partial f_3}{\partial p_{l}}
    \frac{\partial f_3}{\partial p_m}, \\
t_6&=&\sum_{m=1}^3
    \frac{\partial f_4}{\partial q_m}
    \frac{\partial t_4}{\partial p_m}
+ \sum_{l=1}^3\sum_{m=1}^3
    \frac{\partial^2 f_4}{\partial q_{l}\partial q_m}
    \frac{\partial f_3}{\partial p_{l}}
    \frac{\partial f_3}{\partial p_m}, \\
u_6&=&\sum_{m=1}^3
    \frac{\partial f_3}{\partial q_m}
    \frac{\partial t_5}{\partial p_m}
+ 3\sum_{l=1}^3\sum_{m=1}^3
    \frac{\partial^2 f_3}{\partial q_{l}\partial q_m}
    \frac{\partial f_3}{\partial p_{l}}
    \frac{\partial t_4}{\partial p_m}
+ \sum_{k=1}^3\sum_{l=1}^3\sum_{m=1}^3
    \frac{\partial^3 f_3}{\partial q_k\partial q_{l}\partial q_m}
    \frac{\partial f_3}{\partial p_k}
    \frac{\partial f_3}{\partial p_{l}}
    \frac{\partial f_3}{\partial p_m}, \\
}\end{equation}
it can be written as
\begin{equation}\eqarray{
g_3 &=& f_3, \\ 
g_4 &=& f_4 + \frac{1}{2} t_4, \\
g_5 &=& f_5 - \frac{1}{6} t_5
+ \frac{1}{2}\sum_{m=1}^3
    \frac{\partial f_4}{\partial q_k}
    \frac{\partial f_3}{\partial p_k}, \\
g_6 &=& f_6 + \frac{1}{24} t_6 - \frac{1}{2} u_6
+ \sum_{m=1}^3
    \frac{\partial f_5}{\partial q_k}
    \frac{\partial f_3}{\partial p_k}
+ \frac{1}{2}\sum_{m=1}^3
    \frac{\partial f_4}{\partial q_k}
    \frac{\partial f_4}{\partial p_k}. \\
}\end{equation}
The generating function is truncated at the order of the
Lie transformation given.
 
 
% ====================================================================
 
\chapter{TRANSPORT Maps}
\label{transport}
 
\section{Definition}
\index{TRANSPORT map}
\index{map!TRANSPORT form}
A TRANSPORT map (see~\cite{SLAC75,SLAC91}) is the Taylor series
for the exact transfer map, truncated at order two:
\begin{equation}
z^{(2)}_j = \Delta z_j + \sum_{k=1}^6 R_{jk} z^{(1)}_k
          + \sum_{k=1}^6 \sum_{l=1}^6 T_{jkl} z^{(1)}_k z^{(1)}_l,
          \qquad \mathrm{for} \quad j = 1 \ldots 6.
\end{equation}
The $T_{jkl}$ array is symmetric with respect to its second and
third index.
Both indices run from 1~to~6, and by convention the off-diagonal terms
in MAD are half those used in TRANSPORT.
Below we list only non-zero elements for $k \le l$ to save space.
 
Due to truncation a TRANSPORT map is symplectic only in exceptional cases.
TRANSPORT maps have been derived for many elements in~\cite{SLAC75}.
However, due to different variables,
the second-order terms are changed as explained in~\cite{ISE85}.
Reference~\cite{ISE85} also gives the formula to derive a TRANSPORT
map from a Lie-algebraic map (see Chapter~\ref{LIE}):
\begin{equation}\eqarray{
T_{1kl}&=&-\frac{1}{2}\sum_{m=1}^6\sum_{n=1}^6F_{2mn}R_{mk}R_{nl},\qquad
T_{2kl}&=&+\frac{1}{2}\sum_{m=1}^6\sum_{n=1}^6F_{1mn}R_{mk}R_{nl},\\
T_{3kl}&=&-\frac{1}{2}\sum_{m=1}^6\sum_{n=1}^6F_{4mn}R_{mk}R_{nl},\qquad
T_{4kl}&=&+\frac{1}{2}\sum_{m=1}^6\sum_{n=1}^6F_{3mn}R_{mk}R_{nl},\\
T_{5kl}&=&-\frac{1}{2}\sum_{m=1}^6\sum_{n=1}^6F_{6mn}R_{mk}R_{nl},\qquad
T_{6kl}&=&+\frac{1}{2}\sum_{m=1}^6\sum_{n=1}^6F_{5mn}R_{mk}R_{nl}.
}\end{equation}
 
 
\section{Tilted Elements}
\index{TRANSPORT map!tilted element}
\index{tilted element}
\index{map!tilted element}
The effect of the \ttindex{TILT} parameter on an element is that the 
reference system is rotated by the angle $\psi = \ttindex{TILT}$ at element
entrance, and by $-\psi$ at element exit.
Such a rotation ${\cal R}$ has been described in Section~\ref{srot}.
The transfer map ${\cal F}$ for the element must be transformed to
\begin{equation}
\overline{\cal F} = {\cal R \cal F \cal R}^{-1}.
\end{equation}
 
 
\section{Map Composition}
\index{TRANSPORT map!composition}
\index{map!composition}
\index{composition!TRANSPORT map}
Assume that the two maps ${\cal A} = \{R^a, T^a\}$ and
${\cal B} = \{R^b, T^b\}$ occur in this order in the beam.
The transfer matrix for the composition
${\cal C} = {\cal B A} = \{R^c, T^c\}$ is
\begin{equation}
R^c = R^b R^a
\end{equation}
By substitution of~${\cal A}$ in~${\cal B}$ and truncation at second
order one finds the second-order terms of ${\cal C}$:
\begin{equation}
T^c_{ijk} = \sum_{l=1}^6 R^b_{il} T^a_{l jk} +
  \sum_{l=1}^6 \sum_{m=1}^6 T^b_{il m} R^a_{l j} R^a_{mk}.
\end{equation}
 
 
\section{Map Inversion}
\index{TRANSPORT map!inversion}
\index{map!inversion}
\index{inversion!TRANSPORT map}
To first order a TRANSPORT map is inverted by inverting its transfer 
matrix.
Since the matrix is symplectic its inverse can be found by the formula
\begin{equation}
R^{-1} = - S R^T S,
\end{equation}
where the superscript~$T$ denotes the transpose and the matrix $S$ is
the symplectic unit matrix
\begin{equation}
S=\left(\myarray{
 0 & 1 & 0 & 0 & 0 & 0 \\-1 & 0 & 0 & 0 & 0 & 0 \\
 0 & 0 & 0 & 1 & 0 & 0 \\ 0 & 0 &-1 & 0 & 0 & 0 \\
 0 & 0 & 0 & 0 & 0 & 1 \\ 0 & 0 & 0 & 0 &-1 & 0
}\right)
\end{equation}
The composition of a map and its inverse must reproduce the identity map.
The equations for this condition may be solved and give:
\begin{equation}
T^{-1}_{ijk} = - \sum_{l=1}^6 \sum_{m=1}^6 \sum_{n=1}^6
  R^{-1}_{il} T_{l mn} R^{-1}_{mj} R^{-1}_{nk}.
\end{equation}
 
 
\section{Map Reflection}
\index{TRANSPORT map!reflection}
\index{map!reflection}
\index{reflection!TRANSPORT map}
The reflection $\overline{\cal T} = \{\overline R, \overline T\}$ of a
transfer map ${\cal T} = \{R,T\}$ represents the traversal of
a beam line in inverse order. Note that this also reverses
asymmetric elements like RF~cavities. Care must be taken if such
elements occur, since this may not be the desired effect.
 
To compute the reflection, first the formulas of the previous section
are applied.
The signs of all coefficients having an odd number of occurrences
of 2,~4,~or~5 in their indices must be inverted.
The result is:
\begin{equation}
\overline R_{ij} = s_i s_j (R^{-1})_{ij}, \qquad
\overline T_{ijk} = s_i s_j s_k (T^{-1})_{ijk},
\end{equation}
where $s_1 = s_3 = s_6 = 1, s_2 = s_4 = s_5 = -1$.
 
 
\section{Closed Orbit}
\index{TRANSPORT map!closed orbit}
\index{TRANSPORT map!fixed point}
\index{fixed point!TRANSPORT map}
\index{map!fixed point}
\label{s-co}
The closed orbit is the first-order fixed point of the transfer map
for one turn around the machine.
MAD searches for the closed orbit along the following steps:
 
\begin{enumerate}
\item 
Set the initial guess to zero for the transverse phase space
coordinates and to the specified energy error for $\Delta E/p_0c$:
\begin{equation}
Z_0 = \left ( \eqarray{
x_0 \\ p_x0 \\ y_0 \\ p_y0 \\ ct_0 \\ \delta_0
} \right )
= \left ( \myarray{
0 \\ 0 \\ 0 \\ 0 \\ 0 \\ \Delta p/p_0c
} \right ).
\end{equation}
 
\item Find the orbit $Z_1$ after one turn and the Jacobian $R$ of
the map for one turn.
 
\item Use the Jacobian to find a correction to the initial conditions.
The transverse coordinates must close.
 
For dynamic maps (including RF~cavities and synchrotron radiation)
the flight time is constrained such as to give the specified energy
error on average.
This leads to the conditions
\begin{equation}\eqarray{
x_1 &=& x_0 &\qquad p_{x1} &=& p_{x0} \\
y_1 &=& y_0 &\qquad p_{y1} &=& p_{y0} \\
c\cdot t_1 &=& c\cdot t_0 + \frac{C \alpha}{\beta_s} \frac{\Delta p}{p_0 c}
&\qquad \delta_1&=& \delta_0
}\end{equation}
where $C$ is the machine circumference, $\alpha$ is the momentum
compaction, and $\beta_s$ is the ratio of the particle velocity to the
velocity of light.
The equation for one iteration affects all three degrees of
freedom and reads:
\begin{equation}
Z_1 + R_j \Delta Z = Z_0 + \Delta Z \qquad \Rightarrow \qquad
\Delta Z = - (R_j - I)^{-1} (Z_1 - Z_0).
\end{equation}
Note that when searching for the closed orbit in an \ttindex{OPTICS}
command the flight time difference is always zero.
Thus only $\delta=0$ is permitted.
 
For static maps (constant energy around the ring) there is no
condition on the flight time difference, and the energy error is
constant.
Thus the equations affect the transverse degrees of freedom only. 
 
\item Iterate steps~2 and~3 until convergence is achieved.
\end{enumerate} 
 
When the machine is strongly non-linear, the convergence of this
algorithm may be bad.
In this case MAD first attempts to find the closed orbit with reduced
sextupole strengths.
This may still fail if the RF~phase lags are set such that the zero
orbit is close to the unstable fixed point.
 
 
\section{Making First-Order Matrix Symplectic}
\index{Transfer matrix!symplectification}
\index{symplectification}
\label{symplectify}
The Jacobian matrix of a TRANSPORT map is only approximately symplectic,
whereas the theory requires an exactly symplectic matrix.
An elegant method to make a matrix symplectic has been given in~\cite{HEA86}.
A symplectic matrix~$F$ can be written as $\exp(SM)$ with a
symmetric~$M$.
We may rewrite this as
\begin{equation}
F = (I + \tanh(SM/2))(I - \tanh(SM/2))^{-1} = (I + W)(I - W)^{-1},
\end{equation}
where $W$~is symmetric if and only if $F$~is symplectic.
Given a matrix~$F$ which is approximately symplectic,
we define the matrix
\begin{equation}
V = S (I - F)(I + F)^{-1}.
\end{equation}
which is approximately symmetric.
Using the exactly symmetric matrix $W = (V + V^T) / 2$ we generate an
exactly symplectic matrix from the previous equation.
 
 
\section{Tracking}
\index{TRANSPORT map!tracking}
\index{tracking!TRANSPORT map}
Tracking by the TRANSPORT method is straightforward.
It uses the definition of the TRANSPORT map:
\begin{equation}
z^{(2)}_j = \Delta z_j + \sum_{k=1}^6 R_{jk} z^{(1)}_k
          + \sum_{k=1}^6 \sum_{l=1}^6 T_{jkl} z^{(1)}_k z^{(1)}_l,
          \qquad \mathrm{for} \quad j = 1 \ldots 6.
\end{equation}
This method is used by default, or if the option
\myxmp{METHOD=TRANSPORT}
is seen on a {\tt RUN} command.
For magnets defined as thin multipoles the thin lens map is used.
Note that the TRANSPORT map tracking is not symplectic;
for long-term tracking it most certainly causes spurious blow up or
shrinking.
For long-term tracking it is recommended to use the Lie-algebraic
methods, or if CPU time is at premium, thin lens tracking.
 
 
%==============================================================================
 
\chapter{Maps for Physical Elements}
\label{element}
\index{map!element}
\index{element!map}
 
\section{Marker}
\index{marker}
The \ttindex{MARKER} element has no transfer map.
It is ignored during optical calculations.
 
 
\section{Drift Space and Similar Objects}
There are seven drift-like elements in MAD:
\begin{itemize}
\item {\tt DRIFT}, Ordinary drift space,\index{drift}
\item {\tt ECOLLIMATOR}, Elliptic collimator, \index{collimator}
\item {\tt RCOLLIMATOR}, Rectangular collimator,
\item {\tt INSTRUMENT}, Beam instrumentation,  \index{instrument}
\item {\tt MONITOR}, Monitor for both planes, \index{monitor}
\item {\tt HMONITOR}, Monitor for horizontal plane,
\item {\tt VMONITOR}, Monitor for vertical plane.
\end{itemize}
All these element types act as field-free regions.
A beam position monitor also stores the position of the beam in its
centre,
and a collimator limits the aperture during tracking.
 
\subsection{Hamiltonian for a Drift Space}
\index{drift!Hamiltonian}
\index{Hamiltonian!drift}
The drift space has no field.  Its exact Hamiltonian is
\begin{equation}
H = - \sqrt{\left(1+\frac{p_t}{\beta_s}\right)-\aux}+
\frac{\eta \delta_s}{\beta_s} p_t.
\end{equation}
 
\subsection{Exact Solution for Equations of Motion}
\index{drift!exact motion}
\index{exact motion!drift}
The exact solution for the equations of motion is
\begin{equation}\eqarray{
x_2 &=& x_1 + p_{x1} \Biggm/ \sqrt{\left(1+\frac{p_t}{\beta_s}\right)-\aux}, &
\qquad p_{x2} &=& p_{x1}, \\
y_2 &=& y_1 + p_{y1} \Biggm/ \sqrt{\left(1+\frac{p_t}{\beta_s}\right)-\aux}, &
\qquad p_{y2} &=& p_{y1}, \\
t_2 &=& t_1 + \frac{L\eta\delta_s}{\beta_s} -
(1 + \frac{p_{t1}}{\beta_s}) \Biggm/
\sqrt{\left(1+\frac{p_t}{\beta_s}\right)-\aux},
&\qquad p_{t2} &=& p_{t1}.
}\end{equation}
This map is not used in MAD.
 
\subsection{Lie-Algebraic Map for a Drift Space}
\index{drift!Lie transformation}
\index{Lie transformation!drift}
\index{map!drift}
The Lie-algebraic map is found by the techniques of
reference~\cite{DOU82}.
The linear part is represented by the transfer matrix
\begin{equation}
R=\left(\myarray{
1 & L & 0 & 0 & 0 & 0 \\
0 & 1 & 0 & 0 & 0 & 0 \\
0 & 0 & 1 & L & 0 & 0 \\
0 & 0 & 0 & 1 & 0 & 0 \\
0 & 0 & 0 & 0 & 1 & \frac{L}{\beta_s^2\gamma_s^2} \\
0 & 0 & 0 & 0 & 0 & 1 \\
}\right ).
\end{equation}
and the generators are
\begin{equation}\eqarray{
f_1&=&-\frac{\eta\delta_s}{\beta_s} p_t, \\
f_3&=&\frac{L}{2\beta_s}\aux p_t,\\
f_4&=&-\frac{L}{2\beta_s^2}\aux p_t^2-\frac{L}{8}\aux^2.
}\end{equation}
 
\subsection{TRANSPORT Map}
\index{drift!TRANSPORT map}
\index{TRANSPORT map!drift}
\index{map!drift}
The TRANSPORT map for a drift-like element can be derived by
differentiation of the exact form, or from the Lie-algebraic map:
\begin{equation}\eqarray{
x_2 &=& x_1 + L p_{x1} \left( 1 - \frac{p_t}{\beta_s} \right), &
\qquad p_{x2} &=& p_{x1}, \\
y_2 &=& y_1 + L p_{y1} \left( 1 - \frac{p_t}{\beta_s} \right), &
\qquad p_{x2} &=& p_{x1}, \\
t_2 &=& t_1 + \frac{L\eta\delta_s}{\beta_s} +
\frac{p_{t1}}{\beta_s^2\gamma_2^2} - \frac{L}{2\beta_s}
\left(p_{x1}^2+p_{y1}^2+\frac{3p_{t1}^2}{\beta_s^s\gamma_s^2}\right), &
\qquad p_{t2} &=& p_{t1}.
}\end{equation}
 
 
\section{Dipoles}
\label{dipole}
Transfer maps for dipoles in MAD are composed of three maps,
namely the fringing field at the magnet entrance ${\cal F}^{(1)}$,
the body of the dipole ${\cal B}$, 
and the fringing field at the magnet exit ${\cal F}^{(2)}$:
\begin{equation}
{\cal F} = {\cal F}^{(1)} {\cal B} {\cal F}^{(2)}.
\end{equation}
MAD presently (still) treats all dipoles as {\tt SBEND}s,
but for an {\tt RBEND} it uses an additional pole face rotation angle 
equal to half the bend angle.
 
\subsection{Dipole Fringing Fields}
The TRANSPORT map for a fringing field has been derived in~\cite{SLAC75},
and the required change of variables,
together with an equivalent Lie transformation,
is described in~\cite{ISE85}.
Let the pole-face rotation angles at entrance and exit be~$\psi_1$
and~$\psi_2$ respectively,
and the curvature of the pole faces be described by the radii~$R_1$
and~$R_2$.
For a dipole of type \ttindex{RBEND} half the bend angle is added to each
of the~$\psi_i$.
If the fringing fields have a finite extent the vertical focussing
angle is changed according to~\cite{SLAC75}:
\begin{equation}
\overline{\psi_i} = \psi_i - h g I_1 (1 + \sin^2\psi_i).
\end{equation}
$h$~is the curvature of the reference orbit within the dipole,
$g$~is the {\em full} gap height, 
and $I_1$~is the first {\em fringing field integral}
\begin{equation}
I_1=\int_{-\infty}^{\infty}\frac{B_y(s)(B_0-B_y(s))}{g\cdot B_0^2}\,ds.
\end{equation}
 
\subsubsection{Lie-Algebraic Maps for Dipole Fringing Fields}
\index{fringing field!Lie transformation}
\index{Lie transformation!fringing field}
\index{map!fringing field}
For both entrance and exit the transfer matrix is
\begin{equation}
F^{(i)}=\left(\myarray{
 1           &0 &0                       &0 &0 &0 \\
+h\tan\psi_i &1 &0                       &0 &0 &0 \\
 0           &0 &1                       &0 &0 &0 \\
 0           &0 &-h\tan\overline{\psi_i} &1 &0 &0 \\
 0           &0 &0                       &0 &1 &0 \\
 0           &0 &0                       &0 &0 &1 
}\right).
\end{equation}
The generator~$f_3$ has been slightly improved with respect to
reference~\cite{ISE85}.
At the magnet entrance, using the quadrupole coefficient $K_1$ for the
magnet body, it takes the value
\begin{equation}\eqarray{
f^{(1)}_3&=&\frac{1}{6}\left(\frac{h}{R_1}\sec^3\psi_1 + 2 K_1\tan\psi_1 -
      2h^2\tan^3\psi_1\right) x^3 \\
   &-&\frac{1}{2}\left(\frac{h}{R_1}\sec^3\psi_1 + 2 K_1\tan\psi_1 -
      h^2\tan\psi_1(\sec^2\psi_1 - \tan^2\overline{\psi_1})\right) x y^2 \\
   &+&\frac{h}{2}\tan\psi_1\Big(x^2p_x\tan\psi_1 -
      2xyp_y\tan\overline{\psi_1}\Big) - \frac{h}{2} p_x y^2 \sec^2\psi_1.
}\end{equation}
and for the exit it is
\begin{equation}\eqarray{
f^{(2)}_3&=&\frac{1}{6}\left(\frac{h}{R_2}\sec^3\psi_2 + 2 K_1\tan\psi_2 +
      h^2\tan^3\psi_2\right) x^3 \\
   &-&\frac{1}{2}\left(\frac{h}{R_2}\sec^3\psi_2 + 2 K_1\tan\psi_2 -
      h^2\tan\psi_2\tan^2\overline{\psi_2}\right) x y^2 \\
   &-&\frac{h}{2}\tan\psi_2\Big(x^2p_x\tan\psi_2 -
      2xyp_y\tan\overline{\psi_2}\Big) + \frac{h}{2} p_x y^2 \sec^2\psi_2.
}\end{equation}
The generator $f_4$ is not used by MAD.
 
\subsubsection{TRANSPORT Map for Dipole Fringing Fields}
\index{fringing field!TRANSPORT map}
\index{TRANSPORT map!fringing field}
\index{map!fringing field}
For both entrance and exit the transfer matrix is
\begin{equation}
R=\left(\myarray{
1            &0 &0                       &0 &0 &0 \\
+h\tan\psi_i &1 &0                       &0 &0 &0 \\
0            &0 &1                       &0 &0 &0 \\
0            &0 &-h\tan\overline{\psi_i} &1 &0 &0 \\
0            &0 &0                       &0 &1 &0 \\
0            &0 &0                       &0 &0 &1 
}\right).
\end{equation}
The second-order terms for the entrance are
\begin{equation}\eqarray{
T_{111}&=&T_{234}&=&T_{414}&=&-\frac{h}{2}\tan^2\psi_1, \\
T_{212}&=&T_{313}&=&&&+\frac{h}{2}\tan^2\psi_1, \\
T_{133}&=&&&&&+\frac{h}{2}\sec^2\psi_1, \\
T_{423}&=&&&&&-\frac{h}{2}\sec^2\psi_1, \\
T_{211}&=&&&&&+\frac{h}{2R_1}\sec^3\psi_1+K_1\tan\psi_1, \\
T_{233}&=&&&&&-\frac{h}{2R_1}\sec^3\psi_1-K_1\tan\psi_1+
               \frac{h^2}{2}\tan\psi_1(1+\sec^2\psi_1),\\
T_{413}&=&&&&&-\frac{h}{2R_1}\sec^3\psi_1-K_1\tan\psi_1. \\
}\end{equation}
and for the exit
\begin{equation}\eqarray{
T_{111}&=&T_{234}&=&T_{414}&=&+\frac{h}{2}\tan^2\psi_2, \\
T_{212}&=&T_{313}&=&&&-\frac{h}{2}\tan^2\psi_2, \\
T_{133}&=&&&&&-\frac{h}{2}\sec^2\psi_2, \\
T_{423}&=&&&&&+\frac{h}{2}\sec^2\psi_2, \\
T_{211}&=&&&&&+\frac{h}{2R_2}\sec^3\psi_2+K_1\tan\psi_2-
               \frac{h^2}{2}\tan^3\psi_2, \\
T_{233}&=&&&&&-\frac{h}{2R_2}\sec^3\psi_2-K_1\tan\psi_2-
               \frac{h^2}{2}\tan^3\psi_2, \\
T_{413}&=&&&&&-\frac{h}{2R_2}\sec^3\psi_2-K_1\tan\psi_2+
               \frac{h^2}{2}\tan\psi_2\sec^2\psi_2. \\
}\end{equation}
 
\subsection{Body of a Sector Dipole}
\subsubsection{Hamiltonian for the Body of a Sector Dipole}
\index{dipole body!Hamiltonian}
\index{Hamiltonian!dipole body}
Inserting the proper vector potential,
the expanded Hamiltonian for a sector dipole has the terms
\begin{equation}\eqarray{
H_1&=&(K_0-h)x+\frac{\eta \delta_s}{\beta_s} p_t,\\
%
H_2&=&\frac{1}{2}(K_1+hK_0)x^2-\frac{1}{2}K_1y^2+\frac{1}{2}\aux, \\
%
H_3&=&\frac{1}{6}(K_2+2hK_1)x^3-\frac{1}{2}(K_2+hK_1)xy^2+
      \frac{1}{2}\left(hx-\frac{p_t}{\beta_s}\right)\aux, \\
%
H_4&=&\frac{1}{24}(K_3+3hK_2)x^4-\frac{1}{4}(K_3+2hK_2)x^2y^2+
      \frac{1}{24}(K_3+hK_2-h^2K_1)y^4-\\
   & &\frac{p_t}{2\beta_s}\left(hx-\frac{p_t}{\beta_s}\right)\aux +
      \frac{1}{8}\aux^2.
}\end{equation}
We note the following differences with respect to reference~\cite{ISE85}:
\begin{enumerate}
\item
  There is a term~$H_1$, generated by $h \neq K_0$, and by~$p_s \neq p_0$.
\item
  The relativistic parameters $\beta_s = v/c$ and $\gamma_s = E/m$ are
  evaluated for the velocity and energy corresponding to $p_s$.
\item
  For the term in $x^2y^2$ we have corrected a mis-print in the numeric
  factor.
\end{enumerate}
 
\subsubsection{Lie-Algebraic Map for a Sector Dipole}
\index{dipole body!Lie transformation}
\index{Lie transformation!dipole body}
\index{map!dipole body}
The Lie algebraic map of order three for a sector dipole has been
derived in~\cite{ISE85}.
At present MAD knows only terms up to $f_3$ for dipoles.
For notations refer to Section~\ref{integrals}.
The transfer matrix for the body of the dipole is
\begin{equation}
R=\left(\myarray{
c_x       &s_x   &0         &0     &0     &\frac{h}{\beta_s}d_x   \\
-k_x^2s_x &c_x   &0         &0     &0     &\frac{h}{\beta_s}s_x   \\
0         &0     &c_y       &s_y   &0     &0 \\
0         &0     &-k_y^2s_y &c_y   &0     &0 \\
-\frac{h}{\beta_s}s_x &-\frac{h}{\beta_s}d_x &0 &0 &1
   &\frac{L}{\beta_s^2\gamma_s^2}-\frac{h^2}{\beta_s^2}J_1 \\
0         &0     &0         &0     &0     &1
}\right)\end{equation}
The generator~$f_1$ for the constant term and~$f_3$ for the
non-linear terms are
\begin{equation}
f_1=-\frac{\eta\delta_s}{\beta_s} p_t, \qquad
f_3=\frac{1}{6}\sum_{i=1}^6 \sum_{j=1}^6 \sum_{k=1}^6 F_{ijk} Z_i Z_j Z_k, \\
\end{equation}
the latter has the coefficients
\begin{equation}\eqarray{\openup 1\jot
F_{111}&=& -\frac{1}{3} (K_2 + 2 h K_1) s_x (2 + c_x^2) - h k_x^4 s_x^3, \\
F_{112}&=& +\frac{1}{3} (K_2 + 2 h K_1) (d_x + s_x^2 c_x) -
  h k_x^2 s_x^2 c_x, \\
F_{116}&=& -\frac{h}{6\beta_s} (K_2 + 2 h K_1) (3 J_1 - 3 s_x d_x + 2 s_x^3)
  + \frac{h^2}{\beta_s} k_x^2 s_x^3 + \frac{1}{2\beta_s} K_1 (L - s_x c_x), \\
F_{122}&=& -\frac{1}{3} (K_2 + 2 h K_1) s_x^3 - h s_x c_x^2, \\
F_{126}&=& +\frac{h}{6\beta_s} (K_2 + 2 h K_1) d_x^2 (1 + 2 c_x)
  + \frac{h^2}{\beta_s}s_x^2 c_x + \frac{1}{2\beta_s} K_1 s_x^2, \\
F_{133}&=&+ 2 K_1 K_2 (k_x^2 s_x J_d + c_x J_s) + (K_2 + h K_1) s_x, \\
F_{134}&=&- K_2 (k_x^2 s_x J_s + c_x J_c), \\
F_{144}&=&+ 2 K_2 (k_x^2 s_x J_d + c_x J_s) - h s_x, \\
F_{166}&=& -\frac{h^2}{3\beta_s^2} (K_2 + 2 h K_1) (s_x d_x^2 - 2 J_2)
  - \frac{h^3}{\beta_s^2} s_x^3 - \frac{h}{\beta_s^2} K_1 (J_1 + s_x d_x)
  - \frac{h}{\beta_s^2\gamma_s^2} s_x, \\
}\end{equation}
\begin{equation}\eqarray{\openup 1\jot
F_{222}&=& +\frac{1}{3} (K_2 + 2 h K_1) d_x^2 (2 + c_x) +
  h (d_x + s_x^2 c_x), \\
F_{226}&=& -\frac{h}{3\beta_s} (K_2 + 2 h K_1) (s_x d_x^2 + J_2)
  - \frac{h^2}{2\beta_s} \Bigl(J_1 + s_x d_x (1 + 2 c_x)\Bigr)
  + \frac{1}{2\beta_s} (L + s_x c_x), \\
F_{233}&=&+ 2 K_1 K_2 (c_x J_d - s_x J_s) - (K_2 + h K_1) d_x, \\
F_{234}&=&- K_2 (c_x J_s - s_x J_c), \\
F_{244}&=&2 K_2 (c_x J_d - s_x J_s) + h d_x, \\
F_{266}&=& +\frac{h^2}{3\beta_s^2} (K_2 + 2 h K_1) d_x^3
  - \frac{h^3}{\beta_s^2} s_x^2 d_x - \frac{h}{\beta_s^2} s_x^2
  - \frac{h}{\beta_s^2\gamma_s^2} d_x, \\
F_{336}&=&+ \frac{2h}{\beta_s} K_1 K_2 (J_f + d_x J_s - s_x J_d)
  + \frac{h}{\beta_s} (K_2 + h K_1) J_1 - \frac{1}{2\beta_s} K_1 (L - s_y c_y), \\
F_{346}&=&- \frac{h}{\beta_s} K_2 (J_d + d_x J_c - s_x J_s)
  - \frac{1}{2\beta_s} K_1 s_y^2, \\
F_{446}&=&+ \frac{2h}{\beta_s} K_2 (J_f + d_x J_s - s_x J_d)
  - \frac{h^2}{\beta_s} J_1 + \frac{1}{2\beta_s} (L + s_y c_y), \\
F_{666}&=& -\frac{h^3}{\beta_s^3} (K_2 + 2 h K_1) J_3
  - \frac{h^4}{\beta_s^3} (s_x d_x^2 + J_2)
  + \frac{3h^2}{2\beta_s^3} (J_1 + s_x d_x)
  + \frac{3}{\beta_s^3\gamma_s^2} (L - h^2 J_1).
}\end{equation}
 
\subsubsection{TRANSPORT Map for a Sector Dipole}
\index{dipole body!TRANSPORT map}
\index{TRANSPORT map!dipole body}
\index{map!dipole body}
The TRANSPORT map for bending magnets has been derived
in~\cite{ISE85}, based on the work in~\cite{SLAC75}.
For notations, refer to Chapter~\ref{integrals}.
The dipole changes the time reference by
\begin{equation}
\Delta t = \frac{L\eta\delta_x}{\beta_s}.
\end{equation}
The transfer matrix for its body is
\begin{equation}
R=\left(\myarray{
c_x       &s_x   &0         &0     &0     &\frac{h}{\beta_s}d_x   \\
-k_x^2s_x &c_x   &0         &0     &0     &\frac{h}{\beta_s}s_x   \\
0         &0     &c_y       &s_y   &0     &0 \\
0         &0     &-k_y^2s_y &c_y   &0     &0 \\
-\frac{h}{\beta_s}s_x &-\frac{h}{\beta_s}d_x &0 &0 &1
   &\frac{L}{\beta_s^2\gamma_s^2}-\frac{h^2}{\beta_s^2}J_1 \\
0         &0     &0         &0     &0     &1
}\right).
\end{equation}
and the non-zero terms of second order:
\begin{equation}\eqarray{
T_{111}&=&-\frac{1}{6}(K_2+2hK_1)(s_x^2+d_x) - \frac{h}{2}k_x^2 s_x^2,\\
T_{112}&=&-\frac{1}{6}(K_2+2hK_1)s_x d_x + \frac{h}{2}s_x c_x,\\
T_{122}&=&-\frac{1}{6}(K_2+2hK_1)d_x^2 + \frac{h}{2}c_x d_x, \\
T_{116}&=&-\frac{h}{12\beta_s}(K_2+2hK_1)(3s_x J_1-d_x^2)
 + \frac{h^2}{2\beta_s} s_x^2  + \frac{1}{4\beta_s}K_1 Ls_x, \\
T_{126}&=&-\frac{h}{12\beta_s}(K_2+2hK_1)(s_x d_x^2-2c_x J_2)
 + \frac{h^2}{4\beta_s} (s_x d_x+c_x J_1)-\frac{1}{4\beta_s}(s_x+Lc_x), \\
T_{166}&=&-\frac{h^2}{6\beta_s^2}(K_2+2hK_1)(d_x^3-2s_x J_2)
 + \frac{h^3}{2\beta_s^2}s_x J_1 - \frac{h}{2\beta_s^2}Ls_x
 - \frac{h}{2\beta_s^2\gamma_s^2}d_x, \\
T_{133}&=&K_1 K_2 J_d + \frac{1}{2}(K_2+hK_1)d_x, \\
T_{134}&=&\frac{1}{2}K_2 J_s, \\
T_{144}&=&K_2 J_d - \frac{h}{2}d_x, \\
}\end{equation}
\begin{equation}\eqarray{
T_{211}&=&-\frac{1}{6}(K_2+2hK_1)s_x(1+2c_x), \\
T_{212}&=&-\frac{1}{6}(K_2+2hK_1)d_x(1+2c_x), \\
T_{222}&=&-\frac{1}{3}(K_2+2hK_1)s_x d_x - \frac{h}{2}s_x, \\
T_{216}&=&-\frac{h}{12\beta_s}(K_2+2hK_1)(3c_x J_1+s_x d_x)
 - \frac{1}{4\beta_s}K_1(s_x-Lc_x), \\
T_{226}&=&-\frac{h}{12\beta_s}(K_2+2hK_1)(3s_x J_1+d_x^2)
 + \frac{1}{4\beta_s}K_1 Ls_x, \\
T_{266}&=&-\frac{h^2}{6\beta_s^2}(K_2+2hK_1)(s_x d_x^2-2c_x J_2)
 - \frac{h}{2\beta_s^2}K_1(c_x J_1-s_x d_x) - \frac{h}{2\beta_s^2\gamma_s^2}s_x, \\
T_{233}&=&K_1 K_2 J_s + \frac{1}{2}(K_2+hK_1)s_x, \\
T_{234}&=&\frac{1}{2}K_2 J_c, \\
T_{244}&=&K_2 J_s - \frac{h}{2}s_x, \\
}\end{equation}
\begin{equation}\eqarray{
T_{313}&=&\frac{1}{2}K_2(c_y J_c-2K_1 s_y J_s) + \frac{h}{2}K_1 s_x s_y, \\
T_{314}&=&\frac{1}{2}K_2(s_y J_c-2c_y J_s) + \frac{h}{2}s_x c_y, \\
T_{323}&=&\frac{1}{2}K_2(c_y J_s-2K_1 s_y J_d) + \frac{h}{2}K_1 d_x s_y, \\
T_{324}&=&\frac{1}{2}K_2(s_y J_s-2c_y J_d) + \frac{h}{2}d_x c_y, \\
T_{336}&=&\frac{h}{2\beta_s}K_2(c_y J_d-2K_1 s_y J_f)
 + \frac{h^2}{2\beta_s}K_1 J_1 s_y - \frac{1}{4\beta_s}K_1 Ls_y, \\
T_{346}&=&\frac{h}{2\beta_s}K_2(s_y J_d-2c_y J_f)
 + \frac{h^2}{2\beta_s}J_1 c_y - \frac{1}{4\beta_s}(s_y+Lc_y), \\
}\end{equation}
\begin{equation}\eqarray{
T_{413}&=&\frac{1}{2}K_1 K_2(2c_y J_s-s_y J_c) + \frac{1}{2}(K_2+hK_1)s_x c_y, \\
T_{414}&=&\frac{1}{2}K_2(2K_1 s_y J_s-c_y J_c) + \frac{1}{2}(K_2+hK_1)s_x s_y, \\
T_{423}&=&\frac{1}{2}K_1 K_2(2c_y J_d-s_y J_s) + \frac{1}{2}(K_2+hK_1)d_x c_y, \\
T_{424}&=&\frac{1}{2}K_2(2K_1 s_y J_d-c_y J_s) + \frac{1}{2}(K_2+hK_1)d_x s_y, \\
T_{436}&=&\frac{h}{2\beta_s}K_1 K_2(2c_y J_f-s_yJ_d) +
  \frac{h}{2\beta_s}(K_2+hK_1)J_1 c_y + \frac{1}{4\beta_s}K_1(s_y-Lc_y), \\
T_{446}&=&\frac{h}{2\beta_s}K_2(2K_1 s_y J_f-c_yJ_d) +
  \frac{h}{2\beta_s}(K_2+hK_1)J_1 s_y - \frac{1}{4\beta_s}K_1 Ls_y, \\
}\end{equation}
\begin{equation}\eqarray{
T_{511}&=&\frac{h}{12\beta_s}(K_2+2hK_1)(s_x d_x+3J_1)
 - \frac{1}{4\beta_s}K_1(L-s_x c_x), \\
T_{512}&=&\frac{h}{12\beta_s}(K_2+2hK_1)d_x^2 + \frac{1}{4\beta_s}K_1 s_x^2, \\
T_{522}&=&\frac{h}{6\beta_s}(K_2+2hK_1)J_2 - \frac{1}{2\beta_s}s_x
 - \frac{1}{4\beta_s}K_1(J_1-s_x d_x), \\
T_{516}&=&\frac{h^2}{12\beta_s^2}(K_2+2hK_1)(3d_x J_1-4J_2)
 + \frac{h}{4\beta_s^2}K_1 J_1 (1+c_x) + \frac{h}{2\beta_s^2\gamma_s^2}s_x, \\
T_{526}&=&\frac{h^2}{12\beta_s^2}(K_2+2hK_1)(d_x^3-2s_x J_2)
 + \frac{h}{4\beta_s^2}K_1 s_x J_1 + \frac{h}{2\beta_s^2\gamma_s^2}d_x, \\
T_{566}&=&\frac{h^3}{6\beta_s^3}(K_2+2hK_1)(3J_3-2d_x J_2)
 + \frac{h^2}{6\beta_s^3}K_1\left(s_x d_x^2 - J_2(1+2c_x)\right)
 + \frac{3}{2\beta_s^3\gamma_s^2}(h^2J_1-L), \\
T_{533}&=&-\frac{h}{\beta_s}K_1 K_2 J_f - \frac{h}{2\beta_s}(K_2+hK_1)J_1
 + \frac{1}{4\beta_s}K_1(L-c_y s_y), \\
T_{534}&=&-\frac{h}{2\beta_s}K_2 J_d - \frac{1}{4\beta_s}K_1 s_y^2, \\
T_{544}&=&-\frac{h}{\beta_s}K_2 J_f + \frac{h^2}{2\beta_s}J_1
 - \frac{1}{4\beta_s}(L+c_y s_y).
}\end{equation}
 
\subsection{Body of a Rectangular Dipole}
\subsubsection{Hamiltonian for the Body of a Rectangular Dipole}
\index{Hamiltonian!rectangular dipole}
\index{rectangular dipole!Hamiltonian}
For a rectangular Dipole with a straight reference one may set the
curvature~$h=0$,
and the expanded Hamiltonian for a sector dipole has the terms
\begin{equation}\eqarray{
H_1&=&K_0x+\frac{\eta \delta_s}{\beta_s} p_t,\\
%
H_2&=&\frac{1}{2}K_1(x^2-y^2)+\frac{1}{2}\aux, \\
%
H_3&=&\frac{1}{6}(K_2x^3-3xy^2)-\frac{p_t}{2\beta_s}\aux, \\
%
H_4&=&\frac{1}{24}(K_3x^4-6x^2y^2+y^4)+\\
   & &\frac{p_t^2}{2\beta_s^2}\aux+\frac{1}{8}\aux^2.
}\end{equation}
At this time the maps for this case have not been worked out yet.
 
 
\section{Quadrupole}
\label{quadrupole}
The transfer map for a quadrupole is the limit obtained by setting
$h=K_2=K_3=0$ in the map for a combined function dipole.
We use the two quantities
\begin{equation}
k_x^2 = K_1 = (q B_1) / (p_s c), \qquad k_y^2 = -K_1.
\end{equation}
For other definitions refer to Section~\ref{integrals}.
 
\subsection{Hamiltonian for a Quadrupole}
\index{Hamiltonian!quadrupole}
\index{quadrupole!Hamiltonian}
The expanded Hamiltonian for a quadrupole is
\begin{equation}\eqarray{
H_1 &=&\frac{\eta \delta_s}{\beta_s} p_t,\\
%
H_2 &=& \frac{1}{2} K_1 (x^2 - y^2) + \frac{1}{2} \aux, \\
%
H_3 &=&-\frac{p_t}{2\beta_s} \aux, \\
%
H_4 &=& \frac{p_t^2}{2\beta_s^2} \aux + \frac{1}{8} \aux^2.
}\end{equation}
 
\subsection{Lie-Algebraic map for a Quadrupole}
\index{Lie transformation!quadrupole}
\index{quadrupole!Lie transformation}
\index{map!quadrupole}
The transfer matrix for a quadrupole is
\begin{equation}
F=\left(\myarray{
c_x       &s_x   &0         &0     &0     &0 \\
-k_x^2s_x &c_x   &0         &0     &0     &0 \\
0         &0     &c_y       &s_y   &0     &0 \\
0         &0     &-k_y^2s_y &c_y   &0     &0 \\
0         &0     &0         &0     &1     &\frac{L}{\beta_s^2\gamma_s^2} \\
0         &0     &0         &0     &0     &1
}\right).
\end{equation}
The generators for the quadrupole are~\cite{DOU82}:
\begin{equation}\eqarray{
f_1 &=& - \frac{L\eta\delta_s}{\beta_s} p_t, \\
f_3 &=& \frac{p_t}{4\beta_s} \bigl(+K_1(L - s_x c_x) x^2 + 2K_1s_x^2xp_x
        + (L+s_xc_x)p_x^2 \\
    & &-K_1(L-s_yc_y)y^2 - 2K_1s_y^2yp_y + (L+s_yc_y)p_y^2\bigr)
        + \frac{p_t^3}{2\beta_s^3\gamma_s^2}, \\
f_4 &=& \frac{1}{4!} \sum_{i=1}^6 \sum_{j=1}^6 \sum_{k=1}^6 \sum_{l=1}^6
F_{ijkl} Z_i Z_j Z_k Z_l
}\end{equation}
and~$f_4$ has the coefficients
\begin{equation}\eqarray{
F_{1111}&=&+\frac{K_1^2}{64}\Bigl(-s(4k_x,L)+4s(2k_x,L)-3L\Bigr), \\
F_{1112}&=&-\frac{K_1^3}{8}s^4(k_x,L), \\
F_{1122}&=&+\frac{3K_1}{32}\Bigl(s(4k_x,L)-L\Bigr), \\
F_{1222}&=&+\frac{1}{8}\Bigl(c^4(k_x,L)-1\Bigr), \\
F_{2222}&=&-\frac{1}{64}\Bigl(s(4k_x,L)+4s(2k_x,L)+3L\Bigr), \\
}\end{equation}
 
\begin{equation}\eqarray{
F_{3333}&=&+\frac{K_1^2}{64}\Bigl(-s(4k_y,L)+4s(2k_y,L)-3L\Bigr), \\
F_{3334}&=&+\frac{K_1^3}{8}s^4(k_y,L), \\
F_{3344}&=&-\frac{3K_1}{32}\Bigl(s(4k_y,L)-L\Bigr), \\
F_{3444}&=&+\frac{1}{8}\Bigl(c^4(k_y,L)-1\Bigr), \\
F_{4444}&=&-\frac{1}{64}\Bigl(s(4k_y,L)+4s(2k_y,L)+3L\Bigr), \\
}\end{equation}
 
\begin{equation}\eqarray{
F_{1133}&=&+\frac{K_1^2}{32}\biggl(-s(2k_y,L)\Bigl(2-c(2k_x,L)\Bigr)
 -s(2k_x,L)\Bigl(2-c(2k_y,L)\Bigr)+2L\biggr), \\
F_{1134}&=&+\frac{K_1}{32}\biggl(c(2k_y,L)\Bigl(2-c(2k_x,L)\Bigr)
 -4K_1s(2k_x,L)s(2k_y,L)-1\biggr), \\
F_{1144}&=&+\frac{K_1}{32}\biggl(s(2k_x,L)\Bigl(2+c(2k_y,L)\Bigr)
 -s(2k_y,L)\Bigl(2-c(2k_x,L)\Bigr)-2L\biggr), \\
F_{1233}&=&-\frac{K_1}{32}\biggl(c(2k_x,L)\Bigl(2-c(2k_y,L)\Bigr)
 +4K_1s(2k_y,L)s(2k_x,L)-1\biggr), \\
F_{1234}&=&+\frac{K_1}{8}\Bigl(s(2k_x,L)c(2k_y,L)-c(2k_x,L)s(2k_y,L)\Bigr),\\
F_{1244}&=&+\frac{1}{32}\biggl(c(2k_x,L)\Bigl(2+c(2k_y,L)\Bigr)
 -4K_1s(2k_x,L)s(2k_y,L)-3\biggr), \\
F_{2233}&=&-\frac{K_1}{32}\biggl(s(2k_y,L)\Bigl(2+c(2k_x,L)\Bigr)
 -s(2k_x,L)\Bigl(2-c(2k_y,L)\Bigr)-2L\biggr), \\
F_{2234}&=&+\frac{1}{32}\biggl(c(2k_y,L)\Bigl(2+c(2k_x,L)\Bigr)
 +4K_1s(2k_x,L)s(2k_y,L)-3\biggr), \\
F_{2244}&=&-\frac{1}{32}\biggl(s(2k_x,L)\Bigl(2+c(2k_y,L)\Bigr)
 +s(2k_y,L)\Bigl(2+c(2k_x,L)\Bigr)+2L\biggr), \\
}\end{equation}
 
\begin{equation}\eqarray{
F_{1166}&=&+\frac{K_1}{8}\Bigl(L-s(2k_x,L)\Bigr)
 +\frac{K_1}{16\beta_s^2}\Bigl(3s(2k_x,L)+L(c(2k_x,L)-4)\Bigr), \\
F_{1266}&=&-\frac{K_1}{4\beta_s^2}\Bigl(Ls(2k_x,L)+(2-\beta_s^2)s^2(k_x,L)\Bigr),\\
F_{2266}&=&+\frac{1}{8}\Bigl(L+s(2k_x,L)\Bigr)
 -\frac{1}{16\beta_s^2}\biggl(5s(2k_x,L)+L\Bigl(6+c(2k_x,L)\Bigr)\biggr), \\
}\end{equation}
 
\begin{equation}\eqarray{
F_{3366}&=&-\frac{K_1}{8}\Bigl(L-s(2k_y,L)\Bigr)
 -\frac{K_1}{16\beta_s^2}\Bigl(3s(2k_y,L)+L(c(2k_y,L)-4)\Bigr), \\
F_{3466}&=&+\frac{K_1}{4\beta_s^2}\Bigl(Ls(2k_y,L)+(2-\beta_s^2)s^2(k_y,L)\Bigr),\\
F_{4466}&=&+\frac{1}{8}\Bigl(L+s(2k_y,L)\Bigr)
 -\frac{1}{16\beta_s^2}\biggl(5s(2k_y,L)+L\Bigl(6+c(2k_y,L)\Bigr)\biggr), \\
}\end{equation}
 
\begin{equation}\eqarray{
F_{6666}&=&+\frac{1}{8\beta_s^2\gamma_s^2}\Bigl(1-\frac{5}{\beta_s^2}\Bigr).
}\end{equation}
 
\subsection{TRANSPORT Map for a Quadrupole}
\index{TRANSPORT map!quadrupole}
\index{quadrupole!TRANSPORT map}
\index{map!quadrupole}
The complete TRANSPORT map for a quadrupole is:
\begin{equation}\eqarray{
x_2 &=&c_x x_1 + s_x p_{x1}
  +\frac{1}{2\beta_s}(+K_1 Ls_x x_1 p_{t1}-(s_x+Lc_x)p_{x1}p_{t1}), \\
p_{x2} &=& -K_1 s_x x_1 + c_x p_{x1}
  +\frac{K_1}{2\beta_s}(-(s_x-Lc_x)x_1 p_{t1}+Ls_x p_{x1}p_{t1}),\\
y_2 &=& c_y y_1 + s_y p_{y1}
  +\frac{1}{2\beta_s}(-K_1 Ls_y y_1 p_{t1}-(s_y+Lc_y)p_{y1}p_{t1}), \\
p_{y2} &=& +K_1 s_y y_1 + c_y p_{y1}
  +\frac{K_1}{2\beta_s}(+(s_y-Lc_y)y_1 p_{y1}-Ls_y p_{y1}p_{t1}),\\
t_2 &=& t_1 + \frac{L\eta\delta_s}{\beta_s}
  -\frac{L}{\beta_s^2\gamma_s^2} p_{t1}\\
  & &-\frac{1}{4\beta_s}K_1(L - s_x c_x) x_1^2
  +\frac{1}{2\beta_s}K_1 s_x^2 x_1 p_{x1}
  -\frac{1}{4\beta_s}(L + s_x c_x)p_{x1}^2 \\
  & &+\frac{1}{4\beta_s}K_1(L - s_y c_y) y_1^2
  -\frac{1}{2\beta_s}K_1 s_y^2 y_1 p_{y1}
  -\frac{1}{4\beta_s}(L + s_y c_y) p_{y1}^2\\
  & &-\frac{3L}{2\beta_s^3\gamma_s^2} p_{t1}^2,\\
p_{t2} &=& p_{t1}.
}\end{equation}
 
 
\section{Sextupole}
\label{sextupole}
The transfer map for a sextupole is the limits obtained by setting
$h=K_1=K_3=0$ in the map for a combined function dipole.
 
\subsection{Hamiltonian for a Sextupole}
\index{Hamiltonian!sextupole}
\index{sextupole!Hamiltonian}
The expanded Hamiltonian for a sextupole is
\begin{equation}\eqarray{
H_1 &=&\frac{\eta \delta_s}{\beta_s} p_t,\\
%
H_2 &=& \frac{1}{6} K_1 (x^3 - 3xy^2) + \frac{1}{2} \aux, \\
%
H_3 &=&-\frac{p_t}{2\beta_s} \aux, \\
%
H_4 &=& \frac{p_t^2}{2\beta_s^2} \aux + \frac{1}{8} \aux^2.
}\end{equation}
 
\subsection{Lie-Algebraic Map for a Sextupole}
\index{Lie transformation!sextupole}
\index{sextupole!Lie transformation}
\index{map!sextupole}
A sextupole has the same transfer matrix as a drift space
\begin{equation}
F=\left(\myarray{
1 & L & 0 & 0 & 0 & 0 \\
0 & 1 & 0 & 0 & 0 & 0 \\
0 & 0 & 1 & L & 0 & 0 \\
0 & 0 & 0 & 1 & 0 & 0 \\
0 & 0 & 0 & 0 & 1 & \frac{L}{\beta_s^2 \gamma_s^2} \\
0 & 0 & 0 & 0 & 0 & 1 \\
}\right ).
\end{equation}
$L$ is the sextupole length.
The sextupole strength $K_2$, defined as for a combined function dipole,
produces the generators~\cite{DOU82}:
\begin{equation}\eqarray{
f_1&=& - \frac{L\eta\delta_s}{\beta_s} p_t, \\
f_3&=&\frac{L}{2\beta_s}\aux p_t-\frac{L}{6}K_2(x^3-3xy^2)+
      \frac{L^2}{4}K_2\left((x^2-y^2)p_x-2xyp_y\right)-\\
   & &\frac{L^3}{6}K_2\left(x(p_x^2-p_y^2)-2yp_xp_y\right)+
      \frac{L^4}{24}K_2(p_x^3-3p_xp_y^2),\\
f_4&=&\frac{L^3}{48}K_2^2(x^2+y^2)^2-
      \frac{L^4}{24}K_2^2(x^2+y^2)(xp_x+yp_y)+\\
   & &\frac{L^5}{480}K_2^2\left((x^2+y^2)(p_x^2+p_y^2)+
     14(xp_x+yp_y)^2\right)-\\
   & &\frac{L^6}{96}K_2^2(xp_x+yp_y)(p_x^2+p_y^2)+
      \frac{L^7}{672}K_2^2(p_x^2+p_y^2)^2-\\
   & &\frac{L^3}{12}K_2\left(x(p_x^2-p_y^2)-2p_xyp_y\right)p_t+
      \frac{L^4}{24}K_2(p_x^3-3p_xp_y^2)p_t-\\
   & &\frac{L}{2\beta_s^2}\aux p_t^2-\frac{L}{8}\aux^2.
}\end{equation}
 
\subsection{TRANSPORT Map for a Sextupole}
\index{TRANSPORT map!sextupole}
\index{sextupole!TRANSPORT map}
\index{map!sextupole}
The complete TRANSPORT map for a thick sextupole is
\begin{equation}\eqarray{
x_2 &=& x_1 + L \left(1-\frac{p_{t1}}{\beta_s}\right) p_{x1}
-K_2\left(\frac{L^2}{4}(x_1^2 - y_1^2)
         +\frac{L^3}{12}(x_1 p_{x1} - y_1 p_{y1})
         +\frac{L^4}{24}(p_{x1}^2 - p_{y1}^2)\right)
-\frac{L}{2\beta_s}p_{x1}p_{t1}, \\
p_{x2} &=& p_{x1}
-K_2\left(\frac{L}{2}(x_1^2 - y_1^2)
         +\frac{L^2}{4}(x_1 p_{x1} - y_1 p_{y1})
         +\frac{L^3}{6}(p_{x1}^2 - p_{y1}^2)\right), \\
y_2 &=& y_1 + L \left(1-\frac{p_{t1}}{\beta_s}\right) p_{y1}
+K_2\left(\frac{L^2}{4}x_1 y_1
         +\frac{L^3}{12}(x_1 p_{y1} + y_1 p_{x1})
         +\frac{L^4}{24}p_{x1}p_{y1}\right)
-\frac{L}{2\beta_s}p_{y1}p_{t1}, \\
p_{y2} &=& p_{y1}
+K_2\left(\frac{L}{2}x_1 y_1
         +\frac{L^2}{4}(x_1 p_{y1} + y_1 p_{x1})
         +\frac{L^3}{6}p_{x1}p_{y1}\right), \\
t_2 &=& \frac{L\eta\delta_s}{\beta_s} + t_1
+\frac{L}{\beta_s^2 \gamma_s^2}p_{t1}
-\frac{L}{2\beta_s}\left(p_{x1}^2+p_{y1}^2
          +\frac{3p_{t1}^2}{\beta_s^2\gamma_s^2}\right), \\
p_{t2} &=& p_{t1}.
}\end{equation}
 
 
\section{Octupole}
\label{octupole}
The transfer map for an octupole is limit obtained by setting
$h=K_1=K_2=0$ in the map for a combined function dipole.
 
\subsection{Lie-Algebraic Map for an Octupole}
\index{Lie transformation!octupole}
\index{octupole!Lie transformation}
\index{map!octupole}
In the Lie algebraic formalism an octupole can be handled as
a lens of finite length.
It has the same transfer matrix as a drift space
\begin{equation}
F=\left(\myarray{
1 & L & 0 & 0 & 0 & 0 \\
0 & 1 & 0 & 0 & 0 & 0 \\
0 & 0 & 1 & L & 0 & 0 \\
0 & 0 & 0 & 1 & 0 & 0 \\
0 & 0 & 0 & 0 & 1 & \frac{L}{\beta_s^2 \gamma_s^2} \\
0 & 0 & 0 & 0 & 0 & 1 \\
}\right ).
\end{equation}
$L$ is the octupole length.
The generators are~\cite{DOU82}:
\begin{equation}\eqarray{
f_3&=&+\frac{L}{2\beta_s}\aux p_t,\\
f_4&=&-\frac{L}{2\beta_s^2}\aux p_t^2-\frac{L}{8}\aux^2
      -\frac{K_3}{24}(x^4-6x^2y^2+y^4).
}\end{equation}
 
\subsection{TRANSPORT Map for a Thin Octupole}
\index{TRANSPORT map!octupole}
\index{octupole!TRANSPORT map}
\index{map!octupole}
In TRANSPORT form an octupole is treated as a thin lens placed between
two drifts of half the octupole length.
The map for the thin lens with strength $K_3 L = (q B_3) / (p_s c)$,
evaluated with respect to the actual orbit produces a kick
\begin{equation}
p_{x2}=p_{x1}-\frac{1}{6}K_3L(x^3-3xy^2), \qquad
p_{y2}=p_{y1}+\frac{1}{6}K_3L(3x^2y-y^3).
\end{equation}
Hence the transfer matrix for the thin lens with respect to a given
orbit is the unit matrix augmented with the terms
\begin{equation}
-R_{21}=+R_{43}=\frac{1}{2}K_3L(x^2-y^2), \qquad
+R_{23}=+R_{41}=K_3Lxy, \\
\end{equation}
and the second-order terms around that orbit are
\begin{equation}
-T_{211}=+T_{233}=+T_{413}=+T_{431}=\frac{1}{2}K_3Lx, \qquad
+T_{213}=+T_{231}=+T_{411}=-T_{433}=\frac{1}{2}K_3Ly.
\end{equation}
 
 
\section{Thin Multipole}
\label{multipole}
A thin multipole affects the reference system like a dipole,
i.~e. the reference direction changes by the bend angle of its
nominal dipole component.
The total dipole strength thus generates dispersion only;
but a dipole error, if present, also changes the orbit.
 
\subsection{Lie-Algebraic Map for a Thin Multipole}
\index{Lie transformation!thin multipole}
\index{thin multipole!Lie transformation}
\index{map!thin multipole}
A thin multipole affects the reference system like a dipole,
i.~e. the reference direction changes by the bend angle of its
nominal dipole component.
The total dipole strength thus generates dispersion only;
but a dipole error, if present, also changes the orbit with respect to
the reference.
 
With Lie algebraic maps truncated at order of~$f_4$,
field components can be represented up to the octupole.
The multipole deflects the orbit according to the complex kick
\begin{equation}
Z = \Delta K_0 L - K_0 L \frac{\delta}{\beta_s}
  + \sum_{n=1}^3 K_n L \frac{(x + i y)^n}{n!}.
\end{equation}
The geometric terms of the transfer matrix are found by differentiation:
\begin{equation}
-F_{21}=+F_{43}=\Re Z', \qquad +F_{23}=+F_{41}=\Im Z'.
\end{equation}
The {\em total} dipole strength creates the dispersive terms
\begin{equation}
+F_{26}=-F_{51}=\frac{1}{\beta_s}\Re(K_0 L), \qquad
-F_{46}=+F_{53}=\frac{1}{\beta_s}\Im(K_0 L).
\end{equation}
From the equation for~$Z$ one may derive the generators
\begin{equation}
f_1=-\Re\Bigl(\Delta K_0 L (x + iy) \Bigr), \qquad
f_3=-\frac{1}{3!}\Re\Bigl(K_2 L (x - iy)^3 \Bigr), \qquad
f_4=-\frac{1}{4!}\Re\Bigl(K_3 L (x - iy)^4 \Bigr).
\end{equation}
 
\subsection{TRANSPORT Map for a Thin Multipole}
\index{TRANSPORT map!thin multipole}
\index{thin multipole!TRANSPORT map}
\index{map!thin multipole}
The multipole deflects the orbit according to the complex kick
\begin{equation}
P = \Delta K_0 L - K_0 L \frac{p_t}{\beta_s}
  + \sum_{n=1}^N K_n L \frac{(x + i y)^n}{n!},
\end{equation}
i.~e. it produces the orbit change
\begin{equation}\eqarray{
x_2&=&x_1, \qquad p_{x2}&=&p_{x1} - \Re P \\
y_2&=&y_1, \qquad p_{y2}&=&p_{y1} + \Im P \\
t_2&=&t_1 - \frac{1}{\beta_s}\Re\bigl(K_0 L (x + i y)\bigr), \qquad
p_{t2}&=&p_{t1}.
}\end{equation}
 
The transfer matrix can be obtained by differentiating the kick.
Defining
\begin{equation}
P' = \sum_{n=1}^N K_n L \frac{(x + iy)^{n-1}}{(n-1)!},
\end{equation}
the geometric terms of the transfer matrix are
\begin{equation}
-R_{21}=+R_{43}=\Re P', \qquad +R_{23}=+R_{41}=\Im P'.
\end{equation}
The {\em total} dipole strength creates the dispersive terms
\begin{equation}
+R_{26}=-R_{51}=\frac{1}{\beta_s}\Re(K_0 L), \qquad
-R_{46}=+R_{53}=\frac{1}{\beta_s}\Im(K_0 L).
\end{equation}
The second-order terms are due to components $K_2$ and higher.
They are obtained by differentiating the kick twice.
Defining
\begin{equation}
P'' = \sum_{n=2}^N K_n L \frac{(x + iy)^{n-2}}{(n-2)!},
\end{equation}
these terms become
\begin{equation}
-T_{211}=+T_{233}=+T_{413}=+T_{431}=\frac{1}{2}\Re P'', \qquad
+T_{213}=+T_{231}=+T_{411}=-T_{433}=\frac{1}{2}\Im P''.
\end{equation}
 
 
\section{Solenoid}
\label{solenoid}
 
\subsection{Hamiltonian for a Solenoid}
\index{Hamiltonian!solenoid}
\index{solenoid!Hamiltonian}
The expanded Hamiltonian is
\begin{equation}\eqarray{
H_1 &=& \frac{\eta \delta_s}{\beta_s}, \\
H_2 &=& \frac{1}{2} \solaux, \\
H_3 &=&-\frac{p_t}{2\beta_s} \solaux, \\
H_4 &=& \frac{p_t^2}{\beta_s^2} \solaux + \frac{1}{8} \solaux^2.
}\end{equation}
 
\subsection{Lie-Algebraic Map for a Solenoid}
\index{Lie transformation!solenoid}
\index{solenoid!Lie transformation}
\index{map!solenoid}
Solving the Hamiltonian by the techniques of~\cite{DOU82} gives the
transfer matrix
\begin{equation}
F=\left(\myarray{
  C^2 &  \frac{1}{k} S C &   S C & \frac{1}{k}S^2 & 0 & 0 \\
-kS C &              C^2 & -kS^2 &            S C & 0 & 0 \\
- S C & -\frac{1}{k} S^2 &   C^2 & \frac{1}{k}S C & 0 & 0 \\
 kS^2 &             -S C & -kS C &            C^2 & 0 & 0 \\
0 & 0 & 0 & 0 & 1 & \frac{L}{\beta_s^2 \gamma_s^2} \\
0 & 0 & 0 & 0 & 0 & 1 \\
}\right ),
\end{equation}
where
\begin{equation}
k = \frac{q B_0}{2 p_s}, \qquad C = cos(kL), \qquad S = sin(kl).
\end{equation}
The third- and fourth-order parts of the Hamiltonian are invariant
under the linear transformation.
Hence the non-linear generators take a particularly simple form:
\begin{equation}\eqarray{
f_1&=& - \frac{L\eta\delta_s}{\beta_s} p_t, \\
f_3&=&+\frac{L}{2\beta_s}\solaux p_t, \\
f_4&=&-\frac{L}{2\beta_s^2}\solaux p_t^2 - \frac{L}{8}\solaux^2.
}\end{equation}               
Two effects should be considered at the ends of the solenoid.
First, the field lines cannot end abruptly at the ends;
and second, the vector potential is zero outside the solenoid
and finite inside.
The first effect can be estimated by assuming that the magnetic flux lines
bend sharply and concentrate in a radial plane at each end of the solenoid.
The second effect causes the transverse canonical momentum to jump by
the value of $e\vec{A}/B\rho$.
Both transformations are non-symplectic,
but fortunately they cancel in the approximation used.
 
\subsection{TRANSPORT Map for a Solenoid}
\index{TRANSPORT map!solenoid}
\index{solenoid!TRANSPORT map}
\index{map!solenoid}
A solenoid changes the time reference by
\begin{equation}
\Delta t = \frac{L\eta\delta_s}{\beta_s}.
\end{equation}
Its Lie transformation is easily transformed to the TRANSPORT map with
the transfer matrix
\begin{equation}
R=\left(\myarray{
  C^2 &  \frac{1}{k} S C &   S C & \frac{1}{k}S^2 & 0 & 0 \\
-kS C &              C^2 & -kS^2 &            S C & 0 & 0 \\
- S C & -\frac{1}{k} S^2 &   C^2 & \frac{1}{k}S C & 0 & 0 \\
 kS^2 &             -S C & -kS C &            C^2 & 0 & 0 \\
0 & 0 & 0 & 0 & 1 & \frac{L}{\beta_s^2 \gamma_s^2} \\
0 & 0 & 0 & 0 & 0 & 1 \\
}\right ).
\end{equation}
The second-order terms become:
\begin{equation}\eqarray{
T_{116} = \frac{kL}{2\beta_s} \sin(2kL), \quad &
T_{126} = -\frac{L}{2\beta_s} \cos(2kL), \quad &
T_{136} = -\frac{k L}{2\beta_s} \cos(2kL), \quad &
T_{146} = -\frac{L}{2\beta_s} \sin(2kL), \\
T_{216} = \frac{k^2 L}{2\beta_s} \cos(2kL), \quad &
T_{226} = \frac{kL}{2\beta_s} \sin(2kL), \quad &
T_{236} = \frac{k^2 L}{2\beta_s} \sin(2kL), \quad &
T_{246} = -\frac{k L}{2\beta_s} \cos(2kL), \\
T_{316} = \frac{k L}{2\beta_s} \cos(2kL), \quad &
T_{326} = \frac{L}{2\beta_s} \sin(2kL), \quad &
T_{336} = \frac{kL}{\beta_s} \sin(2kL), \quad &
T_{346} = -\frac{L}{2\beta_s} \cos(2kL), \\
T_{416} = -\frac{k^2 L}{2\beta_s} \sin(2kL), \quad &
T_{426} = \frac{k L}{2\beta_s} \cos(2kL), \quad &
T_{436} = \frac{k^2 L}{2\beta_s} \cos(2kL), \quad &
T_{446} = \frac{kL}{2\beta_s} \sin(2kL), \\
T_{511} = -\frac{k^2 L}{2\beta_s}, \quad &
T_{514} = \frac{k L}{2\beta_s}, \quad &
T_{544} = -\frac{L}{2\beta_s}, \\
T_{533} = -\frac{k^2 L}{2\beta_s}, \quad &
T_{523} = -\frac{k L}{2\beta_s}, \quad &
T_{522} = -\frac{L}{2\beta_s}, \\
T_{566} = -\frac{3L}{2\beta_s^2\gamma_s^2}.
}\end{equation}
 
 
\section{Orbit Correctors}
\label{corrector}
\index{orbit corrector}
\index{corrector}
\index{map!corrector}
An orbit corrector is modelled as a zero-length dipole between two
drifts of half the corrector length.
The reference system is {\em not changed} by the corrector.
The effect of the thin dipole is simply
\begin{equation}
p_{x2} = p_{x1} + K_{0x} / (1 + \delta_s), \qquad
p_{y2} = p_{y1} + K_{0y} / (1 + \delta_s),
\end{equation}
where $K_{0x}$ and $K_{0y}$ are the given kicks in the respective plane.
There are no non-linear terms.
 
 
\section{RF Cavity}
\label{cavity}
\index{RF cavity!model}
An RF~cavity is treated in the impulse approximation.
The length of a cavity is simulated by placing the accelerating gap
between two drifts of half the cavity length.
The voltage of the thin cavity is
\begin{equation}
V = \hat{V} \sin(\phi_s - \omega t/c).
\end{equation}
The phase lag of the cavity is defined as the RF~phase seen by a
particle arriving with a time difference of zero relative to the
time frame.
$\hat{V}$ is the peak RF~voltage,
$f_{RF} = h_{RF} / T_s$ is the RF~frequency,
and $h_{RF}$~is the harmonic number.
The circular frequency of the RF~cavity is
\begin{equation}
\omega = \frac{2 \pi h_{RF}}{T_s}
\end{equation}
 
\subsection{Exact Solution for a Thin RF Cavity}
\index{exact motion!RF cavity}
\index{RF cavity!exact motion}
The general particle sees an RF~phase of
\begin{equation}
\phi = \phi_s - \omega t / c.
\end{equation}
Hence the cavity causes an accelerating kick of
\begin{equation}
p_{t2} = p_{t1} + \frac{q \hat{V}}{p_s c}\sin\phi.
\end{equation}
Higher-order terms are obtained easily by differentiation of this
expression.
 
\subsection{Lie-Algebraic Map for a Thin RF Cavity}
\index{Lie transformation!RF cavity}
\index{RF cavity!Lie transformation}
\index{map!RF cavity}
The cavity produces an accelerating kick of
\begin{equation}
\delta_2 = \delta_1 + \frac{\hat{V}}{p c}\sin(\phi - \omega t).
\end{equation}
With respect to the actual orbit it has the transfer matrix
\begin{equation}
F=\left(\myarray{
1 & 0 & 0 & 0 & 0 & 0 \\
0 & 1 & 0 & 0 & 0 & 0 \\
0 & 0 & 1 & 0 & 0 & 0 \\
0 & 0 & 0 & 1 & 0 & 0 \\
0 & 0 & 0 & 0 & 1 & 0 \\
0 & 0 & 0 & 0 & -\omega\frac{e \hat{V}}{p c}\cos\phi & 1 \\
}\right ).
\end{equation}
and the generators~\cite{DOU82}:
\begin{equation}
f_1 = +\frac{e \hat{V}}{p c} t \sin\phi, \qquad
f_3 = -\frac{\omega^2}{3!} \frac{e \hat{V}}{p c} t^3 \sin\phi, \qquad
f_4 = +\frac{\omega^3}{4!}\frac{e \hat{V}}{p c} t^4 \cos\phi.
\end{equation}
 
\subsection{TRANSPORT Map for a Thin RF Cavity}
\index{TRANSPORT map!RF cavity}
\index{RF cavity!TRANSPORT map}
\index{map!RF cavity}
Using the definitions in Section~\ref{cavity},
we find the accelerating kick:
\begin{equation}
p_{t2} = p_{t1} + \frac{q \hat{V}}{p_s c}\sin\phi.
\end{equation}
By differentiation one finds that the transfer matrix is
\begin{equation}
R=\left(\myarray{
1 & 0 & 0 & 0 & 0 & 0 \\
0 & 1 & 0 & 0 & 0 & 0 \\
0 & 0 & 1 & 0 & 0 & 0 \\
0 & 0 & 0 & 1 & 0 & 0 \\
0 & 0 & 0 & 0 & 1 & 0 \\
0 & 0 & 0 & 0 & -\frac{\omega}{c} \frac{\hat{V}}{p c}\cos\phi & 1 \\
}\right ).
\end{equation}
and one non-zero second-order term
\begin{equation}
T_{655}=-\frac{\omega^2}{c^2} \frac{\hat{V}}{2 p c}\sin\phi.
\end{equation}
 
 
\subsection{Electrostatic Separator}
\label{separator}
By convention an electrostatic separator does {\em not} change the
reference orbit.
We use the definition
\begin{equation}
k = \frac{q E_y}{p_0 c}
\end{equation}
 
\subsection{Hamiltonian for an Electrostatic Separator}
\index{Hamiltonian!separator}
\index{separator!Hamiltonian}
The exact Hamiltonian for a separator is
\begin{equation}
H=-\sqrt{\left(\frac{1}{\beta_s}+p_t+ky\right)^2-
  \frac{1}{\beta_s^2\gamma_s^2}-\left(p_x^2 + p_y^2\right)}+
\frac{1+\eta\delta_s}{\beta_s}\left(p_t+\frac{1}{\beta_s}\right).
\end{equation}
 
\subsection{Exact Solution for an Electrostatic Separator}
\index{separator!exact motion}
\index{exact motion!separator}
The Hamiltonian is an integral of motion, and is constant on any orbit.
It is easily verified that the exact solution for the equations of
motion is 
\begin{equation}\eqarray{
x_2    &= x_1 + \frac{s}{|H|} p_{x1}, \\
p_{x2} &= p_{x1}, \\
y_2    &= \cosh\left(\frac{ks}{|H|}\right)y_1
         +\frac{1}{k}\sinh\left(\frac{ks}{|H|}\right)p_{y1}
         +\frac{1}{k}\left(\cosh\left(\frac{ks}{|H|}\right)-1\right)E_1,\\
p_{y2} &= k\sinh\left(\frac{ks}{|H|}\right)y_1
         +\cosh\left(\frac{ks}{|H|}\right)p_{y1}
         +\sinh\left(\frac{ks}{|H|}\right)E_1,\\
t_2    &=\frac{L\eta\delta_s}{\beta_s}-
         \sinh\left(\frac{ks}{|H|}\right)y_1
         -\frac{1}{k}\left(\cosh\left(\frac{ks}{|H|}\right)-1\right)p_{y1}
         +t_1-\frac{1}{k}\sinh\left(\frac{ks}{|H|}\right)E_1,\\
p_{t2} &=p_{t1}.\\
}\end{equation}
 
\subsection{Lie-Algebraic Map for an Electrostatic Separator}
\index{Lie transformation!separator}
\index{separator!Lie transformation}
\index{map!separator}
The Lie algebraic map for a separator can be derived from the
TRANSPORT map listed below.
We use the abbreviations
\begin{equation}
C = \cosh(kL), \qquad S = \sinh(kL).
\end{equation}
The transfer matrix is
\begin{equation}
F=\left(\myarray{
1 &L &0 &0 &0 &0 \\
0 &1 &0 &0 &0 &0 \\
0 &0 
&C-\frac{kL}{\beta_s^2}S &\frac{1}{k}S 
&0 &\frac{1}{k}\left(C-1\right)-\frac{L}{\beta_s^2}S \\
0 &0 
&k\left(S-\frac{kL}{\beta_s^2}C\right) &C 
&0 &S-\frac{kL}{\beta_s^2}C \\
0 &0 
&-\left(S-\frac{kL}{\beta_s^2}C\right) &-\frac{1}{k}(C-1)
&1 &-\frac{1}{k}S+\frac{L}{\beta_s^2}C \\ 
0 &0 &0 &0 &0 &1    
}\right).
\end{equation}
The generators are
\begin{equation}\eqarray{
f_1 &=&-\frac{L\eta\delta_s}{\beta_s} p_t+
    \frac{1}{\beta_s}\left(S-\frac{kL}{\beta_s^2}C\right)y+
    \frac{1}{k\beta_s}(C-1)p_y, \\
f_3 &=&\frac{L}{2\beta_s} p_x^2 \Big(C(ky+p_t)-Sp_y\Big)
    +\left(\frac{k^2L^3}{6\beta_s^5} + \frac{L}{2\beta_s^3\gamma_s^2}\right)
        \Big(C(ky+p_t) - Sp_y\Big)^3 \\
    & &-\frac{kL^2}{2\beta_s^3}\Big(C(ky+p_t) - Sp_y\Big)^2
        \Big(S(ky + p_t) - Cp_y\Big)
    +\frac{L}{2\beta_s}\Big(C(ky+p_t) - Sp_y\Big)
        \Big(S(ky + p_t) - Cp_y\Big)^2. \\
}\end{equation}       
The generator $f_4$ is ignored,
because its coefficients are all small of order one.
 
\subsection{TRANSPORT Map for an Electrostatic Separator}
\index{TRANSPORT map!separator}
\index{separator!TRANSPORT map}
\index{map!separator}
The TRANSPORT map is the expansion of the exact map to second order.
We use the abbreviations
\begin{equation}
C = \cosh(kL), \qquad S = \sinh(kL).
\end{equation}
The TRANSPORT map becomes
\begin{equation}\eqarray{
x_2   &= x_1 + L p_{x1} - \frac{L}{\beta_s} p_{x1} (k y_1 + p_{t1}),\\
p_{x2}&= p_{x1},\\
y_2   &= \frac{1}{k\beta_s}(C - 1)
        +\left(C-\frac{kL}{\beta_s^2}S\right)y_1
        +\frac{1}{k}Sp_{y1}
        +\frac{1}{k}\left(C-1-\frac{kL}{\beta_s^2}S\right)p_{t1}\\
      & +\frac{L}{2\beta_s}S(p_{x1}^2+p_{y1}^2)
        -\frac{L}{2\beta_s}C(ky_1+p_{t1})p_{y1}
        +\frac{L}{2\beta_s^3}\left(kLC+\frac{3}{\gamma_s^2}S\right)
          (ky_1+p_{t1})^2,\\
p_{y2}&= \frac{1}{\beta_s}S
        +k\left(S-\frac{kL}{\beta_s^2}C\right)y_1
        +Cp_{y1}
        +\left(S-\frac{kL}{\beta_s^2}C\right)p_{t1}\\
      & +\frac{kL}{2\beta_s}C(p_{x1}^2+p_{y1}^2)
        -\frac{kL}{2\beta_s}S(ky_1+p_{t1})p_{y1}
        +\frac{kL}{2\beta_s^3}\left(kLS+\frac{3}{\gamma_s^2}C\right)
           (ky_1+p_{t1})^2,\\
t_2   &=\frac{L\eta\delta_s}{\beta_s}-
        \left(S-\frac{kL}{\beta_s^2}C\right)y_1
        -\frac{1}{k}(C-1)p_{y1}
        +t_1
        -\frac{1}{k}\left(S-\frac{kL}{\beta_s^2}C\right)p_{t1}\\
      & -\frac{L}{2\beta_s}C(p_{x1}^2+p_{y1}^2)
        +\frac{L}{2\beta_s}S(ky_1+p_{t1})p_{y1}
        -\frac{L}{2\beta_s^3}\left(kLS+\frac{3}{\gamma_s^2}C\right)
           (ky_1+p_{t1})^2,\\
p_{t2}&=p_{t1}.\\
}\end{equation}
 
 
\section{Misalignments}
\label{misalign}
Misalignments are defined by three displacements and three angles:
\begin{mylist}
\item[$\Delta x$]
Horizontal displacement,
\item[$\Delta y$]
Vertical displacement,
\item[$\Delta s$]
Longitudinal displacement.
\item[$\theta$]
Rotation about the $s$-axis,
\item[$\phi$]
Rotation about the $x$-axis,
\item[$\psi$]
Rotation about the $y$-axis.
\end{mylist}
Lie algebraic maps for misalignments have been derived in~\cite{HEA86}
as separate maps for each of the six components of the misalignment.
In practice misalignments are not known to high precision.
For this and for speed reasons,
MAD uses a linear approximation of its effects.
The six transformations are also combined into one map.
 
\subsection{Exact Map for Misalignments}
\index{misalignment!exact map}
\index{exact map!misalignment}
The three translational components can be composed to form a vector
\begin{equation}
V = \left(\myarray{ v_1      \\ v_2      \\ v_3      }\right)
  = \left(\myarray{ \Delta x \\ \Delta y \\ \Delta s }\right).
\end{equation}
and the three rotational components are represented by an orthogonal matrix
\begin{equation}\myarray{
W&=&\left(\myarray{
w_{11} & w_{12} & w_{13} \\
w_{21} & w_{22} & w_{23} \\
w_{31} & w_{32} & w_{33} \\
}\right)\\
&=&\left(\myarray{
+\cos\theta \cos\psi-\sin\theta \sin\phi \sin\psi&
-\cos\theta \sin\psi-\sin\theta \sin\phi \cos\psi&\sin\theta \cos\phi\\
\cos\phi \sin\psi & \cos\phi \cos\psi & \sin\phi \\
-\sin\theta \cos\psi-\cos\theta \sin\phi \sin\psi&
+\sin\theta \sin\psi-\cos\theta \sin\phi \sin\psi&\cos\theta \cos\phi \\
}\right).
}\end{equation}
The \emindex{misalignment pivot},
i.~e. the point around which the rotation takes place,
is the origin of the reference system at element entrance.
The misalignment generates a canonical transformation.
With the abbreviations
\begin{equation}\eqarray{
s_2 = w_{13} (x_1 - v_1) + w_{23} (y_1 - v_2) - w_{33} v_3, \\
p_{si}=\sqrt{1+\frac{2p_{ti}}{\beta_s}+p_{ti}^2-p_{xi}^2-p_{yi}^2},\quad i = 1, 2.
}\end{equation} 
the transformation for the entrance can be written as
\begin{equation}\eqarray{
p_{x2}&=&w_{11} p_{x1} + w_{21} p_{y1} + w_{31} p_{s1}), \\
p_{y2}&=&w_{12} p_{x1} + w_{22} p_{y1} + w_{32} p_{s1}), \\
p_{t2}&=&p_{t2}.
}\end{equation}
\begin{equation}\eqarray{
x_2&=&w_{11}(x_1-v_1)+w_{21}(y_1-v_2)-w_{31}v_3-\frac{p_{x2}}{p_{s2}}s_2,\\
y_2&=&w_{12}(x_1-v_1)+w_{22}(y_1-v_2)-w_{32}v_3-\frac{p_{y2}}{p_{s2}}s_2,\\
t_2&=&t_1 + \frac{\beta_s^{-1} + p_{t2}}{p_{s2}} s_2.
}\end{equation}
 
\subsection{Linear Approximation for Misalignments}
\index{misalignment!linear map}
\index{linear map!misalignment}
The linear part for defines the transfer matrix
\begin{equation}
R=\left(\myarray{
+\frac{w_{22}}{w_{33}} & +\frac{w_{22}}{w_{33}}s &
-\frac{w_{12}}{w_{33}} & -\frac{w_{12}}{w_{33}}s &
0                      & 0 \\
0 & w_{11} &0 &w_{21} &0 & \frac{w_{31}}{\beta_s} \\
-\frac{w_{21}}{w_{33}} & -\frac{w_{21}}{w_{33}}s &
+\frac{w_{11}}{w_{33}} & +\frac{w_{11}}{w_{33}}s &
0                      & 0 \\
0 &w_{12} &0 & w_{22} &0 & \frac{w_{32}}{\beta_s} \\
\frac{w_{13}}{\beta_s w_{33}} &\frac{w_{13}}{\beta_s w_{33}}s &
\frac{w_{23}}{\beta_s w_{33}} &\frac{w_{23}}{\beta_s w_{33}}s &
1 &\frac{s}{\beta_s^2\gamma_s^2} \\
0 &0 &0 &0 &0 &1
}\right),
\end{equation}
where
\begin{equation}
s_2 = - w_{13} v_1 - w_{23} v_2 - w_{33} v_3, \\
\end{equation}
The Lie-algebraic map contains a first-order generator
\begin{equation}
f_1 = - \Big (w_{13} (x - s p_x) + w_{23} (y - s p_y)\Big) / w_{33}
      + (v_1 p_x + v_2 p_y + \beta_s^{-1} v_3 p_t),
\end{equation}
which gives rise to the kick, to be used for the TRANSPORT map:
\begin{equation}\eqarray{
\Delta x&=&- (w_{22} v_1 - w_{12} v_2) / w_{33}, &
 \qquad & \Delta p_x&=&w_{31}, \\
\Delta y&=&- (w_{11} v_2 - w_{21} v_1) / w_{33}, &
 \qquad & \Delta p_y&=&w_{32}, \\
\Delta t&=&0 & \qquad & \Delta p_t = 0.
}\end{equation}
 
\subsection{Misalignment at Exit}
\index{misalignment!linear map}
\index{linear map!misalignment}
For the misalignment at element exit the displacement and rotation
must be transformed to the new reference system as follows:
\begin{equation}
\overline{V} = W_e^{-1} (V + W V_e - V_e), \qquad
\overline{W} = W_e^{-1} W W_e,
\end{equation}
where $V_e$ and $W_e$ are the displacement vector and rotation matrix
describing the change of reference when proceeding through the element.
The inverse transformation has the transfer matrix
\begin{equation}
F=\left(\myarray{
\wbar_{11} &-\frac{\wbar_{22}}{\wbar_{33}}s &
\wbar_{21} & \frac{\wbar_{21}}{\wbar_{33}}s &
0 & \frac{\wbar_{13}}{\beta_s \wbar_{33}}s \\
0 & \frac{\wbar_{22}}{\wbar_{33}} &
0 &-\frac{\wbar_{21}}{\wbar_{33}} &
0 & \frac{\wbar_{13}}{\beta_s \wbar_{33}} \\
\wbar_{12} & \frac{\wbar_{12}}{\wbar_{33}}s &
\wbar_{22} &-\frac{\wbar_{11}}{\wbar_{33}}s &
0 &-\frac{\wbar_{23}}{\beta_s \wbar_{33}}s \\
0 &-\frac{\wbar_{12}}{\wbar_{33}} &
0 & \frac{\wbar_{11}}{\wbar_{33}} &
0 & \frac{\wbar_{23}}{\beta_s \wbar_{33}} \\
\frac{\wbar_{31}}{\beta_s} &0 &
\frac{\wbar_{32}}{\beta_s} &0 &
1 & \frac{s}{\beta_s^2\gamma_s^2} \\
0 &0 &0 &0 &0 &1
}\right),
\end{equation} 
were
\begin{equation}
s = (w_{13} v_1 + w_{23} v_2 + w_{33} v_3) / w_{33} .
\end{equation} 
The Lie-algebraic map contains the generator
\begin{equation}
f_1 = - (\wbar_{31} x + \wbar_{32} y)
- \Big((+ \wbar_{22} \vbar_1 - \wbar_{12} \vbar_2) p_x +
       (- \wbar_{21} \vbar_1 + \wbar_{11} \vbar_2) p_y \Big) / \wbar_{33},
\end{equation}
For the TRANSPORT map the orbit kicks are
\begin{equation}\eqarray{
\Delta x&=&(w_{22} v_1 - w_{12} v_2) / w_{33}, &
 \qquad & \Delta p_x&=&-w_{31}, \\
\Delta y&=&(w_{21} v_2 - w_{21} v_1) / w_{33}, &
 \qquad & \Delta p_y&=&-w_{32}, \\
\Delta t&=&- \beta_s^{-1} s, & \qquad & \Delta p_t = 0.
}\end{equation}
 
 
\section{Rotation of Reference about the $s$-Axis}
\index{rotation!map}
\index{map!rotation}
\label{srot}
A rotation about the $s$-axis by an angle $\psi$ is a linear map $\cal R$.
It is completely described by the transfer matrix
\begin{equation}
R = \left(\myarray{
 \cos\psi & 0        & \sin\psi & 0        & 0 & 0 \\
 0        & \cos\psi & 0        & \sin\psi & 0 & 0 \\
-\sin\psi & 0        & \cos\psi & 0        & 0 & 0 \\
 0        &-\sin\psi & 0        & \cos\psi & 0 & 0 \\
 0        & 0        & 0        & 0        & 1 & 0 \\ 
 0        & 0        & 0        & 0        & 0 & 1    
}\right),
\end{equation}
 
 
\section{Rotation of Reference About $y$-Axis}
\index{rotation!map}
\index{map!rotation}
A rotation about the $y$-axis is a special case of a misalignment,
derived in~\cite{HEA86}.
With the rotation angle $\phi$ we have the transfer matrix
\begin{equation}
F=\left(\myarray{
 \cos\phi               & 0          & 0 & 0 & 0 & 0                        \\
 0                      & 1/\cos\phi & 0 & 0 & 0 & -\frac{1}{\beta_s}\tan\phi \\
 0                      & 0          & 1 & 0 & 0 & 0                        \\
 0                      & 0          & 0 & 1 & 0 & 0                        \\
\frac{1}{\beta_s}\sin\phi & 0          & 0 & 0 & 1 & 0                        \\
 0                      & 0          & 0 & 0 & 0 & 1  
}\right).
\end{equation}
and only the first-order generator
\begin{equation}
r_1=- x \sin\phi
\end{equation}
is considered.
For the TRANSPORT map this represents a kick
\begin{equation}
\Delta p_x = - \tan\phi.
\end{equation}
 
 
\section{Beam-Beam Interactions}
\label{beam-beam}
 
\subsection{Lie-Algebraic Map for Beam-Beam Interaction}
\index{Lie transformation!beam-beam}
\index{beam-beam!Lie transformation}
\index{map!beam-beam}
The beam-beam interaction is not yet implemented in MAD
for the Lie algebraic formalism.
 
\subsection{TRANSPORT Map for Beam-Beam Interactions}
\index{TRANSPORT map!beam-beam}
\index{map!beam-beam}
\index{beam-beam!TRANSPORT map}
For a two-dimensional Gaussian particle distribution a closed formula
for the electric field has been given in~\cite{BAS80}.
It uses the following parameters:
\begin{mylist}
\item[$\sigma_x$]
The horizontal standard deviation of the opposite beam,
\item[$\sigma_y$]
The vertical standard deviation of the opposite beam,
\item[$\Delta x$]
The horizontal displacement of the opposite beam with respect to the
ideal orbit.
\item[$\Delta y$]
The vertical displacement of the opposite beam with respect to the
ideal orbit.
\item[$q$]
The number of unit charges per particle in the beam under consideration.
\item[$q'$]
The number of unit charges per particle in the opposite beam.
\item[$N'$]
The number of particles per unit length in the opposite beam,
or the number of particles per bunch in the opposite beam.
\item[$r_e$]
The classical particle radius.
\item[E]
The energy per particle.
\end{mylist}
Note that the electric field can be computed from a scalar potential $\phi$.
The kick acting on the particle can be computed as
\begin{equation}
K_y + i K_x =
\frac{qq'N'}{E} (\phi_y + i \phi_x) =
\frac{2\sqrt{\pi}r_eN'}{r\gamma_s} \Bigl(w(z_1) - \exp(z_2^2-z_1^2) w(z_2)\Bigr).
\end{equation}
$w$ is the complex error function
\begin{equation}
w(z)=\exp(-z^2)\left(1+\frac{2i}{\sqrt{\pi}}\int_0^z\exp(z^2)dt\right),
\end{equation}
and
\begin{equation}
\xi  = x + \Delta x, \quad
\eta = y + \Delta y, \quad
r    = \sqrt{2 (\sigma_x^2 - \sigma_y^2)}, \quad
z_1  = \frac{\xi}{r} + i \frac{\eta}{r}, \quad
z_2  = \frac{\sigma_y \xi}{\sigma_x r} + i \frac{\sigma_x \eta}{\sigma_y r}.
\end{equation}
For a round beam the above formula produces $0/0$,
and must be replaced by the limit
\begin{equation}
K_y + i K_x =
\frac{qq'N'}{E} (\phi_y + i \phi_x) =
\frac{2 i r_e N'}{\gamma_s}
\frac{1 - \exp(- (\xi^2 + \eta^2) / 2 \sigma^2)}{\xi + i\eta}.
\end{equation}                                                          
For optical calculations we need the transfer matrix with respect 
to the actual orbit.
It is the identity matrix except for the elements
\begin{equation}
R_{21} = \frac{qq'N'}{E} \phi_{xx}, \quad
R_{23} = R_{41} = \frac{qq'N'}{E} \phi_{xy}, \quad
R_{43} = \frac{qq'N'}{E} \phi_{yy}.
\end{equation}
and the second-order coefficients
\begin{equation}\eqarray{
T_{211}&=&&&&&\frac{qq'N'}{2E} \phi_{xxx}, \qquad
T_{213}&=&T_{231}&=&T_{411}&=&\frac{qq'N'}{2E} \phi_{xxy}, \\
T_{233}&=&T_{413}&=&T_{431}&=&\frac{qq'N'}{2E} \phi_{xyy}, \qquad
T_{433}&=&&&&&\frac{qq'N'}{2E} \phi_{yyy}.
}\end{equation}
which are found by differentiation of the kick.
For an elliptic beam we find
\begin{equation}\eqarray{
\phi_{xx} &=& \frac{2}{r^2} \left(- (x \phi_x + y \phi_y)
  + \rho_0 \left(1 - \frac{\sigma_y}{\sigma_x}
    \exp\left(-\frac{x^2}{2\sigma_x^2}
              -\frac{y^2}{2\sigma_y^2}\right) \right) \right), \\
%
\phi_{xy} &=& \frac{2}{r^2} \bigl(- (x \phi_y - y \phi_x) \bigr), \\
%
\phi_{yy} &=& \frac{2}{r^2} \left(+ (x \phi_x + y \phi_y)
  - \rho_0 \left(1 - \frac{\sigma_x}{\sigma_y}
    \exp\left(-\frac{x^2}{2\sigma_x^2}
              -\frac{y^2}{2\sigma_y^2}\right) \right) \right), \\
%
\phi_{xxx} &=& \frac{1}{r^2} \left(- \phi_x - (x \phi_{xx} + y \phi_{xy})
  + \rho_0 \frac{x\sigma_y}{\sigma_x^3}
    \exp\left(-\frac{x^2}{2\sigma_x^2}
              -\frac{y^2}{2\sigma_y^2}\right) \right), \\
%
\phi_{xxy} &=& \frac{1}{r^2} \left(-\phi_y-(x\phi_{xy}-y\phi_{xx})\right),\\
%
\phi_{xyy} &=& \frac{1}{r^2} \left(+\phi_x-(x\phi_{yy}-y\phi_{xy})\right),\\
%
\phi_{yyy} &=& \frac{1}{r^2} \left(+ \phi_y + (x \phi_{xy} + y \phi_{yy})
  - \rho_0 \frac{y\sigma_x}{\sigma_y^3}
    \exp\left(-\frac{x^2}{2\sigma_x^2}
              -\frac{y^2}{2\sigma_y^2}\right) \right), \\
}\end{equation}
and for a round beam
\begin{equation}\eqarray{
\phi_{xx} &=& \rho_0 \left(- \frac{x^2-y^2}{(x^2+y^2)^2} (1 - E)
  + \frac{x^2}{\sigma^2(x^2+y^2)} E \right), \\
%
\phi_{xy} &=& \rho_0 \left(- \frac{2xy}{(x^2+y^2)^2} (1 - E)
  + \frac{xy}{\sigma^2(x^2+y^2)} E \right), \\
%
\phi_{yy} &=& \rho_0 \left(+ \frac{x^2-y^2}{(x^2+y^2)^2} (1 - E)
  + \frac{y^2}{\sigma^2(x^2+y^2)} E \right), \\
%
\phi_{xxx} &=& \rho_0 \left(+ \frac{x^3-3xy^2}{(x^2+y^2)^3} (1 - E)
  - \frac{x^3-3xy^2}{2 \sigma^2 (x^2+y^2)^2} E
  - \frac{x^3}{2 \sigma^4 (x^2+y^2)} E \right), \\
%
\phi_{xxy} &=& \rho_0 \left(+ \frac{3x^2y-y^3}{(x^2+y^2)^3} (1 - E)
  - \frac{3x^2y-y^3}{2 \sigma^2 (x^2+y^2)^2} E
  - \frac{x^2y}{2 \sigma^4 (x^2+y^2)} E \right), \\
%
\phi_{xyy} &=& \rho_0 \left(- \frac{x^3-3xy^2}{(x^2+y^2)^3} (1 - E)
  + \frac{x^3-3xy^2}{2 \sigma^2 (x^2+y^2)^2} E
  - \frac{xy^2}{2 \sigma^4 (x^2+y^2)} E \right), \\
%
\phi_{yyy} &=& \rho_0 \left(- \frac{3x^2y-y^3}{(x^2+y^2)^3} (1 - E)
  + \frac{3x^2y-y^3}{2 \sigma^2 (x^2+y^2)^2} E
  - \frac{y^3}{2 \sigma^4 (x^2+y^2)} E \right), \\
}\end{equation}
where
\begin{equation}
E = \exp\left(\frac{x^2+y^2}{2\sigma^2}\right).
\end{equation}
It is easy to check that
\begin{equation}
\phi_{xx} + \phi_{yy} = \rho, \quad
\phi_{xxx} + \phi_{xyy} = \frac{\partial \rho}{\partial x}, \quad
\phi_{xxy} + \phi_{yyy} = \frac{\partial \rho}{\partial y}.
\end{equation}
where $\rho$ is the relevant ``space charge''.
 
 
% ====================================================================
 
\chapter{Uncoupled Linear Optics}
\index{linear optics!uncoupled}
\index{uncoupled motion}
In MAD lattice functions are always computed with respect to the
computed closed orbit.
Coupling is always ignored in \ttindex{OPTICS} and ignored by default
in \ttindex{TWISS}.
For a description of coupling effects refer to Chapters~\ref{transverse}
and~\ref{full}.
The methods used for uncoupled calculations are described
in~\cite{SLAC75,COU58}. 
In the commands \ttindex{TWISS} and \ttindex{OPTICS} MAD uses
TRANSPORT maps except for thin multipoles,
where the exact thin multipole maps are used.
When the closed orbit deviates from the design orbit,
MAD uses the Jacobian of the TRANSPORT map for the linear transfer
matrix.
For thick elements this may perturb the symplecticity of the map.
For this reason the Jacobian is by default made symplectic according to
Section~\ref{symplectify}.
\index{symplectification}
Symplectification is controlled by the option \ttindex{SYMPLEC}.
 
 
\section{Courant-Snyder Lattice Functions}
\label{courant}
\index{Courant}\index{Snyder}
\index{lattice functions!Courant-Snyder}
By default the \ttindex{TWISS} and \ttindex{OPTICS} commands track the
lattice functions for {\em periodic} initial conditions.
Any initial conditions specified (except~$\beta_x$ and~$\beta_y$)
override the periodic initial conditions.
 
However, if an initial condition is specified for at least one
of $\beta_x$ and $\beta_y$, the closed orbit is not computed, and MAD
tracks the lattice functions for the initial conditions specified.
Unspecified initial conditions are set to zero in this case.
Unwary users may fool MAD into thinking that a value has been given
for~$\beta_x$.
The most frequent case is entering the \ttindex{CENTRE} flag on an
{\tt OPTICS} command in the American spelling:
\myxmp{OPTICS, CENTER ! should read OPTICS, CENTRE}
The decoder assumes in this case that {\tt CENTRE} is the name of an
undefined global parameter,
sets it to zero,
and stores it in the slot for the initial value of~$\beta_x$.
 
\subsection{Initial Conditions for the Periodic Case}
When the \ttindex{COUPLE} option is not set, MAD computes the lattice
functions as defined in~\cite{COU58}.
Note that the integer parts of the betatron phases may be wrong when a
beam line contains \ttindex{LUMP}~elements, or when it contains
negative element lengths,
as in these cases an incorrect branch of the arctangent function may
be taken.
 
For the functions $\beta,\alpha,\mu$ MAD uses the relevant diagonal
blocks of the transfer matrix, denoted below as
\begin{equation}
R = \left( \myarray{R_{11} & R_{12} \\ R_{21} & R_{22}} \right)
\end{equation}
For optimal numeric precision MAD uses the following formulas
for the tunes:
\begin{equation}\eqarray{
\cos\mu&=&(R_{11}+R_{22})/2,\\
\sin\mu&=&\sign R_{12}\cdot\sqrt{R_{12}R_{21}-(R_{11}-R_{22})^2/4},\\
Q      &=&\frac{1}{2\pi}\arctan \frac{\sin\mu}{\cos\mu} \\
}\end{equation}
and the following for the initial lattice functions:
\begin{equation}\eqarray{
\beta_0 &=&R_{12}/\sin\mu,\\
\alpha_0&=&(R_{11}-R_{22})/(2\sin\mu).
}\end{equation}
The way $\sin\mu$ is computed greatly improves the numerical accuracy
for~$\mu$ when its value is close to $1/2$.
 
\subsection{Tracking the Lattice Functions}
The formulas for advancing through an element are well known:
\begin{equation}\eqarray{
\beta_2&=&\frac{1}{\beta_1}
  \left((R_{11}\beta_1-R_{12}\alpha_1)^2+R_{12}^2\right), \\
\alpha_2&=&-\frac{1}{\beta_1}
  \left((R_{11}\beta_1-R_{12}\alpha_1)(R_{21}\beta_1-R_{22}\alpha_1)+
   R_{12}R_{22}\right), \\
\mu_2&=&\mu_1+\arctan\frac{R_{12}}{R_{11}\beta_1-R_{12}\alpha_1}.
}\end{equation}
They are valid for both planes.
 
 
\section{Dispersion}
\index{dispersion}
MAD computes the dispersion only for static machines,
i.~e. for machines whose momentum is constant.
 
\subsection{Initial Values for Dispersion}
Knowing the TRANSPORT map with respect to the closed orbit for one
turn enables us to find two derivatives of the closed orbit with
respect to $p_t$,
named the first- and second-order dispersions.
Defining
\begin{equation}
D = d/dp_t
\end{equation}
the first derivative of the closed orbit
\begin{equation}
Z_0 = \left ( \myarray{
   x_0 \\ p_{x0} \\ y_0 \\ p_{y0} \\ t_0 \\ p_{t0}
} \right ),
\end{equation}
the first-order dispersion is its first derivative with respect to
$p_t$:
\begin{equation}
DZ_0 = \left ( \myarray{
   d x/dp_t \\
   d p_x/dp_t \\
   d y/dp_t \\
   d p_y/dp_t \\
   0 \\
   dp_{t0}/dp_t
} \right ) = \left ( \myarray{
   Dx \\ Dp_x \\ Dy \\ Dp_y \\ 0 \\ 1
} \right ) = \left ( \myarray{
   Dz_1 \\ Dz_2 \\ Dz_3 \\ Dz_4 \\ Dz_5 \\ Dz_6
} \right ).
\end{equation}
The second derivative of the closed orbit with respect to $p_t$ is
the second-order dispersion:
\begin{equation}
D^2Z_0 = \left ( \myarray{
   D^2z_1 \\ D^2z_2 \\ D^2z_3 \\ D^2z_4 \\ D^2z_5 \\ D^2z_6
} \right ) = \left ( \myarray{
   d^2 x/dp_t^2 \\
   d^2 p_x/dp_t^2 \\
   d^2 y/dp_t^2 \\
   d^2 p_y/dp_t^2 \\
   0 \\
   0
} \right ) = \left ( \myarray{
   D^2x \\ D^2p_x \\ D^2y \\ D^2p_y \\ 0 \\ 1
} \right )  
\end{equation}
The dispersions are computed following a method developed
in~\cite{PEG81}.
With respect to the closed orbit the orbit behaves as
\begin{equation}
Z=Z_0+p_t\cdot DZ_0+\frac{p_t^2}{2}\cdot D^2Z_0+o(p_t^3)
\end{equation}
The vectors $DZ_0$ and $D^2Z_0$ are found by substitution of this
expression in the TRANSPORT map and by separation of like powers
of~$p_t$.
Substitution produces
\begin{equation}\eqarray{
z_k+p_t\cdot Dz_k+\frac{p_t^2}{2}\cdot D^2z_k=
  \sum_{l=1}^6 R_{kl}
    (z_l+p_t\cdot Dz_l+\frac{p_t^2}{2}\cdot D^2z_l)+\\
\qquad\sum_{l=1}^6 \sum_{m=1}^6 T_{klm}
    (z_l+p_t\cdot Dz_l+\frac{p_t^2}{2}\cdot D^2z_l)
    (z_m+p_t\cdot Dz_m+\frac{p_t^2}{2}\cdot D^2z_m)+
  o(p_t^3).
}\end{equation}
The terms in $p_t$ are:
\begin{equation}
Dz_k = \sum_{l=1}^6 R_{kl} Dz_l
\end{equation}
Using $Dz_5=0, Dz_6=1$ and defining
\begin{equation}
A = \left ( \myarray{
R_{11} & R_{12} & R_{13} & R_{14} \\
R_{21} & R_{22} & R_{23} & R_{24} \\
R_{31} & R_{32} & R_{33} & R_{34} \\
R_{41} & R_{42} & R_{43} & R_{44} \\
} \right ),
\end{equation}
we obtain
\begin{equation}
\left ( \myarray{ Dz_1 \\ Dz_2 \\ Dz_3 \\ Dz_4} \right ) =
(A - I)^{-1} 
\left ( \myarray{ R_{16} \\ R_{26} \\ R_{36} \\ R_{46}} \right ).
\end{equation}
The terms in $p_t^2$ give
\begin{equation}
D^2z_k = \sum_{l=1}^6 R_{kl} D^2z_l +
  2 \sum_{l=1}^6 \sum_{m=1}^6 T_{klm} Dz_l Dz_m,
\end{equation}
thus
\begin{equation}
\left ( \myarray{ D^2z_1 \\ D^2z_2 \\ D^2z_3 \\ D^2z_4} \right ) =
2 (A - I)^{-1} \left ( \myarray{
  \sum_{l=1}^6 \sum_{m=1}^6 T_{1l m} Dz_l Dz_m \\
  \sum_{l=1}^6 \sum_{m=1}^6 T_{2l m} Dz_l Dz_m \\
  \sum_{l=1}^6 \sum_{m=1}^6 T_{3l m} Dz_l Dz_m \\
  \sum_{l=1}^6 \sum_{m=1}^6 T_{4l m} Dz_l Dz_m
} \right ).
\end{equation}
 
\subsection{Tracking the Dispersion}
\label{deriveR}
The first-order dispersion is tracked through an element as follows:
\begin{equation}
Dz \leftarrow R Dz.
\end{equation}
By differentiation we find for the second-order dispersion:
\begin{equation}
D^2z \leftarrow R D^2z + DR Dz.
\end{equation}
The total derivative~$DR=dR/dp_t$ of the transfer matrix must take
into account the displacement of the orbit due to the dispersion.
The particle orbit can be related to the dispersion orbit $Dz$ by
\begin{equation}
Z = Z_0 + p_t\cdot Dz + \frac{p_t^2}{2} D^2Z.
\end{equation}
It transforms according to the equation
\begin{equation}
z_k+p_t\cdot Dz_k+\ldots\leftarrow
  \sum_{l=1}^6 R_{kl} (z_l+p_t\cdot Dz_l+\ldots) +
  \sum_{l=1}^6 \sum_{m=1}^6 T_{klm}
    (z_l+p_t\cdot Dz_l+\ldots)(z_m+p_t\cdot Dz_m+\ldots).
\end{equation}
The total derivative of the transfer matrix is found by partial
differentiation as
\begin{equation}
DR_{kl} = \frac{dR_{kl}}{dp_t} = 2 \sum_{m=1}^6 T_{klm} Dz_m.
\end{equation}
 
 
\section{Chromatic Effects}
\index{chromatic functions}
\index{lattice functions!chromatic}
The chromatic effects are computed for the uncoupled case only.
 
\subsection{Initial Values for Chromatic Functions}
\label{chrom}
Using the total derivative of the transfer matrix for one turn we find
the following equations:
\begin{equation}\eqarray{
\frac{1}{2}(DR_{11}+DR_{22})&=&D(\cos\mu)&=&-D\mu\sin\mu,\\
DR_{12}&=&D(\beta\sin\mu)&=&D\beta \sin\mu-\beta D\mu\cos\mu,\\
\frac{1}{2}(DR_{11}-DR_{22})&=&D(\alpha\sin\mu)&=&
D\alpha \sin\mu-\alpha D\mu\cos\mu.\\
}\end{equation}
The derivative~$DR = dR/dp_t$ of~$R$ has been given in Section~\ref{deriveR}.
From these we derive the chromaticity
\begin{equation}\eqarray{
D\mu&=&\frac{DR_{11}+DR_{22}}{2\sin\mu},\\
DQ  &=&\frac{D\mu}{2\pi},\\
}\end{equation}
and the initial values for the chromatic functions
\begin{equation}\eqarray{
D\beta_0 &=&(DR_{12}+\beta_0 D\mu\cos\mu) / \sin\mu,\\
D\alpha_0&=&(DR_{11}-DR_{22}+2\alpha_0 D\mu\cos\mu) / (2\sin\mu),\\
B_0      &=&D\beta_0 / \beta_0,\\
A_0      &=&(D\alpha_0\beta_0-D\beta_0\alpha_0) / \beta_0,\\
W_0      &=&\sqrt{B_0^2+A_0^2},\\
\Phi_0   &=&\arctan(A_0/B_0).
}\end{equation}
 
\subsection{Tracking the Chromatic Functions}
The derivative of the phase advance by~$p_t$ is found easily:
\begin{equation}
D\mu_2=D\mu_2=D\mu_1+
  \frac{(R_{11}DR_{12}-R_{12}DR_{11})\beta_1-
        R_{12}(R_{11}D\beta-R_{12}D\alpha)}
       {(R_{11}\beta-R_{12}\alpha)^2+R_{P12}^2}.
\end{equation}
Given the values~$W_1$ and~$\Phi_1$ one may write
\begin{equation}\eqarray{
A_1      &=&W_1\cos\Phi_1,\\
B_1      &=&W_1\sin\Phi_1,\\
D\beta_1 &=&B_1\beta_1,\\
D\alpha_1&=&A_1+B_1\alpha_1,\\
B_2      &=&\frac{1}{\beta_1\beta_2}\biggl(
            \Bigl(\bigl(R_{11}\beta_1-R_{12}\alpha_1\bigr)^2-R_{12}^2\Bigr)B_1-
                  2\bigl(R_{11}\beta_1-R_{12}\alpha_1\bigr)R_{12}A_1\biggr)\\
         &+&\frac{2}{\beta_2}\Bigl(
            \bigl(R_{11}\beta_1-R_{12}\alpha_1\bigr)DR_{11}-
            \bigl(R_{11}\alpha_1-R_{12}\gamma_1\bigr)DR_{12}\Bigr),\\
A_2      &=&\frac{1}{\beta_1\beta_2}\biggl(
            \Bigl(\bigl(R_{11}\beta_1-R_{12}\alpha_1\bigr)^2-R_{12}^2\Bigr)A_1+
                  2\bigl(R_{11}\beta_1-R_{12}\alpha_1\bigr)R_{12}B_1\biggr)\\
         &-&\frac{1}{\beta_2}\Bigl(
                  \bigl(R_{11}\beta_1-R_{12}\alpha_1\bigr)
                  \bigl(DR_{11}\alpha_2+DR_{21}\beta_2\bigr)-
                  \bigl(R_{11}\alpha_1-R_{12}\gamma_1\bigr)
                  \bigl(DR_{12}\alpha_2+DR_{22}\beta_2\bigr)\\
         & &\qquad+\bigl(R_{11}DR_{12}-R_{12}DR_{11}\bigr)\Bigr),\\
W_2      &=&\sqrt{B_2^2+A_2^2},\\
\Phi_2   &=&\arctan\frac{A_2}{B_2}.\\
}\end{equation}
Note that the ``partial'' chromaticity $D\mu_2-D\mu_1$ for a piece of
the ring depends not only on the transfer map, but also on the initial
values~$D\beta_1$ and~$D\alpha_1$.
 
 
% ====================================================================
 
\chapter{Transverse Coupling (Method by Edwards and Teng)}
\label{transverse}
\index{lattice functions!Edwards-Teng}
\index{Edwards}\index{Teng}
\section{Initial Values}
 
When the \ttindex{COUPLE} option is set, the \ttindex{TWISS} command
uses a method similar to reference~\cite{TEN71}.
Consider the linear transfer map $\mathrm{\bf M}$ in {\em two}
degrees of freedom partitioned into four $2\times2$~blocks:
\begin{equation}
\mathrm{\bf M}=\left(\myarray{
m_{11} & m_{12} & m_{13} & m_{14} \\
m_{21} & m_{22} & m_{23} & m_{24} \\
m_{31} & m_{32} & m_{33} & m_{34} \\
m_{41} & m_{42} & m_{43} & m_{44} \\
}\right) = \left(\myarray{A & B \\ C & D}\right).
\end{equation}
The 4-dimensional phase space vector shall also be partitioned
according to the horizontal and vertical planes.
Edwards and Teng introduce a ``symplectic rotation''
\begin{equation}
\mathrm{\bf R}=\left(\myarray{
  I\cos\phi & \overline{R}\sin\phi \\ -R\sin\phi & I\cos\phi
}\right)
\end{equation}
$R$~is a $2 \times 2$~matrix with unit determinant,
and~$\overline{R}$ denotes its symplectic conjugate:
\begin{equation}
R = \left( \myarray{ a & b \\ c & d } \right), \qquad
|R| = \left| \myarray{ a & b \\ c & d } \right| = 1, \qquad
\overline{R} = \left( \myarray{ d & -b \\ -c & a } \right ).
\end{equation}
This leaves three free parameters for the elements of $R$,
and a fourth parameter $\phi$.
Edwards and Teng then determine $\mathrm{\bf R}$ such
that~$\mathrm{\bf M}$ conjugated with~$\mathrm{\bf R}$ becomes block
diagonal:
\begin{equation}
\mathrm{\bf R} \mathrm{\bf M} \mathrm{\bf R}^{-1} =
  \left( \myarray{ E & 0 \\ 0 & F } \right)
\end{equation}
If $|B + \overline{C}| < 0$ both $\phi$ and all elements of $R$ become
imaginary.
This may be avoided by redefining
\begin{equation}
\mathrm{\bf R}=\frac{1}{\sqrt{1+|R|}}\left(\myarray{I&\overline{R}\\
  -R&I}\right).
\end{equation}
where all four elements of $R$ are free parameters.
The solutions is:
\begin{equation}\eqarray{
R&=&-\left(\frac{1}{2}(\Tr A-\Tr D)+\sign(|B+\overline{C}|)
  \sqrt{|B + \overline{C}| + \frac{1}{4}(\Tr A - \Tr D)^2}\right)^{-1}
  \left(B + \overline{C}\right), \\
E &=& A - B R, \qquad
F = D + \overline{R} C.
}\end{equation}
The block diagonal matrix can be parametrised as usual.
From the eigenvectors of the conjugated system
\begin{equation}\eqarray{
V_1&=&\left(\myarray{\sqrt{\beta_1}&0\\
  \frac{\alpha_1}{\sqrt{\beta_1}}&\frac{1}{\sqrt{\beta_1}}}\right),\qquad
V_2&=&\left(\myarray{\sqrt{\beta_2}&0\\
  \frac{\alpha_2}{\sqrt{\beta_2}}&\frac{2}{\sqrt{\beta_2}}}\right)
}\end{equation}
one may find the eigenvectors of the coupled system:
\begin{equation}\eqarray{
\mathrm{\bf V}_1&=&\frac{1}{\sqrt{1+|R|}}
\left(\myarray{V_1\\\overline{R}V_1}\right),\qquad
\mathrm{\bf V}_2&=&\frac{1}{\sqrt{1+|R|}}
\left(\myarray{-RV_2\\V_2}\right).
}\end{equation}
 
\section{Tracking the Edwards-Teng Functions}
 
For tracking the coupled lattice functions we assume that the
transfer matrix for one element is partitioned as above:
\begin{equation}
\mathrm{\bf R}_e = \left( \myarray{ A_e & B_e \\ C_e & D_e } \right),
\end{equation}
The symplectic rotation at element entrance changes the diagonal
blocks to
\begin{equation}
E_e = (A_e - B_e R_1) / \sqrt{|A_e - B_e R_1|}, \qquad
F_e = (D_e + \overline{R}_1 C_e) / \sqrt{|A_e - B_e R_1|}, \qquad
\end{equation}
and the new coupling matrix at exit becomes
\begin{equation}
R_2 = - (C_e - D_e R_1) \overline{(A_e - B_e R_1)} / |A_e - B_e R_1|.
\end{equation}
We may track the decoupled lattice functions using the
matrices~$E_e$ for mode~1 and~$F_e$ for mode~2.
 
 
% ====================================================================
 
\chapter{Fully Coupled Motion}
\index{coupled motion}
\index{lattice functions!coupled}
\label{full}
 
Throughout this chapter we make the assumption that all eigenvalues 
$\lambda_k$ of a real symplectic matrix $\bold{M}$ are distinct. 
This implies that the eigenvectors $\bold{v}_k$ are all linearly
independent and form a basis in $2N$-dimensional space. 
Most facts derived remain valid when this condition is relaxed, 
but the proofs become much more difficult.
 
 
\section{Eigenvectors of a $2N \times 2N$ Symplectic Matrix}
\label{complex}
\index{eigenvectors!complex}
An eigenvector of a general matrix $\bold{M}$ obeys the relation
\begin{equation}
\bold{M} \bold{v}_k = \lambda_k \bold{v}_k.
\end{equation}
where $\lambda_k$ is the corresponding eigenvalue.
If all eigenvalues $\lambda_k$ are distinct
the eigenvectors $\bold{v}_k$ are all linearly independent
and form a basis in $2N$-dimensional space.
Thus an arbitrary vector $\bold{z}$ can be uniquely decomposed in
terms of the eigenvectors:
\begin{equation}
\bold{z} = \sum_{k=1}^{2N} c_k \bold{v}_k = \bold{V} \bold{c},
\end{equation}
where the eigenvectors are arranged as the column vectors of a matrix
$\bold{V}$.
The vector
\begin{equation}
\bold{c} = \left(\begin{array}{c}
c_1 \\ c_2 \\ \vdots \\ c_{2N} \end{array} \right) =
\bold{V}^{-1} \bold{z},
\end{equation}
is unique.
Using the diagonal matrix formed from the eigenvalues
\begin{equation}
\Lambda = \left( \begin{array}{ccc}
\lambda_1 & & \\ & \ddots & \\ & & \lambda_{2n}
\end{array} \right),
\end{equation}
the linear transformation for $\bold{z}$ can be written as
\begin{equation}
\bold{M} \bold{z} = \bold{M} \bold{V} \bold{c} =
(\bold{M} \bold{V}) \bold{c} =
(\bold{V} \Lambda) (\bold{V}^{-1} \bold{z}) =
\bold{V} \Lambda \bold{V}^{-1} \bold{z}.
\end{equation}
Since $\bold{z}$ was arbitrary, there must hold an identity
\begin{equation}
\bold{M} = \bold{V} \Lambda \bold{V}^{-1}.
\end{equation}
This representation is known as the Jordan normal form of the matrix.
By definition two eigenvectors $\bold{v}_k$ and $\bold{v}_l$ obey
the equations
\begin{equation}
\bold{M} \bold{v}_k = \lambda_k \bold{v}_k, \qquad
\bold{M} \bold{v}_l = \lambda_l \bold{v}_l.
\end{equation}
The bilinear form
\begin{equation}
(\bold{M} \bold{v}_k)^T \bold{S} (\bold{M} \bold{v}_l) =
(\lambda_k \bold{v}_k)^T \bold{S} (\lambda_l \bold{v}_l),
\end{equation}
built using the matrix
\begin{equation}
\bold{S} = \left( \begin{array}{rrrrr}
0 & 1 & & & \\ -1 & 0 & & & \\ & & \ddots & & \\
 & & & 0 & 1 \\ & & & -1 & 0
\end{array} \right)
\end{equation}
is asymmetric, therefore it vanishes identically for $k=l$.
Evaluation of both sides of the equation and use of the symplecticity of
the transfer matrix $\bold{M}$ leads to
\begin{equation}
\bold{v}_k^T (\bold{M}^T \bold{S} \bold{M}) \bold{v}_l =
\bold{v}_k^T \bold{S} \bold{v}_l =
\lambda_k \lambda_l \bold{v}_k^T \bold{S} \bold{v}_l.
\label{form}
\end{equation}
This equation can only be true if either $\lambda_k \lambda_l = 1$
or if $\bold{v}_k^T S \bold{v}_l = 0$.
 
It is well known that a symplectic matrix with all eigenvalues distinct
has $N$~pairs of eigenvalues for which $\lambda_k\lambda_l=1$.
If the motion is stable, these eigenvalue and the corresponding
eigenvectors also form complex conjugate pairs.
Due to the asymmetry of the bilinear form for all $k$
\begin{equation}
\bold{v}_k^T \bold{S} \bold{v}_k \equiv 0.
\end{equation}
For all $k$ the vector $\bold{v}_k^*$ is an eigenvector with the
eigenvalue $\lambda_k^*$.
Inserting the eigenvectors $\bold{v}_k$ and $\bold{v}_k^*$ in
equation~\ref{form} one gets an identity, and 
\begin{equation}
\bold{v}_k^T \bold{S} \bold{v}_k^* = \hbox{\rm arbitrary}.
\end{equation}
where the arbitrary quantity on the right-hand side can be chosen freely.
For all other pairs, $\lambda_k \lambda_l \ne 1$,
and therefore equation~\ref{form} leads to the identities
\begin{equation}
\bold{v}_k^T \bold{S} \bold{v}_l \equiv 0, \qquad
\bold{v}_k^T \bold{S} \bold{v}_l^* \equiv 0.
\end{equation} 
 
 
\section{Everything is Real ...}
\label{real}
\index{eigenvectors!components}
Since the transfer matrix is real,
the eigenvalues and eigenvectors form complex conjugate pairs.
Decomposing the eigenvalues and eigenvectors into their real and
imaginary parts:
\begin{equation}
\bold{v}_k = \bold{a}_k \pm i \bold{b}_k, \qquad
\lambda_k = \cos \mu_k \pm i \sin \mu_k.
\end{equation}
For stable motion the $\mu_k$ are real,
but the following formulae are also valid if this is not the case.
The definitions of the eigenvectors become
\begin{equation} \label{realvec}
\bold{M} \bold{v}_k = \bold{M} ( \bold{a}_k+i \bold{b}_k ) =
(\cos\mu_k\bold{a}_k - \sin\mu_k\bold{b}_k)
 + i (\cos\mu_k\bold{b}_k + \sin\mu_k \bold{a}_k).
\end{equation}
The ``real eigenvector matrix''
\begin{equation} \label{realmat}
\bold{W} = \left( \begin{array}{ccccccc}
a_{1,1} & b_{1,1} & a_{2,1} & b_{2,1} &        & a_{N,1} & b_{N,1} \\
a_{1,2} & b_{1,2} & a_{2,2} & b_{2,2} &        & a_{N,2} & b_{N,2} \\
\vdots  & \vdots  & \vdots  & \vdots  & \cdots & \vdots  & \vdots  \\
a_{1,2N}& b_{1,2N}& a_{2,2N}& b_{2,2N}&        & a_{N,2N}& b_{N,2N}
\end{array} \right),
\end{equation}
is built by arranging the real and imaginary parts of the eigenvectors
as alternating columns of the matrix.
From Equation~\ref{realvec} and~\ref{realmat} it is obvious that the
matrix $\bold{W}$ transforms like
\begin{equation}
\bold{M} \bold{W} = \bold{W} \bold{R},
\end{equation}
where $\bold{R}$ is the matrix
\begin{equation}
\bold{R} = \left( \begin{array}{rrrrr}
\cos\mu_1 & \sin\mu_1 & & & \\ -\sin\mu_1 & \cos\mu_1 & & & \\
& & \ddots & & \\
& & & \cos\mu_N & \sin\mu_N \\ & & & -\sin\mu_N & \cos\mu_N
\end{array} \right), 
\end{equation}
which rotates the phase space coordinates pairwise.
Again an arbitrary vector $\bold{z}$ can be written uniquely as
\begin{equation}
\bold{z} = \bold{W} \bold{d}, \quad \Rightarrow \quad
\bold{d} = \bold{W}^{-1} \bold{z}.
\end{equation}
In the same way as for the complex form one finds
\begin{equation}
\bold{M} \bold{z} = \bold{M} \bold{W} \bold{d} =
(\bold{M} \bold{W}) \bold{d} =
(\bold{W} \bold{R}) (\bold{W}^{-1} \bold{z}) =
\bold{W} \bold{R} \bold{W}^{-1} \bold{z}.
\end{equation}
and, since $\bold{z}$ is arbitrary,
the transfer matrix $\bold{M}$ has the real
representation
\begin{equation}
\bold{M} = \bold{W} \bold{R} \bold{W}^{-1}.
\end{equation}
This is real equivalent to the Jordan normal form.
The linear transformation for one turn around the machine can be
written as
\begin{equation}
\bold{z}(C) = \bold{M} \bold{z}(0) =
\bold{W} \bold{R} \bold{W}^{-1} \bold{z}(0),
\end{equation}
Pre-multiplying with $\bold{W}^{-1}$ gives
\begin{equation}
\bold{W}^{-1} \bold{z}(C) = \bold{R} \bold{W}^{-1} \bold{z}(0).
\end{equation}
Now define the vector of {\em normalised coordinates}
\begin{equation}
\bold{u} = \bold{W}^{-1} \bold{z} = 
\left( \begin{array}{c}
u_1 \\ u_2 \\ \vdots \\ u_{2N-1} \\ u_{2N}
\end{array} \right),
\end{equation}
and partition this vector into pairs of coordinates.
Each pair is transformed under the action of $\bold{R}$ like
\begin{equation}
\left( \begin{array}{l}
u_{2k-1}(C) \\ u_{2k}(C)
\end{array} \right) = \left( \begin{array}{rcl}
u_{2k-1}(0) \cos\mu_k &+& u_{2k}(0) \sin\mu_k \\
-u_{2k-1}(0) \sin\mu_k &+& u_{2k}(0) \cos\mu_k \\
\end{array} \right).
\end{equation}
Thus the quantities $I_k = \frac{1}{2}(u_{2k-1}^2 + u_{2k}^2)$ are
invariant under the motion.
 
In Section~\ref{complex} we introduced the bilinear form
\begin{equation}
\bold{v}_k^T \bold{S} \bold{v}_l =
(\bold{a}_k + i \bold{b}_k)^T \bold{S} (\bold{a}_l + i \bold{b}_l) =
(\bold{a}_k^T \bold{S} \bold{a}_l - \bold{b}_k^T \bold{S} \bold{b}_l) +
i (\bold{a}_k^T \bold{S} \bold{b}_l + \bold{b}_k^T \bold{S} \bold{a}_l).
\end{equation}
For various choices of $k$ and $l$ it has the values
\begin{equation}\begin{array}{ccccccc}
\bold{v}_k^T \bold{S} \bold{v}_k &=&
(\bold{a}_k^T \bold{S} \bold{a}_k - \bold{b}_k^T \bold{S} \bold{b}_k) &+&
i (+ \bold{a}_k^T \bold{S} \bold{b}_k + \bold{b}_k^T \bold{S} \bold{a}_k)
&=& 0, \\
\bold{v}_k^T \bold{S} \bold{v}_k^* &=&
(\bold{a}_k^T \bold{S} \bold{a}_k + \bold{b}_k^T \bold{S} \bold{b}_k) &+&
i (- \bold{a}_k^T \bold{S} \bold{b}_k + \bold{b}_k^T \bold{S} \bold{a}_k)
&=& -2i, \\
\bold{v}_k^T \bold{S} \bold{v}_l &=&
(\bold{a}_k^T \bold{S} \bold{a}_l-\bold{b}_k^T \bold{S} \bold{b}_l) &+&
i (+ \bold{a}_k^T \bold{S} \bold{b}_l+\bold{b}_k^T \bold{S} \bold{a}_l)
&=& 0, \\
\bold{v}_k^T \bold{S} \bold{v}_l^* &=&
(\bold{a}_k^T \bold{S} \bold{a}_l+\bold{b}_k^T \bold{S} \bold{b}_l) &+&
i (- \bold{a}_k^T \bold{S} \bold{b}_l+\bold{b}_k^T \bold{S} \bold{a}_l)
&=& 0,
\end{array}\label{form1}\end{equation}
where the right-hand side of the second equation has been chosen
so as to achieve a simple normalisation of the eigenvectors.
Note that this still leaves us free to multiply each eigenvector by a
phase factor $\exp(\mu)$ which will be determined later.
Solving equations~\ref{form1} for $k = l$ we obtain:
\begin{equation}
\bold{a}_k^T \bold{S} \bold{a}_k = \bold{b}_k^T \bold{S} \bold{b}_k = 0, \qquad
\bold{a}_k^T \bold{S} \bold{b}_k = - \bold{b}_k^T \bold{S} \bold{a}_k = 1.
\end{equation}
and for $k \ne l$:
\begin{equation}
\bold{a}_k^T \bold{S} \bold{a}_l = \bold{a}_k^T \bold{S} \bold{b}_l =
\bold{b}_k^T \bold{S} \bold{a}_l = \bold{b}_k^T \bold{S} \bold{b}_l = 0.
\end{equation}
These results can be combined in matrix form and result in the
condition for $\bold{W}$
\begin{equation}
\bold{W}^T \bold{S} \bold{W} = \bold{S}.
\end{equation}
It turns out that with $\bold{W}$ is symplectic under the
normalisation chosen.
Since $\bold{W}$ is symplectic, its inverse is
\begin{equation}
\bold{W}^{-1} = \bold{S}^T \bold{W}^T \bold{S} =
\bold{S}^T \left( \begin{array}{cccc}
a_{1,1} & a_{1,2} & \cdots & a_{1,2N} \\
b_{1,1} & b_{1,2} & \cdots & b_{1,2N} \\
a_{2,1} & a_{2,2} & \cdots & a_{2,2N} \\
b_{2,1} & b_{2,2} & \cdots & b_{2,2N} \\
        & \vdots &          \\
a_{N,1} & a_{N,2} & \cdots & a_{N,2N} \\
b_{N,1} & b_{N,2} & \cdots & b_{N,2N}
\end{array} \right) \bold{S}.
\end{equation}
The components of $\bold{u} = \bold{W}^{-1} \bold{z}$ are
\begin{equation}
u_{2k-1} = \bold{b}_k^T \bold{S} \bold{z}, \qquad
u_{2k} = - \bold{a}_k^T \bold{S} \bold{z},
\end{equation}
and the invariants become
\begin{equation}
I_k =
\frac{1}{2} \left(
        (\bold{b}_k^T \bold{S} \bold{z})^2 +
        (\bold{a}_k^T \bold{S} \bold{z})^2
\right) = \frac{1}{2} \left(
        \left(\sum_{i=1}^N (a_{k,2i-1} \bold{z}_{2i} -
                a_{k,2i} \bold{z}_{2i-1})\right)^2 +
        \left(\sum_{i=1}^N (b_{k,2i-1} \bold{z}_{2i} -
                b_{k,2i} \bold{z}_{2i-1})\right)^2
\right).
\end{equation} 
 
 
\section{Linear Invariants}
\index{invariants!linear}
\index{linear invariants}
Cutting the ring in two positions $1$ and $2$ produces two pieces with
the matrices $\bold{M}_{1 \rightarrow 2}$ and $\bold{M}_{2 \rightarrow 1}$.
The one-turn matrices for the two positions are
\begin{equation}
\bold{M}_1 = \bold{M}_{2 \rightarrow 1} \bold{M}_{1 \rightarrow 2} = 
\bold{W}_1 \bold{R}_1 \bold{W}_1^{-1}, \qquad
\bold{M}_2 = \bold{M}_{1 \rightarrow 2} \bold{M}_{2 \rightarrow 1} =
\bold{W}_2 \bold{R}_2 \bold{W}_2^{-1}.
\end{equation}
The two matrices are similar to each other:
\begin{equation}
\bold{M}_2 = \bold{M}_{1 \rightarrow 2} \bold{M}_1 \bold{M}_{1 \rightarrow
2}^{-1},
\end{equation}
Therefore they have the same eigenvalues, and $\bold{R}_1 = \bold{R}_2$.
Also,
\begin{equation}
\bold{M}_2 = \bold{W}_2 \bold{R}_2 \bold{W}_2^{-1} =
\bold{M}_{1\rightarrow 2} \bold{W}_1 \bold{R}_1 \bold{W}_1^{-1}
\bold{M}_{1 \rightarrow 2}^{-1}.
\end{equation}
One sees easily that
\begin{equation}
\bold{W}_2 = \bold{M}_{1 \rightarrow 2} \bold{W}_1.
\end{equation}
The transformation for $\bold{u}$ from position $1$ to $2$ is
\begin{equation}
\bold{u}_2 = \bold{W}_2^{-1} \bold{z}_2 =
\bold{W}_2^{-1} \bold{M}_{1 \rightarrow 2} \bold{z}_1 =
\bold{W}_2^{-1} \bold{M}_{1 \rightarrow 2} \bold{W}_1 \bold{u}_1 =
(\bold{M}_{1 \rightarrow 2}\bold{W}_1)^{-1}
(\bold{M}_{1 \rightarrow 2}\bold{W}_1)\bold{u}_1 = \bold{u}_1.
\end{equation}
Obviously the invariants are conserved.
For each mode the motion can be written as
\begin{equation}
\bold{z}_k = \sqrt{2 I_k} \Re \left( \bold{v}_k^*(s) e^{i \psi_k} \right),
\end{equation}
or expressed in real quantities
\begin{equation}
\bold{z}_k = \sqrt{2 I_k} \left(
        \bold{a}_k(s) \cos\psi_k +
        \bold{b}_k(s) \sin\psi_k
\right),
\end{equation}
where $\psi_k$ is the {\em initial phase} of the motion.
For a given value of the invariant the particle is found on an ellipse in
$2N$-dimensional phase space.
The plane of the ellipse can have any orientation with respect to the
$2N$~coordinate axes.
Note that there is {\em no} plane of motion in physical space.
However, the projections of this ellipse onto a plane spanned by two
phase space coordinates can be easily determined,
as we shall see below. 
 
 
\section{Phase Factors}
\index{phase factors}
We found $\bold{u}(s) = \bold{u}(0)$,
which seems to imply that the phases $\mu_k$ are constant,
but we also found $\bold{u}(C) = \bold{R} \bold{u}(0)$.
This is only an apparent contradiction,
since the eigenvectors contain the phase advances.
 
In general the projections of an eigenmode onto the $2N$~axes of
phase space all have different phases relative to the initial
phase~$\psi_k$.
However, all projections advance by the same phase difference,
known as the tune, when moving once around the ring.
For one degree of freedom~\cite{COU58} it is customary to force
$b_{1,1}=0$ along the ring by multiplying with a phase factor
$\exp(\mu(s))$.
To generalise to $N$~degrees of freedom we associate each eigenvector
$\bold{v}_k$ with one coordinate direction $q_k$.
We then multiply it in all positions $s$ by a factor $\exp(\mu_k(s))$
such that its component along that direction is real:
\begin{equation}
b_{k,2k-1}(s) = 0, \qquad k = 1, \ldots , N.
\end{equation}
 
Usually the $k^{th}$~eigenvector has a largest component
whose magnitude is its largest extent along the $k^{th}$~axis of
physical space.
This a natural association of eigenvectors to directions,
but it may also be made simply by arbitrary numbering.
The re-normalisation does not affect the invariants.
It merely causes the components of $\bold{u}$ to proceed around the ring with
a phase advance which reaches $2 \pi Q_k$ after one turn.
In the case of an uncoupled machine these phases become the usual
Courant-Snyder phase functions.
 
The representation of one eigenmode along~$s$ takes the final form
\begin{equation}
\bold{z}_k = \sqrt{2 I_k} \Re \left(
        \bold{v}_k^*(s) \exp (i \psi_k + \mu_k(s))
\right),
\end{equation}
or expressed in real quantities
\begin{equation}
\bold{z}_k = \sqrt{2 I_k} \left(
        \bold{a}_k(s) \cos(\psi_k + \mu_k(s)) +
        \bold{b}_k(s) \sin(\psi_k + \mu_k(s))
\right).
\end{equation}
The ellipse generated by one mode varies with the position~$s$,
and the eigenvectors $\bold{v}_k$ are re-normalised at each~$s$ such that
\begin{equation}
b_{k,2k-1}(s) = 0, \qquad k = 1, \ldots , N.
\end{equation}
 
 
\section{TRANSPORT Sigma Matrix}
\index{Sigma matrix}
\index{covariance matrix of beam}
\index{beam!covariance matrix}
It is easily seen that the particle motion for mode $k$ generates
the second moments
\begin{equation}
\sigma_{m,n} = \langle z_m z_n\rangle =
a_{k,m} a_{k,n} + b_{k,m} b_{k,n} = v_{k,m} b^*_{k,n}.
\end{equation}
They can be assembled to form a matrix
\begin{equation}
\Sigma_k = \bold{v}_k \bold{v}_k^{*T}.
\end{equation}
The $\Sigma$~matrix has been introduced in TRANSPORT~\cite{SLAC75}
for the first time.
For a given value of~$I_k$ the largest extents
of the eigenmode in each direction of phase space are
\begin{equation}\begin{array}{lllllll}
\sigma_{2m-1} &=& \max q_{k,m} &=& \max z_{k,2m-1} &=&
\sqrt{2 I_k (a_{k,2m-1}^2 + b_{k,2m-1}^2)}, \\
\sigma_{2m} &=& \max p_{k,m} &=& \max z_{k,2m} &=&
\sqrt{2 I_k (a_{k,2m}^2 + b_{k,2m}^2)}.
\end{array}\end{equation}
The correlations between phase space coordinates are
\begin{equation}
r_{m,n} = \sqrt{\frac{4 I_m I_n}{\max z_m \max z_n}}
(a_{k,m} a_{k,n} + b_{k,m} b_{k,n}).
\end{equation}
If one assumes a Gaussian distribution of $\sqrt{2 I_k}$,
each mode represents a Gaussian distribution in a plane
with the density function
\begin{equation}
N(\bold{u}) \propto \exp (-I_k) = \exp \left(
        - \frac{1}{2} \bold{u}^T \bold{u}*
\right) = \exp \left(
        - \frac{1}{2} \bold{z}^T \Sigma_k^{-1} \bold{z}
\right).
\end{equation}
The three modes can be superposed and give a $2N$-dimensional Gaussian
distribution with the $\Sigma$~matrix
\begin{equation}
\Sigma = \sum_{k=1}^3 E_k \Sigma_k.
\end{equation}
where $E_k$ is the emittance belonging to the mode $k$,
and the distribution is
\begin{equation}
N(\bold{u}) \propto \exp \left(
        - \frac{1}{2} \bold{z}^T \Sigma^{-1} \bold{z}
\right).
\end{equation}
 
 
\section{Coupling Angles}
\index{coupling angles}
Since the canonical variables do not all have the same units,
it is not possible to talk about an orientation of the ellipsoid
in $2N$-dimensional phase space.
However, we can project the ellipsoid onto the planes $(x,y)$,
$(x,t)$, and $(y,t)$.
The momenta are conserved by this operation.
They can be extracted as the relevant components from the phase space
vector and the $\Sigma$~matrix.
The projection is then given by the equation
\begin{equation}
\frac{1}{2 (\sigma_{x,x} \sigma_{y,y} - \sigma{x,y}^2)} \left(
        \sigma_{y,y} x^2 - 2 \sigma_{1,2} x y + \sigma_{x,x} y^2
\right) = \hbox{constant,}
\label{eqell}
\end{equation}
where we have denoted the selected phase planes by $x$ and $y$.
For each plane one has the situation of Fig.~\ref{ellipse}.
\begin{figure}[ht]
\centering
\setlength{\unitlength}{0.9pt}
\begin{picture}(500,400)(-250,-200)
\thinlines
% frame and axes
\put(-200,-160){\framebox(400,320){}}
\put(-250,0){\vector(1,0){500}}
\put(230,-20){\makebox(20,20){$z_1$}}
\put(0,-200){\vector(0,1){400}}
\put(0,180){\makebox(20,20){$z_2$}}
\put(0,0){\circle*{5}}
% label the maxima on the axes
\put(200,0){\circle*{5}}
\put(200,-20){\makebox(20,20){$\sigma_1$}}
\put(0,160){\circle*{5}}
\put(0,160){\makebox(20,20){$\sigma_2$}}
% mark the correlation (vertical)
\put(200,86){\circle*{5}}
\put(200,86){\line(1,0){30}}
\put(220,0){\vector(0,1){86}}
\put(220,33){\makebox(40,20){$r_{12}\sigma_2$}}
% mark the correlation (horizontal)
\put(108,160){\circle*{5}}
\put(108,160){\line(0,1){30}}
\put(0,180){\vector(1,0){108}}
\put(34,180){\makebox(40,20){$r_{12}\sigma_1$}}
% tangent points on opposite sides
\put(-200,-86){\circle*{5}}
\put(-108,-160){\circle*{5}}
% ellipse axes
\put(0,0){\line(3,2){220}}
\put(0,0){\line(-2,3){100}}
\put(0,0){\line(-3,-2){220}}
\put(0,0){\line(2,-3){100}}
\put( 189, 126){\circle*{5}}
\put( 100,  63){\makebox(0,0)[tl]{a}}
\put(-189,-126){\circle*{5}}
\put(- 66,  99){\circle*{5}}
\put(- 35,  50){\makebox(0,0)[tr]{b}}
\put(  66,- 99){\circle*{5}}
% angle of rotation
\bezier{50}(50,0)(50,14)(42,28)
\put(46,22){\vector(-2,3){4}}
\put(26,7){\makebox{$\phi$}}
% ellipse
\thicklines
\bezier{100}( 200,  86)( 200, 110)( 189, 126)
\bezier{100}(-200,- 86)(-200,-110)(-189,-126)
\bezier{100}( 189, 126)( 166, 160)( 108, 160)
\bezier{100}(-189,-126)(-166,-160)(-108,-160)
\bezier{200}( 108, 160)(  26, 160)(- 66,  99)
\bezier{200}(-108,-160)(- 26,-160)(  66,- 99)
\bezier{300}(- 66,  99)(-200,   9)(-200,- 86)
\bezier{300}(  66,- 99)( 200,-  9)( 200,  86)
\end{picture}
\caption{Projection of the phase space ellipsoid onto one plane}
\label{ellipse}
\end{figure}
 
Analysing equation~\ref{eqell} gives the maximum extents
\begin{equation}
\sigma_x = \sqrt{\sigma_{x,x}}, \qquad
\sigma_y = \sqrt{\sigma_{y,y}}.
\end{equation}
The correlation is as before
\begin{equation}
r_{x,y} = \sigma_{x,y} / (\sigma_{x} \sigma_{y}).
\end{equation}
The points where the ellipse touches the circumscribed rectangle are
indicated with their coordinates in figure~\ref{ellipse}.
If and only if the units for both axes are the same,
one may computed the principal axes $a,b$  and the angle $\phi$
between the principal axes and the coordinate axes:
\begin{equation}\begin{array}{ccl}
a^2 &=& \frac{1}{2} \left( \sigma_{x,x} + \sigma_{y,y} +
        \sqrt{(\sigma_{x,x} - \sigma_{y,y})^2 + 4 \sigma_{x,y}^2}\right), \\
b^2 &=& \frac{1}{2} \left( \sigma_{x,x} + \sigma_{y,y} -
        \sqrt{(\sigma_{x,x} - \sigma_{y,y})^2 + 4 \sigma_{x,y}^2}\right), \\
\tan 2\phi &=& (2 \sigma_{x,y}) (\sigma_{x,x} - \sigma_{y,y}).
\end{array}\end{equation}
Note that for $\sigma_{x,x} = \sigma_{y,y}$ the angle $\phi$ is not
defined.
 
 
\section{Implementation}
\index{coupled motion}
 
\subsection{EMIT Command}
\index{EMIT}
All commands described in the following sections {\em must} be
preceded by the {\tt EMIT} command.
This fills in the RF~frequencies using the revolution frequency and
the harmonic number.
Note that the RF~lag must be set close to the stable phase,
i.~e. above the transition energy it should be close to~$0.5$.
 
\subsection{Eigenvectors}
\index{EIGEN}
\index{eigenvectors}
MAD lists them in selected position in the command EIGEN.
They can be used to find the invariants.
 
\subsection{Linear Invariants}
\index{invariants!linear}
\index{linear invariants}
The invariants can be built easily as shown above from the
eigenvectors:
\begin{equation}
I_k = \frac{1}{2} \left(
        (\bold{b}_k^T \bold{S} \bold{z})^2 +
        (\bold{a}_k^T \bold{S} \bold{z})^2
\right) = \frac{1}{2}\left(
        \left(
                \sum_{i=1}^N (a_{k,2i-1} \bold{z}_{2i} -
                a_{k,2i} \bold{z}_{2i-1})
        \right)^2 + \left(
                \sum_{i=1}^N (b_{k,2i-1} \bold{z}_{2i} -
                b_{k,2i} \bold{z}_{2i-1})
        \right)^2
\right).
\end{equation} 
 
\subsection{Phase advances}
\index{coupled phases}
\index{phases!coupled}
MAD prints the phase advances in the commands TWISS3 and EIGEN.
They are defined as described above;
i.~e. as the phase factors which re-normalise the principal components
of the eigenvectors and makes them real.
 
\subsection{Envelopes}
\label{EMENGO}
\index{ENVELOPE}
\index{covariance matrix of beam}
\index{beam!covariance matrix}
The ENVELOPE command (must be preceded by EMIT !) computes the
TRANSPORT $\Sigma$~matrix from the eigenvectors.
It is computed and listed in selected positions.
 
\subsection{Mais-Ripken Lattice Functions}
\label{EMTWGO}
\index{Mais}\index{Ripken}
\index{lattice functions!Mais-Ripken}
This method is due to H.~Mais and G.~Ripken~\cite{MAI82,RIP70}.
It is used in the {\tt TWISS3} command (which must be preceded by EMIT !),
which computes the projections of the ellipses of motion onto the
phase planes.
The extents are expressed in the form similar to the Courant-Snyder
parameters.
The command lists these parameters along with the phases in selected
positions.
The projection of the Courant-Snyder lattice functions on the three
planes $(x,p_x)$, $(y,p_y)$, and $(t,p_t)$ are:
\begin{equation}\begin{array}{lclcll}
\beta_{k,j}  &=&a_{k,2j-1}a_{k,2j-1}&+&b_{k,2j-1}b_{k,2j-1},\\
\gamma_{k,j} &=&a_{k,2j  }a_{k,2j  }&+&b_{k,2j  }b_{k,2j  },\\
\alpha_{k,j} &=&a_{k,2j-1}a_{k,2j  }&+&b_{k,2j  }b_{k,2j-1}
\end{array}\end{equation}
The index~$k$ refers to the eigenmode, and the index~$j$ to the plane.
MAD prints the functions~$\beta_{k,j}$, $\alpha_{k,j}$, and $\gamma_{k,j}$.
It also prints the phase advances $\mu_{k,j}$ of the projections onto
the three planes.
Note that we have the equations
\begin{equation}
\beta_{k,j} \gamma_{k,j} - \alpha_{k,j}^2 = 
\left( \beta_{k,j} \frac{d \mu_{k,j}}{ds} \right)^2, \quad
\hbox{\bf\rm but} \quad \beta_{k,j} \frac{d \mu_{k,j}}{ds} \neq 1.
\end{equation}
Note that it is easy to convert the eigenvectors to the Mais-Ripken
functions, but that this does not reduce the redundancy in
information.
A conversion in the opposite direction has not been found,
it may well be impossible.
 
 
\section{Transformations between Representations of Beam}
\index{transformation of beam}
\index{beam!transformation}
 
\subsection{Eigenvectors to Internal Sigma Matrix}
\index{eigenvectors}
\index{covariance matrix}
\index{Sigma matrix}
Given the eigenvectors and the emittances for the three eigen-modes the
beam ellipsoid $\Sigma$~\cite{SLAC75} can be computed as
\begin{equation}
\Sigma = \left(\myarray{
\Sigma_{11} & \cdots & \Sigma_{14} \\
\vdots      &        & \vdots \\
\Sigma_{41} & \cdots & \Sigma_{44}
}\right) =
\sum_{k=1}^3 E_k \Re (V_k^T V_k^*).
\end{equation}
Assuming a Gaussian distribution, the particle distribution is then
\begin{equation}
N(Z) \propto exp(\frac{1}{2} Z^T \Sigma^{-1} Z).
\end{equation}
 
\subsection{Internal Sigma Matrix to TRANSPORT Notation}
\index{covariance matrix}
\index{Sigma matrix}
Given the $\Sigma$ matrix of the previous section the standard
deviation of~$z_k$ is~\cite{SLAC75}:
\begin{equation}
\sigma_k = \sqrt{\Sigma_{kk}},
\end{equation}
and the correlations between~$z_k$ and~$z_m$ as
\begin{equation}
r_{km} = \frac{\Sigma_{km}}{\sigma_k \sigma_m}.
\end{equation}
MAD prints beam envelopes in this form.
 
\subsection{TRANSPORT Notation to Internal Sigma Matrix}
\index{covariance matrix}
\index{Sigma matrix}
The formulas of the previous section can be inverted as:
\begin{equation}
\Sigma_{kk} = \sigma_k^2, \qquad k = 1 \ldots 4,
\end{equation}
and the off-diagonal elements are
\begin{equation}
\Sigma_{km} = r_{km} \sigma_k \sigma_m, k, m = 1 \ldots 4, k \neq m.
\end{equation}
MAD uses this formula to find the internal $\Sigma$ matrix from a
\ttindex{SIGMA0} command.
 
 
%==============================================================================
 
\chapter{Survey}
\label{survey}
\index{survey}
\index{geometry}
 
\section{Global Reference System}
\label{layout}
\index{global reference}
\index{global coordinates}
\index{coordinates}
\index{local origin}
The reference orbit of the accelerator is uniquely defined by the
sequence of physical elements.
The local reference system $(x, y, s)$ may thus be referred
to a global Cartesian coordinate system $(X, Y, Z)$
(see Figure~\ref{global}).
The positions between beam elements are numbered $0,\ldots,i,\ldots,n$.
The local reference system $(x_{i}, y_{i}, z_{i})$
at position $i$,
i.e. the displacement and direction of the reference orbit
with respect to the system $(X, Y, Z)$ are characterised by
three displacements $(X_{i}, Y_{i}, Z_{i})$
and three angles $(\theta_{i}, \phi_{i}, \psi_{i})$
The above quantities are defined more precisely as follows:
 
\begin{figure}[ht]
\centering
\setlength{\unitlength}{1pt}
\begin{picture}(400,270)
% global axes
\thicklines
\put(20,150){\vector(2,-1){280}}
\put(290,35){\makebox(0,0){$Z$}}
\put(20,100){\vector(3,1){360}}
\put(370,200){\makebox(0,0){$X$}}
\put(80,0){\vector(0,1){270}}
\put(70,240){\makebox(0,0){$Y$}}
%local axes
\put(133.3,0){\vector(1,3){90}}
\put(213.3,260){\makebox(0,0){$x$}}
\put(300,150){\vector(-2,1){180}}
\put(150,215){\makebox(0,0){$y$}}
\put(0,100){\vector(2,1){270}}
\put(260,240){\makebox(0,0){$s$}}
% projection of s onto ZX
\thinlines
\put(80,120){\circle*{4}}
\put(200,200){\circle*{4}}
\put(0,110){\line(1,0){290}}
\put(300,110){\makebox(0,0)[l]{\shortstack{projection of $s$ \\
onto $ZX$-plane}}}
\put(20,110){\circle*{4}}
\put(50,110){\circle*{4}}
\put(100,110){\circle*{4}}
% displacement of local system
\put(140,140){\line(2,-1){60}}
\put(160,128){\makebox(0,0)[tr]{$Z$}}
\put(140,90){\line(3,1){60}}
\put(190,100){\makebox(0,0)[tl]{$X$}}
\put(200,110){\line(0,1){90}}
\put(205,140){\makebox(0,0)[l]{$Y$}}
\put(140,90){\circle*{4}}
\put(140,140){\circle*{4}}
\put(200,110){\circle*{4}}
% intersection of xy and ZX
\put(130,0){\line(1,1){230}}
\put(135,5){\circle*{4}}
\put(193.3,63.3){\circle*{4}}
\put(240,110){\circle*{4}}
\put(286.7,156.7){\circle*{4}}
\put(335,205){\circle*{4}}
\thicklines
\put(180,20){\vector(-1,1){12}}
\put(180,20){\makebox(0,0)[tl]{\shortstack{intersection of \\
$xy$ and $ZX$ planes}}}
% reference orbit
\bezier{80}(140,150)(170,185)(200,200)
\bezier{80}(200,200)(230,215)(260,220)
\put(260,220){\makebox(0,0)[l]{\shortstack{reference \\orbit}}}
% roll angle
\bezier{30}(160,30)(160,40)(150,50)
\put(152,48){\vector(-1,1){2}}
\put(150,30){\makebox(0,0){$\psi$}}
\put(140,30){\makebox(0,0)[br]{roll angle}}
% pitch angle
\bezier{20}(60,110)(60,120)(55,125)
\put(57,123){\vector(-1,2){2}}
\put(50,118){\makebox(0,0){$\phi$}}
\put(40,125){\makebox(0,0)[br]{pitch angle}}
% azimuth
\bezier{20}(130,95)(140,100)(140,110)
\put(140,105){\vector(0,1){5}}
\put(130,105){\makebox(0,0){$\theta$}}
\put(115,95){\makebox(0,0)[t]{azimuth}}
\end{picture}
\caption{Global Reference System}
\label{global}
\end{figure}
 
\begin{mylist}
\item[$X$]
\index{X}
Displacement of the local origin in $X$-direction.
\item[$Y$]
\index{Y}
Displacement of the local origin in $Y$-direction.
\item[$Z$]
\index{Z}
Displacement of the local origin in $Z$-direction.
\keyitem{$\theta$}
\index{rotation angle}
\index{angle of rotation}
\index{azimuth}
Angle of rotation (azimuth) about the global $Y$-axis,
between the global $Z$-axis and the projection
of the reference orbit onto the $(Z, X)$-plane.
A positive angle $\theta$ forms a right-hand screw with the
$Y$-axis.
\keyitem{$\phi$}
\index{elevation angle}
Elevation angle,
i.e. the angle between the reference orbit and its projection
onto the $(Z, X)$-plane.
A positive angle $\phi$ correspond to increasing $Y$.
If only horizontal bends are present,
the reference orbit remains in the $(Z, X)$-plane.
In this case $\phi$ is always zero.
\keyitem{$\psi$}
\index{roll angle}
\index{tilt angle}
Roll angle about the local $s$-axis,
i.e. the angle between the intersection
of the $(x, y)$ and $(Z, X)$-planes and the local $x$-axis.
A positive angle $\psi$ forms a right-hand screw with the $s$-axis.
\end{mylist}
The angles $(\theta, \phi, \psi)$ are {\it not} the Euler angles.
The reference orbit starts at the origin and points by default
in the direction of the positive $Z$-axis.
\index{local axes}
\index{global axes}
The initial local axes $(x, y, s)$ coincide with the global axes
$(X, Y, Z)$ in this order.
The six quantities
$(X_{0}, Y_{0}, Z_{0}, \theta_{0}, \phi_{0}, \psi_{0}),$
thus all have zero initial values by default.
The program user may however specify different initial conditions.
 
Internally the displacement is described by a vector $V$,
\index{displacement}
\index{orientation}
and the orientation by a unitary matrix $W$.
The {\em column vectors} of $W$ are the unit vectors spanning
the local coordinate axes in the order $(x, y, s)$.
$V$ and $W$ have the values
\begin{equation}
V=\left(\myarray{
   X \\
   Y \\
   Z
}\right),
\qquad
W=\Theta\Phi\Psi,
\end{equation}
where
\begin{equation}
\Theta=\left(\myarray{
   \cos\theta &  0 &  \sin\theta \\
     0         &  1 &   0 \\
   -\sin\theta &  0 &  \cos\theta
}\right), \quad
\Phi=\left(\myarray{
    1 &  0        &  0 \\
    0 &  \cos\phi &  \sin\phi \\
    0 & -\sin\phi &  \cos\phi
}\right), \quad
\Psi=\left(\myarray{
    \cos\psi & -\sin\psi &  0 \\
    \sin\psi &  \cos\psi &  0 \\
    0        &  0        &  1
}\right).
\end{equation}
The reference orbit should be closed and it should not be twisted.
This means that the displacement of the local reference system
must be periodic with the revolution frequency of the accelerator,
while the position angles must be periodic (modulo $2\pi$)
with the revolution frequency.
If $\psi$ is not periodic (modulo $2\pi$),
coupling effects are introduced.
When advancing through a beam element,
MAD computes $V_{i}$ and $W_{i}$
by the recurrence relations
\begin{equation}
V_{i}=W_{i-1}R_{i}+V_{i-1}, \qquad
W_{i}=W_{i-1}S_{i}.
\end{equation}
The vector $R_{i}$ is the displacement and the matrix $S_{i}$ is
the rotation of the local reference system at the exit of the
element~$i$ with respect to the entrance of the same element. 
The values of $R$ and $S$ are listed below for each physical element
type. 
 
\section{Single Elements}
 
\subsection{Markers}
\index{MARKER}
Marker elements do not affect the reference orbit.
They are ignored for geometry calculations.
 
\subsection{Straight Elements}
The reference system for all straight elements is shown in
Figure~\ref{F-DRF}.
It is valid for:
 
\indent\begin{tabular}{llll}
$\bullet$ \tt DRIFT \index{drift} &
$\bullet$ \tt RCOLLIMATOR \index{collimator} &
$\bullet$ \tt ECOLLIMATOR &
$\bullet$ \tt INSTRUMENT \index{instrument} \\
$\bullet$ \tt MONITOR \index{monitor} &
$\bullet$ \tt HMONITOR \index{drift} &
$\bullet$ \tt VMONITOR \index{drift} &
$\bullet$ \tt QUADRUPOLE \index{quadrupole} \\
$\bullet$ \tt SEXTUPOLE \index{sextupole} &
$\bullet$ \tt OCTUPOLE \index{octupole} &
$\bullet$ \tt SOLENOID \index{solenoid} &
$\bullet$ \tt RFCAVITY \index{RF cavity} \index{cavity} \\
$\bullet$ \tt ELSEPARATOR \index{separator}
\index{electrostatic separator} &
$\bullet$ \tt KICKER \index{corrector} &
$\bullet$ \tt HKICKER &
$\bullet$ \tt VKICKER \\
$\bullet$ \tt MONITOR \index{monitor} &
$\bullet$ \tt HMONITOR &
$\bullet$ \tt VMONITOR \\
\end{tabular}
 
\begin{figure}[ht]
\centering
\setlength{\unitlength}{1pt}
\begin{picture}(400,100)
\thinlines
% axes
\put(150,50){\circle{8}}\put(150,50){\circle*{2}}
\put(140,40){\makebox(0,0){$y_1$}}
\put(250,50){\circle{8}}\put(250,50){\circle*{2}}
\put(260,40){\makebox(0,0){$y_2$}}
\put(100,50){\line(1,0){46}}
\put(154,50){\line(1,0){92}}
\put(254,50){\vector(1,0){46}}
\put(290,40){\makebox(0,0){$s$}}
\put(150,0){\line(0,1){46}}
\put(150,54){\vector(0,1){46}}
\put(140,90){\makebox(0,0){$x_1$}}
\put(250,0){\line(0,1){46}}
\put(250,54){\vector(0,1){46}}
\put(260,90){\makebox(0,0){$x_2$}}
% magnet outline
\thicklines
\put(150,54){\line(0,1){26}}
\put(150,46){\line(0,-1){26}}
\put(250,54){\line(0,1){26}}
\put(250,46){\line(0,-1){26}}
\put(150,20){\line(1,0){100}}
\put(150,80){\line(1,0){100}}
\put(200,2){\vector(1,0){50}}
\put(200,2){\vector(-1,0){50}}
\put(200,10){\makebox(0,0){L}}
\end{picture}
\caption{Reference System for Straight Beam Elements}
\label{F-DRF}
\end{figure}
 
\noindent The corresponding $R$ and $S$ are
\begin{equation}
   R=\left(\myarray{
      0 \\
      0 \\
      L
   }\right),
   \qquad
   S=\left(\myarray{
      1 & 0 & 0 \\
      0 & 1 & 0 \\
      0 & 0 & 1 \\
   }\right).
\end{equation}
A rotation of the {\em element} about the $s$-axis has an effect
on $R$ and $S$ only for dipoles,
since for all other elements the rotations of the reference system
before and after the element cancel.
 
\subsection{Dipoles}
\index{bending magnet}
\index{dipole}
Bending magnets affect the reference orbit due to their curvature.
For both rectangular and sector bending magnets
\begin{equation}
   R=\left(\myarray{
      \rho(\cos\alpha-1) \\
      0 \\
      \rho\sin\alpha
   }\right),
   \qquad
   S=\left(\myarray{
       \cos\alpha & 0 & -\sin\alpha \\
       0          & 1 &  0 \\
       \sin\alpha & 0 &  \cos\alpha
   }\right),
\end{equation}
where $\alpha$ is the bend angle.
A positive bend angle represents a bend to the right,
i.e. towards negative $x$ values.
For sector bending magnets,
the bend radius is given by $\rho=L/\alpha$,
and for rectangular bending magnets it has the value
$\rho=L/(2\sin(\alpha/2))$.
 
The reference system for type \ttindex{SBEND} is shown in
Figure~\ref{F-SBND},
for type \ttindex{RBEND} it is shown in Figure~\ref{F-RBND}.
 
\begin{figure}[ht]
\centering
\setlength{\unitlength}{1pt}
\begin{picture}(400,215)
% axes
\thinlines
\put(150,150){\circle{8}}\put(150,150){\circle*{2}}
\put(160,140){\makebox(0,0){$y_1$}}
\put(250,150){\circle{8}}\put(250,150){\circle*{2}}
\put(240,140){\makebox(0,0){$y_2$}}
\put(74,124.7){\vector(3,1){72}}
\put(84,135){\makebox(0,0){$s_1$}}
\put(254,148.7){\vector(3,-1){72}}
\put(316,135){\makebox(0,0){$s_2$}}
\put(200,0){\vector(-1,3){48.7}}
\put(165,75){\makebox(0,0){$\rho$}}
\put(148.7,154){\vector(-1,3){18}}
\put(118,206){\makebox(0,0){$x_1$}}
\put(200,0){\vector(1,3){48.7}}
\put(235,75){\makebox(0,0){$\rho$}}
\put(251.3,154){\vector(1,3){18}}
\put(282,206){\makebox(0,0){$x_2$}}
\bezier{20}(190.5,28.5)(200,31.7)(209.5,28.5)
\put(200,20){\makebox(0,0){$\alpha$}}
\put(154,150){\line(1,0){92}}
\put(200,150){\circle*{4}}
\put(200,150){\vector(0,1){60}}
\put(210,200){\makebox(0,0){$x$}}
\put(150,154){\line(0,1){44}}
\put(150,146){\line(0,-1){46}}
\put(151,154){\line(1,4){11}}
\put(250,154){\line(0,1){44}}
\put(250,146){\line(0,-1){46}}
\put(249,154){\line(-1,4){11}}
% magnet outline
\thicklines
\put(200,102){\vector(-1,0){50}}
\put(200,102){\vector(1,0){50}}
\put(200,110){\makebox(0,0){L}}
\put(151,154){\line(1,4){6}}
\put(149,146){\line(-1,-4){6}}
\put(249,154){\line(-1,4){6}}
\put(251,146){\line(1,-4){6}}
\put(157,178){\line(1,0){86}}
\put(143,122){\line(1,0){114}}
\bezier{10}(150,195)(155.5,195)(160.9,193.7)
\put(155.5,195){\vector(3,-1){5.4}}
\put(150,205){\makebox(0,0)[l]{$e_1$}}
\bezier{10}(250,195)(244.5,195)(239.1,193.7)
\put(244.5,195){\vector(-3,-1){5.4}}
\put(250,205){\makebox(0,0)[r]{$e_2$}}
\end{picture}
\caption[Reference System for a Rectangular Bending Magnet]%
{Reference System for a Rectangular Bending Magnet;
the signs of pole-face rotations are positive as shown.}
\label{F-RBND}
\end{figure}
 
\begin{figure}[ht]
\centering
\setlength{\unitlength}{1pt}
\begin{picture}(400,215)
% axes
\thinlines
\put(150,150){\circle{8}}\put(150,150){\circle*{2}}
\put(160,140){\makebox(0,0){$y_1$}}
\put(250,150){\circle{8}}\put(250,150){\circle*{2}}
\put(240,140){\makebox(0,0){$y_2$}}
\put(74,124.7){\vector(3,1){72}}
\put(84,135){\makebox(0,0){$s_1$}}
\put(254,148.7){\vector(3,-1){72}}
\put(316,135){\makebox(0,0){$s_2$}}
\put(200,0){\vector(-1,3){48.7}}
\put(165,75){\makebox(0,0){$\rho$}}
\put(148.7,154){\vector(-1,3){18}}
\put(118,206){\makebox(0,0){$x_1$}}
\put(200,0){\vector(1,3){48.7}}
\put(235,75){\makebox(0,0){$\rho$}}
\put(251.3,154){\vector(1,3){18}}
\put(282,206){\makebox(0,0){$x_2$}}
\bezier{20}(190.5,28.5)(200,31.7)(209.5,28.5)
\put(200,20){\makebox(0,0){$\alpha$}}
\put(200,158.8){\circle*{4}}
\put(200,158.8){\vector(0,1){50}}
\put(210,200){\makebox(0,0){$r$}}
\put(151,154){\line(1,4){10}}
\put(249,154){\line(-1,4){10}}
% magnet outline
\thicklines
\bezier{100}(154,151.3)(200,166.7)(246,151.3)
\put(162,154){\vector(-3,-1){8}}
\put(238,154){\vector(3,-1){8}}
\put(210,168){\makebox(0,0){L}}
\put(151,154){\line(1,4){6}}
\put(149,146){\line(-1,-4){6}}
\put(249,154){\line(-1,4){6}}
\put(251,146){\line(1,-4){6}}
\bezier{90}(157,178)(200,188.4)(243,178)
\bezier{110}(143,122)(200,148.6)(257,122)
\bezier{20}(137.4,187.9)(149.1,191.5)(159.7,188.8)
\put(153.7,190.8){\vector(3,-1){6}}
\put(150,180){\makebox(0,0){$e_1$}}
\bezier{20}(262.6,187.9)(250.9,191.5)(240.3,188.8)
\put(246.3,190.8){\vector(-3,-1){6}}
\put(250,180){\makebox(0,0){$e_2$}}
\end{picture}
\caption[Reference System for a Sector Bending Magnet]%
{Reference System for a Sector Bending Magnet;
the signs of pole-face rotations are positive as shown.}
\label{F-SBND}
\end{figure}
 
If the magnet is rotated about the $s$-axis by an angle $\psi$,
$R$ and $S$ are transformed by
\begin{equation}
   \overline{R}=TR,
   \qquad
   \overline{S}=TST^{-1}
\end{equation}
where $T$ is the rotation matrix
\begin{equation}
   T=\left(\myarray{
       \cos\psi & -\sin\psi &  0 \\
       \sin\psi &  \cos\psi &  0 \\
       0        &  0        &  1 \\
   }\right).
\end{equation}
The special value $\psi=\pi/2$ represents
a bend down.
 
\subsection{Rotation of Reference System about $s$-Axis}
\index{rotation}
\index{reference system}
The reference system for the \ttindex{SROT} element which  rotates the
local reference system about the longitudinal axis is shown in
Figure~\ref{F-SROT}.
{\tt SROT} has no effect on the beam,
but it causes the beam to be referred to the new coordinate system
\begin{equation}
x_2=x_1 \cos\psi + y_1 \sin\psi,
\qquad
y_2=x_1 \sin\psi + y_1 \cos\psi.
\end{equation}
A positive angle means that the new reference system is rotated clockwise
about the $s$-axis with respect to the old system.
The {\em reference system} is changed using
\begin{equation}
R = \left( \myarray{ 0 \\ 0 \\ 0 } \right), \qquad
S = \left( \myarray{
    \cos\psi & -\sin\psi &  0 \\
    \sin\psi &  \cos\psi &  0 \\
    0        &  0        &  1 \\
} \right).       
\end{equation}
 
\subsection{Rotation of Reference System about $y$-Axis}
\index{rotation}
\index{reference system}
The reference system for a rotation by an angle~$\theta$
about the vertical axis (\ttindex{YROT})
the reference system is shown in Figure~\ref{F-YROT}.
{\tt YROT} has no effect on the beam,
but it causes the beam to be referred to the new coordinate system
\begin{equation}
x_2=x_1 \cos\theta - s_1 \sin\theta,
\qquad
s_2=x_1 \sin\theta + s_1 \cos\theta.
\end{equation}
A positive angle rotates the reference system clockwise about the
local $y$-axis with respect to the old system:
\begin{equation}
R = \left( \myarray{ 0 \\ 0 \\ 0 } \right), \qquad
S = \left(\myarray{
    \cos\theta &  0 & -\sin\theta \\
    0          &  1 &  0 \\
    \sin\theta &  0 &  \cos\theta
} \right).
\end{equation}
 
\begin{figure}[ht]
\centering
\setlength{\unitlength}{1pt}
\begin{picture}(400,200)
\thinlines
\put(200,100){\circle{8}}\put(200,100){\circle*{2}}
\put(190,90){\makebox(0,0){$s$}}
\put(100,100){\line(1,0){96}}
\put(204,100){\vector(1,0){96}}
\put(290,90){\makebox(0,0){$x_1$}}
\put(200,0){\line(0,1){96}}
\put(200,104){\vector(0,1){96}}
\put(190,210){\makebox(0,0){$y_1$}}
\put(103,75.75){\line(4,1){93}}
\put(204,101){\vector(4,1){93}}
\put(287,134){\makebox(0,0){$x_2$}}
\put(224.25,3){\line(-1,4){23.25}}
\put(199,104){\vector(-1,4){23.25}}
\put(166,187){\makebox(0,0){$y_2$}}
\bezier{20}(260,100)(260,107.5)(258,114.5)
\put(260,106.5){\vector(-1,4){2}}
\put(250,106.25){\makebox(0,0){$\psi$}}
\put(220,150){\circle{8}}\put(220,150){\circle*{2}}
\put(220,140){\makebox(0,0){beam}}
\end{picture}
\caption{Reference System for a Rotation Around the s-Axis}
\label{F-SROT}
\end{figure}
 
\begin{figure}[ht]
\centering
\setlength{\unitlength}{1pt}
\begin{picture}(400,200)
\thinlines
\put(200,100){\circle{8}}\put(200,100){\circle*{2}}
\put(190,90){\makebox(0,0){$y$}}
\put(100,100){\line(1,0){96}}
\put(204,100){\vector(1,0){96}}
\put(290,110){\makebox(0,0){$s_1$}}
\put(200,0){\line(0,1){96}}
\put(200,104){\vector(0,1){96}}
\put(190,190){\makebox(0,0){$x_1$}}
\put(103,124.25){\line(4,-1){93}}
\put(204,99){\vector(4,-1){93}}
\put(287,66){\makebox(0,0){$s_2$}}
\put(175.75,3){\line(1,4){23.25}}
\put(201,104){\vector(1,4){23.25}}
\put(234,187){\makebox(0,0){$x_2$}}
\bezier{20}(260,100)(260,92.5)(258,85.5)
\put(260,93.5){\vector(-1,-4){2}}
\put(250,93.75){\makebox(0,0){$\theta$}}
\thicklines
\put(100,130){\vector(1,0){200}}
\put(290,140){\makebox(0,0){beam}}
\end{picture}
\caption{Reference System for a Rotation Around the y-Axis}
\label{F-YROT}
\end{figure}
 
\section{Sequences of Elements}
\label{surseq}
\index{composition!survey}
\index{survey!composition}
The displacement and rotation of the reference system due to an
element sequence can be accumulated by the recurrence relations given
above:
\begin{equation}
V_{i}=W_{i-1}R_{i}+V_{i-1}, \qquad
W_{i}=W_{i-1}S_{i}, \qquad i = 1 \ldots n.
\end{equation}
Accumulating these quantities in {\em beam order} one finds for the
sequence the values
\begin{equation}
R = V_{n}, \qquad S = W_{n}.
\end{equation}
 
 
% ====================================================================
 
\chapter{Closed Orbit and Dispersion Correction}
This chapter describes the algorithms used for orbit and dispersion
correction.
The data structures used are documented in the MAD Programmer's Manual.
 
 
\section{Effect of a Kick on the Closed Orbit}
\label{kick}
***** section to be filled in *****
 
 
\section{Orbit Correction Only by MICADO Algorithm}
***** section to be filled in *****
 
 
\section{Influence Matrix for Orbit for a Plane}
***** section to be filled in *****
 
 
\section{Orbit and Dispersion Correction by MICADO Algorithm}
***** section to be filled in *****
 
 
\section{Influence Matrix for Orbit and Dispersion for a Plane}
***** section to be filled in *****
 
 
% ====================================================================
 
\chapter{Synchrotron Radiation and Equilibrium Emittances}
\label{synchro}
\index{radiation}
\index{synchrotron radiation}
 
Synchrotron radiation effects are optional in MAD:
\begin{itemize}
\item
  If {\tt RADIATE} on the {\tt BEAM} command is false,
  synchrotron radiation is ignored in {\em all commands}.
\item
  \index{synchrotron radiation!systematic effects}
  If {\tt RADIATE} is true,
  all commands apply synchrotron radiation as a {\em systematic effects}.
  The energy loss due to radiation changes the closed orbit,
  and the instantaneous momentum on the {\em closed orbit} affects the
  optics.
  In all magnetic elements MAD tracks the closed orbit in a three-step
  process:
  \begin{enumerate}
  \item
    Compute the local magnetic field on the closed orbit where it enters
    the magnetic element.
    Assume that this field acts for half the length of the element on
    the particle, and use it to compute the energy radiated at the
    entrance of the element.
  \item
    Track the particle through the element, leaving its momentum
    unchanged.
  \item
    Compute the local magnetic field on the closed orbit where it leaves
    the magnetic element.
    Assume that this field acts for half the length of the element on
    the particle, and use it to compute the energy radiated at the
    exit of the element.
  \end{enumerate}
\end{itemize}
 
The {\tt TRACK} command has two additional flags,
{\tt DAMP} and {\tt QUANTUM}:
\begin{itemize}
\item
  \index{synchrotron radiation!systematic effects}
  If {\tt DAMP} is false, the value of {\tt QUANTUM} is ignored,
  and all particles loose energy as described above,
  the radiation being determined by the curvature of the
  {\em closed orbit}. 
  Hence tracking uses the optics defined by the instantaneous momentum
  on the closed orbit and the particles see {\em no damping}.
  However, the {\em saw-tooth} like energy variation due to
  synchrotron radiation exists.
\item
  \index{synchrotron radiation!damping}
  If {\tt DAMP} is true and {\tt QUANTUM} is false,
  the local curvature of the actual particle orbit determines the
  systematic component of the radiation loss.
  The orbits are damped but there is no quantum excitation.
  This causes all trajectories to be attracted to the closed orbit,
  and allows to verify the damping times computed in {\tt EMIT}. 
\item
  \index{synchrotron radiation!quantum effects}
  If both {\tt DAMP} and {\tt QUANTUM} are true,
  the local curvature of the actual particle orbit determines the
  radiation damping and quantum excitation.
  This is the most realistic model for the particle behaviour.
\end{itemize}
 
 
\section{Local Curvature for Different Elements}
 
\subsection{Dipole}
\index{dipole body!synchrotron radiation}
For radiation effects MAD ignores the dipole fringing fields.
The reason is that the fringing field acts like a thin lens,
it changes only the orbit angles but not its position.
To determine the local curvature,
the orbit must first be related to the mid-plane of the possibly
rotated dipole:
\begin{equation}
x_r=x\cos\psi+y\sin\psi, \qquad y_r=-x\sin\psi+y\cos\psi,
\end{equation}
where $\psi$ is the roll angle of the dipole.
The total curvature~$h$ is then computed from the curvature in both
planes:
\begin{equation}\eqarray{
h_x&=&K_0 + K_1x_r + K_2(x_r^2 - y_r^2)/2 - K_0K_1y_r^2/2, \\
h_y&=&-K_1y_r - K_2x_ry_r, \\
h  &=&\sqrt{h_x^2+h_y^2}.
}\end{equation}
 
\subsection{Quadrupole, Sextupole, Octupole}
\index{quadrupole!synchrotron radiation}
\index{sextupole!synchrotron radiation}
\index{octupole!synchrotron radiation}
The total local curvature is
\begin{equation}
h=K_n (x^2 + y^2)^{(n/2)}.
\end{equation}
where $n=1$ for a quadrupole, $n=2$ for a sextupole, and $n=3$ for an
octupole.
 
\subsection{Thin Multipole}
\index{thin multipole!synchrotron radiation}
For thin multipole the change in orbit direction is instantaneous,
hence the energy loss would be infinite.
To avoid this problem, MAD defines a fictitious
length~$L_{\hbox{rad}}$ as a parameter for the multipole, which is
used for the radiation loss calculation.
The local curvature is then $h = |P|/L_{\hbox{rad}}$,
where~$P$ is the {\em total kick} of the multipole.
The value of~$P$ can be found in Section~\ref{multipole}.
 
 
\section{Systematic Energy Loss by Synchrotron Radiation}
\index{synchrotron radiation!systematic effects}
The total relative energy lost over half the element length by a
particle is
\begin{equation}
\Delta E/(2 p_s) = \frac{2 r_e}{3} \gamma_s^3 h^2 \half L,
\end{equation}
where the local curvature is evaluated separately for the entrance and
the exit.
$r_e$ is the classical radius for the particle in metres.
The particle momentum vector is then changed at the entrance and at the
exit by
\begin{equation}
p_{x2} = p_{x1} - \Delta E/p_s (1 + p_{t1}) p_{x1}, \qquad
p_{y2} = p_{y1} - \Delta E/p_s (1 + p_{t1}) p_{y1}, \qquad
p_{t2} = p_{t1} - \Delta E/p_s (1 + p_{t1})^2.
\end{equation}
 
 
\section{Quantum Excitation by Synchrotron Radiation in Tracking}
\index{synchrotron radiation!quantum effects}
The algorithm for quantum excitation is based on~\cite{JOW80}.
The first step is to determine the mean number of photons emitted over
half the length of the element as
\begin{equation}
\half \langle N \rangle = \frac{5\sqrt{3}r_e}{6 \bar{h} c}p_s h \half L,
\end{equation}
and the actual number~$N$ of photons generated over half the length
is selected from the Poisson distribution with
average~$\langle N \rangle$. 
The critical energy, divided by~$p_s$, is computed as
\begin{equation}
U_c/p_s = \frac{3\bar{h}c}{2m_e}\gamma_s^2 h.
\end{equation}
The local curvature is evaluated separately for the entrance and the
exit.
$r_e$ is the classical radius for the particle in metres,
and $m_e$ is its mass in~GeV.
The second step is to sample $N$~photons from the distribution
\begin{equation}
\xi,
\end{equation}
and to sum up the energies of the photons, divided by~$p_s$:
\begin{equation}
\Delta E/p_s = \sum_{k=1}^N U_c/p_s \xi_k.
\end{equation}
Finally the particle momentum vector is changed by
\begin{equation}
p_{x2} = p_{x1} - \Delta E/p_s (1 + p_{t1}) p_{x1}, \qquad
p_{y2} = p_{y1} - \Delta E/p_s (1 + p_{t1}) p_{y1}, \qquad
p_{t2} = p_{t1} - \Delta E/p_s (1 + p_{t1})^2.
\end{equation}
 
 
\section{Equilibrium Emittances}
\index{synchrotron radiation!equilibrium emittances}
\index{emittances!equilibrium}
The algorithm for damping is based on~\cite{CHA79}.
***** section to be filled in *****
 
 
% ====================================================================
 
\chapter{HARMON Module}
\index{HARMON}
The HARMON calculations are based on a program originally written by
M.~Donald and linked to MAD by D.~Schofield~\cite{DON82}.
Various changes and improvements have been made later in this module.
 
 
\section{General Organisation of HARMON}
\index{HARMON!organisation}
HARMON ignores all coupling effects.
It uses thin lens approximations for all computations.
Integrations around the ring are replaced by summations over the
elements; all functions appearing in the integrands are approximated
by their average over the element.
Thus, for example, the integral $\int\!\!\sqrt{\beta_x}ds$ over the
length of an element is replaced by $L\sqrt{\langle\beta_x\rangle}$.
 
When the HARMON module is started by the command \ttindex{HARMON},
it sets up a table containing the averaged lattice functions for
all dipoles, quadrupoles, sextupoles, and thin multipoles.
It stores the averaged lattice functions
\begin{equation}\eqarray{
&\langle\beta_x\rangle  &=& \int\beta_x ds/L, \qquad
&\langle\alpha_x\rangle &=& \int\alpha_x ds/L, \qquad
&\langle\mu_x\rangle    &=& \int\mu_x ds/L, \\
&\langle Dx\rangle      &=& \int Dx ds/L, \qquad
&\langle Dp_x\rangle    &=& \int Dp_x ds/L, \\
&\langle\beta_y\rangle  &=& \int\beta_y ds/L, \qquad
&\langle\alpha_y\rangle &=& \int\alpha_y ds/L, \qquad
&\langle\mu_y\rangle    &=& \int\mu_y ds/L, \\
&\langle Dy\rangle      &=& \int D_y ds/L, \qquad
&\langle Dp_y\rangle    &=& \int Dp_y ds/L
}\end{equation}
in this table.
In subsequent print-outs HARMON lists these {\em averaged functions} for
all active elements.
 
HARMON evaluates multiple integrals in a very efficient way.
\index{HARMON!multiple integrals}
\index{multiple integrals in HARMON}
Applying addition theorems for trigonometric functions it separates
functions of different positions.
This makes it possible to evaluate a double integral as a double sum
whose outer sum runs over partial sums of the inner sum,
which may be evaluated in a single loop.
A similar method applies to triple integrals.
 
 
\section{First-Order Chromaticity}
\index{chromaticity!first order}
\index{HARMON!chromaticity}
\index{tune!shift with energy}
The linear chromaticity is found according to a formula given by
J\"ager and M\"ohl~\cite{JAE81}.
Defining
\begin{equation}
D = \frac{d}{dp_t}, \qquad
D^n = \frac{d^n}{dp_t^n}, \qquad ' = \frac{d}{ds}
\end{equation}
the chromaticities can be written as:
\begin{equation}\eqarray{
\frac{d Q_x}{dp_t} &=&
  \frac{1}{4 \pi} \left(
    - \int_0^C (K_1 + h^2) \beta_x ds
    + \int_0^C h Dx (2 K_1\beta_x + \gamma_x) ds
    - 2 \int_0^C h Dx' \alpha_x ds
    + \int_0^C K_2 Dx \beta_x ds
  \right), \\
\frac{d Q_y}{d p_t} &=&
  \frac{1}{4 \pi} \left(
    + \int_0^C K_1 \beta_x ds 
    + \int_0^C h Dx (- K_1 \beta_y + \gamma_y) ds
    + \int_0^C h'Dx' \beta_y ds
    - \int_0^C K_2 Dx \beta_y ds
  \right), \hfill \\
}\end{equation}
These authors have shown the importance that the term containing $h'$
is not omitted.
The first-order chromaticities are evaluated by exact integration,
and the results agree very well with the results found from
\ttindex{TWISS} (see\ref{chrom}).
 
 
\section{Higher-order Chromaticities}
\index{HARMON!chromaticity}
\index{chromaticity!second order}
\index{chromaticity!third order}
\index{tune!shift with energy}
The second- and third-order chromaticities consider only the effects
of quadrupoles and higher-order multipoles.
We introduce the functions
\begin{equation}\eqarray{
f_{x1}(s)&=&\beta_x(s)\Bigl(K_1(s)-K_2(s)Dx(s)\Bigr), \\
f_{y1}(s)&=&\beta_y(s)\Bigl(K_1(s)-K_2(s)Dx(s)\Bigr), \\
f_{x2}(s)&=&-f_{x1}-
  \beta_x(s)\Bigl(K_2(s)D^2x(s)+\half K_3(s)(Dx(s))^2\Bigr), \\
f_{y2}(s)&=&-f_{y1}-
  \beta_y(s)\Bigl(K_2(s)D^2x(s)+\half K_3(s)(Dx(s))^2\Bigr), \\
f_{x3}(s)&=&-f_{x2}-
  \beta_x(s)\Bigl(K_2(s)D^3x(s)+K_3(s)Dx(s)D^2x(s)+
                  \sixth K_4(s)(Dx(s))^3\Bigr), \\
f_{y3}(s)&=&-f_{y2}-
  \beta_y(s)\Bigl(K_2(s)D^3x(s)+K_3(s)Dx(s)D^2x(s)+
                  \sixth K_4(s)(Dx(s))^3\Bigr).
}\end{equation}
The second-order and third-order chromaticities can be found by the
methods given in reference~\cite{COU58} as the following integrals:
\begin{equation}\eqarray{
\frac{d^2 Q_x}{dp_t^2}&=&-\frac{1}{4\pi}\int_0^C\!\!f_{x2}(s) ds \\
& &-\frac{1}{4\pi\sin(2\pi Q_x)}\int_0^C\!\!\!\int_0^{s_1}\!\!
f_{x1}(s_1)f_{x1}(s_2)\times\\& &\qquad\times
\sin\Bigl(\mu_x(s_2)-\mu_x(s_1)\Bigr)
\sin\Bigl(2\pi Q_x+\mu_x(s_1)-\mu_x(s_2)\Bigr)
ds_2 ds_1\\
%
\frac{d^2 Q_y}{dp_t^2}&=&+\frac{1}{4\pi}\int_0^C\!\!f_{y2}(s) ds \\
& &-\frac{1}{4\pi\sin(2\pi Q_y)}\int_0^C\!\!\!\int_0^{s_1}\!\!
f_{y1}(s_1)f_{y1}(s_2)\times\\& &\qquad\times
\sin\Bigl(\mu_y(s_2)-\mu_y(s_1)\Bigr)
\sin\Bigl(2\pi Q_y+\mu_y(s_1)-\mu_y(s_2)\Bigr)
ds_2 ds_1\\
%
\frac{d^3 Q_x}{dp_t^3}&=&-\frac{1}{4\pi}\int_0^C\!\!f_{x3}(s) ds \\
& &-\frac{1}{4\pi\sin(2\pi Q_x)}\int_0^C\!\!\!\int_0^{s_1}\!\!
\Bigl(f_{x1}(s_1) f_{x2}(s_2) + f_{x2}(s_1) f_{x1}(s_2)\Bigr)\times\\
& &\qquad\times
\sin\Bigl(\mu_x(s_2)-\mu_x(s_1)\Bigr)
\sin\Bigl(2\pi Q_x+\mu_x(s_1)-\mu_x(s_2)\Bigr)
ds_2 ds_1\\
& &+\frac{1}{4\pi\sin(2\pi Q_x)}\int_0^C\!\!\!\int_0^{s_1}\!\!\!
\int_0^{s_2}\!\!f_{x1}(s_1) f_{x1}(s_2) f_{x1}(s_3)\times\\
& &\qquad\times
\sin\Bigl(\mu_x(s_3)-\mu_x(s_2)\Bigr)
\sin\Bigl(\mu_x(s_2)-\mu_x(s_1)\Bigr)
\sin\Bigl(2\pi Q_x+\mu_x(s_1)-\mu_x(s_3)\Bigr)
ds_3 ds_2 ds_1\\
%
\frac{d^3 Q_y}{dp_t^3}&=&+\frac{1}{4\pi}\int_0^C\!\!f_{y3}(s) ds \\
& &-\frac{1}{4\pi\sin(2\pi Q_y)}\int_0^C\!\!\!\int_0^{s_1}\!\!
\Bigl(f_{y1}(s_1) f_{y2}(s_2) + f_{y2}(s_1) f_{y1}(s_2)\Bigr)\times\\
& &\qquad\times
\sin\Bigl(\mu_y(s_2)-\mu_y(s_1)\Bigr)
\sin\Bigl(2\pi Q_y+\mu_y(s_1)-\mu_y(s_2)\Bigr)
ds_2 ds_1\\
& &-\frac{1}{4\pi\sin(2\pi Q_y)}\int_0^C\!\!\!\int_0^{s_1}\!\!\!
\int_0^{s_2}\!\!f_{y1}(s_1) f_{y1}(s_2) f_{y1}(s_3)\times\\
& &\qquad\times\sin\Bigl(\mu_y(s_3)-\mu_y(s_2)\Bigr)
\sin\Bigl(\mu_y(s_2)-\mu_y(s_1)\Bigr)
\sin\Bigl(2\pi Q_y+\mu_y(s_1)-\mu_y(s_3)\Bigr)
ds_3 ds_2 ds_1
}\end{equation}
 
 
\section{Variation of the Dispersion with Energy}
\index{chromatic functions}
\index{HARMON!chromatic functions}
The variation of the dispersion with energy can be found by the method
given in Section~\ref{kick}.
The closed orbit due to a distributed kick is
\begin{equation}
x(0)=\frac{1}{2\sin(\pi Q_x)}\int_0^C
\sqrt{\beta_x(0)\beta_x(s)}\cos\Bigl(\pi Q_x-\mu_x(s)\Bigr)k(s)ds
\end{equation}
Expanding the orbit as
\begin{equation}
x(p_t) = x + Dx p_t + D^2x p_t^2 + D^3x p_t^3 + \ldots
\end{equation}
and separating like powers of $p_t$ one gets the differential
equations
\begin{equation}\eqarray{
\frac{d^2Dx}{ds^2}   + K_1 Dx   &=& -h^2,\\
\frac{d^2D^2x}{ds^2} + K_1 D^2x &=& +h^2-K_1Dx+\half K_2(Dx)^2,\\
\frac{d^2D^3x}{ds^3} + K_1 D^3x &=& -h^2-K_1Dx+\half K_2(Dx)^2
-K_1D^2x+K_2DxD^2x+\sixth K_3(Dx)^3.\\
}\end{equation}
The first-order dispersion~$Dx = dx/dp_t$ is already known in the
table of averaged values.
At the origin of the ring the second-order dispersion~$D^2x$ is:
\begin{equation}
D^2x(0)=-Dx(0)+\frac{1}{2\sin(\pi Q_x)}\int_0^C
\sqrt{\beta_x(0)\beta_x(s)}
\cos\Bigl(\pi Q_x-\mu_x(s)\Bigr)
\Bigl(K_1(s)Dx(s)-\half K_2(s)(Dx(s))^2\Bigr)ds.
\end{equation}
The integral is evaluated as two sums to facilitate computation
of~$D^2x$ around the ring:
\begin{equation}\eqarray{
S_c&=&\sume\sqrt{\langle\beta_x\rangle}
\cos\Bigl(\langle\mu_x\rangle\Bigr)
\Bigl(K_1\langle Dx\rangle-\half K_2\langle Dx(s)\rangle^2\Bigr)ds,\\
%
S_s&=&\sume\sqrt{\langle\beta_x\rangle}
\sin\Bigl(\langle\mu_x\rangle\Bigr)
\Bigl(K_1\langle Dx\rangle-\half K_2\langle Dx(s)\rangle^2\Bigr)ds,\\
%
D^2x(0)&\approx&-Dx(0)+\frac{\sqrt{\beta_x(0)}}{2\sin(\pi Q_x)}
\Bigl(S_c\cos(\pi Q_x) + S_s\sin(\pi Q_y)\Bigr)
}\end{equation}
One may now proceed from one element to the next as follows:
\begin{equation}
\langle D^2x_i\rangle\approx-\langle Dx_i\rangle
+\frac{\sqrt{\langle\beta_x\rangle}}{2\sin(\pi Q_x)}
\Bigl(S_c\cos(\pi Q_x+\langle\mu_x\rangle)
+S_s\sin(\pi Q_x+\langle\mu_x\rangle)\Bigr).
\end{equation}
After each element one has to step the sums by
\begin{equation}\eqarray{
S_c&\leftarrow&S_c+\sqrt{\langle\beta_x\rangle}
\Bigl(K_1\langle Dx\rangle-\half K_s(\langle Dx\rangle)^2\Bigr)
\cos(2\pi Q_x+\langle\mu_x\rangle),\\
%
S_s&\leftarrow&S_s+\sqrt{\langle\beta_x\rangle}
\Bigl(K_1\langle Dx\rangle-\half K_s(\langle Dx\rangle)^2\Bigr)
\sin(2\pi Q_x+\langle\mu_x\rangle),
}\end{equation}
Once the value of~$D^2x(s)$ is known,
$D^3x(0)$ is found from the integral:
\begin{equation}\eqarray{
D^3x(0)&=&-D^2x(0)+\frac{1}{2\sin(\pi Q_x)}\int_0^C
\sqrt{\beta_x(0)\beta_x(s)}
\cos\Bigl(\pi Q_x-\mu_x(s)(s)\Bigr)
\times\\&&\qquad\times
\biggl(\Bigl(K_1(s) - K_2(s) Dx(s)\Bigr)D^2x(s)-
\sixth K_3(s)(Dx(s))^3\biggl)ds.
}\end{equation}
Evaluation of the integral and stepping around the ring is done as
for~$D^2x$.
 
 
\section{Variation of $\beta$ with Energy}
\index{chromatic functions}
\index{HARMON!chromatic functions}
Using the techniques of reference~\cite{COU58},
the total derivatives of the $\beta$~functions at the origin are:
\begin{equation}\eqarray{
\frac{1}{\beta_x(0)}\frac{d\beta_x(0)}{dp_t}&=&
+\frac{1}{2\sin(2\pi Q_x)}
\int_0^C\beta_x(s)\Bigl(K_1(s)-K_2(s)D_x(s)\Bigr)
\cos\Bigl(2\pi Q_x - 2 \mu_x(s)(s)\Bigr) ds\\
%
\frac{1}{\beta_y(0)}\frac{d\beta_y(0)}{dp_t}&=&
-\frac{1}{2\sin(2\pi Q_x)}
\int_0^C\beta_x(s)\Bigl(K_1(s)-K_2(s)D_x(s)\Bigr)
\cos\Bigl(2\pi Q_x - 2 \mu_x(s)(s)\Bigr) ds
}\end{equation}
These integrals are evaluated and stepped in HARMON like the ones in
the preceding section.
 
 
\section{Variation of Tunes with Amplitude}
\index{HARMON!anharmonicity}
\index{HARMON!tune shift with amplitude}
\index{tune!shift with amplitude}
The variation of tunes with amplitude has been described in
reference~\cite{DON77,DON84}.
The tune shifts due to the betatron amplitudes are
\begin{equation}
\Delta Q_x = G_{xx}\hat{x}^2 + G_{xy}\hat{y}^2, \qquad
\Delta Q_y = G_{xy}\hat{x}^2 + G_{yy}\hat{y}^2.
\end{equation}
With the abbreviations
\begin{equation}
\phi_x = Q_x - \mu_x(s_2) + \mu_x(s_1), \qquad
\phi_y = Q_y - \mu_y(s_2) + \mu_y(s_1), \qquad
\end{equation}
the coefficients are given as sums of double integrals:
\begin{equation}\eqarray{
G_{xx}&=&-\frac{1}{64\pi\sin(3\pi Q_x)}\int_0^C\!\!\!\int_0^{s_2}\!\!\!
\beta_x^{3/2}(s_1)K_2(s_1)\beta_x^{3/2}(s_2)K_2(s_2)
\cos(3\phi_x) ds_1 ds_2\\
&&-\frac{3}{64\pi\sin(\pi Q_x)}\int_0^C\!\!\!\int_0^{s_2}
\beta_x^{3/2}(s_1)K_2(s_1)\beta_x^{3/2}(s_2)K_2(s_2)
\cos(\phi_x) ds_1 ds_2\\
%
G_{xy}&=&+\frac{1}{16\pi\sin(\pi Q_x)}\int_0^C\!\!\!\int_0^{s_2}\!\!\!
\beta_x^{3/2}(s_1)K_2(s_1) \beta_x^{1/2}(s_2)\beta_y(s_2)K_2(s_2)
\cos(\phi_x) ds_1 ds_2\\
&&-\frac{1}{32\pi\sin(\pi(Q_x+2Q_y))}\int_0^C\!\!\!\int_0^{s_2}\!\!\!
\beta_x^{1/2}(s_1)\beta_y(s_1)K_2(s_1)
\beta_x^{1/2}(s_2)\beta_y(s_2)K_2(s_2)
\cos(\phi_x+2\phi_y) ds_1 ds_2\\
&&+\frac{1}{32\pi\sin(\pi(Q_x-2Q_y))}\int_0^C\!\!\!\int_0^{s_2}\!\!\!
\beta_x^{1/2}(s_1)\beta_y(s_1)K_2(s_1)
\beta_x^{1/2}(s_2)\beta_y(s_2)K_2(s_2)
\cos(\phi_x-2\phi_y) ds_1 ds_2,\\
%
G_{yy}&=&-\frac{1}{16\pi\sin(\pi Q_x)}\int_0^C\!\!\!\int_0^{s_2}\!\!\!
\beta_x^{1/2}(s_1)\beta_y(s_1)K_2(s_1)
\beta_x^{1/2}(s_2)\beta_y(s_2)K_2(s_2)
\cos(\phi_x) ds_1 ds_2\\
&&-\frac{1}{64\pi\sin(\pi(Q_x+2Q_y))}\int_0^C\!\!\!\int_0^{s_2}\!\!\!
\beta_x^{1/2}(s_1)\beta_y(s_1)K_2(s_1)
\beta_x^{1/2}(s_2)\beta_y(s_2)K_2(s_2)
\cos(\phi_x+2\phi_y) ds_1 ds_2\\
&&-\frac{1}{64\pi\sin(\pi(Q_x-2Q_y))}\int_0^C\!\!\!\int_0^{s_2}\!\!\!
\beta_x^{1/2}(s_1)\beta_y(s_1)K_2(s_1)
\beta_x^{1/2}(s_2)\beta_y(s_2)K_2(s_2)
\cos(\phi_x-2\phi_y) ds_1 ds_2.\\
}\end{equation}
 
 
\section{Resonances}
***** section is incomplete *****
 
\subsection{HRESONANCE Command}
The {\tt HRESONANCE} command computes the effect of an $n$-order
resonance.
 
\subsection{Sum Resonances}
For all combinations $n_1 + n_2 = n$ {\tt HRESONANCE} computes the
two multiples of the number of super-periods~$N_s$ just above and
below the expression $(n_1 Q_x + n_2 Q_y)$.
Defining
\begin{equation}
f_s = 2\pi (n_1 Q_x + n_2 Q_y - p) / C
\end{equation}
For the $k$-th sextupole the resonance contribution is
\begin{equation}
c_k = \frac{1}{2^n \pi n_1! n_2!}
  \int_0^{L_k} K_2 \beta_x(s)^{|n_1|/2} \beta_y(s)^{|_2|/2}
  \exp(i(n_1 \mu_x(s) + n_2 \mu_y(s) - f_s s)) ds,
\end{equation}
where the integration is over the sextupole length.
The integral is evaluated by fitting a cubic polynomial through the
values of the integrand and its derivatives at both ends of the
sextupole, and integrating this polynomial.
The results are given in the form of real part (cosine terms),
imaginary part (sine terms) and modulus (amplitude).
The random effects are
\begin{equation}
\hbox{random} = \sqrt{\sum_{\hbox{sextupoles}} (2 |c_k|^2)}.
\end{equation}
 
\iffalse
     +       7X,'dE(s)',7X,'dE(r)',7X,'dQ(s)',7X,'dQ(r)',5X,
     +       'dQ20(s)',5X,'dQ20(r)')
      data beat
     +  / 1.0d0,      1.0d0,      2.0d0, 1.553774d0, 1.384812d0 /
      data fact
     +  / 1.0d0, 5.545455d0, 2.066667d0, 1.446735d0, 1.194130d0 /
 
          fact2 = (N_1**2 / ex0 + N_2**2 / ey0) *
     +            (sqrt(ex0)**n1 * sqrt(ey0)**n2)
          fact3 = 1.0
          if (norder .gt. 2  .and.  norder .lt. 6) then
            fact2 = fact2 * beat(norder)**(norder-2)
            fact3 = fact(norder)
          endif
          des = summ * fact2
          der = rsum * fact2
          dqs = des / sqrt(N_1**2+N_2**2)
          dqr = der / sqrt(N_1**2+N_2**2)
          dqs20 = dqs * fact3
          dqr20 = dqr * fact3
          write (iqpr2, 950) sum, summ, rsum,
     +                       des, der, dqs, dqr, dqs20, dqr20
\fi
 
 
\section{Third-Integer Resonances in HFUNCTION Command}
***** section to be filled in *****
\iffalse
+dk hareso
      subroutine hareso(iprint, dh1, dv1, dh2, dv2)
*----------------------------------------------------------------------*
* purpose:                                                             *
*   calculation of 1/3rd integer resonances due to sextupoles.         *
*   penalty of the configuration is expressed as distortion of         *
*   the 4-dimensional phase space.                                     *
*   a = - n1 / sin(2*pi*q*n) * integral over one superperiod of        *
*       g * beta**(k/2) * k2 * complex exponent of                     *
*       n*q * (pi + phi - psi) * ds.                                   *
*   horizontal distortion is:                                          *
*   (emittance)**(k2/2) / eh * a.                                      *
*   vertical distortion is                                             *
*   (emittance)**(k2/2) / ev * a.                                      *
*     n1 = phase integer, horz or vert.                                *
*     n  = ditto          combined.                                    *
*     k  = amplitude integer, combined.                                *
* input:                                                               *
*   iprint    (integer) print flag.                                    *
* output:                                                              *
*   dh1       (real)    horizontal resonances at i.p.                  *
*   dv1       (real)    vertical resonances   at i.p.                  *
*   dh2       (real)    horizontal resonances at s.p.                  *
*   dv2       (real)    vertical resonances   at s.p.                  *
*----------------------------------------------------------------------*
+ca implicit
+ca memory
+ca bankhead
+ca prcgroup
+ca beam
+ca hacomm
+ca halink
+ca hasbuf
+ca range
+ca refer
+ca zunit
 
+ca pi
      parameter         (2\pi = 2.0d0 * pi)
 
*---- local variables.
      parameter         (nres = 5)
      dimension         src(nres), srs(nres), pbet(nres),
     +                  cb(nres), sb(nres), aa(nres),
     +                  cg(nres), sg(nres), sum1(5), sum2(5),
     +                  disth1(nres), distv1(nres),
     +                  disth2(nres), distv2(nres)
      character*(mcnam) elmnam
*     src, srs          accumulation of cos & sin components of sums.
*     pbet              k2 * beta**(k/2).
*     cb, sb            sin & cos of (n*pi*q).
*     aa                1 / (2 * sin(n*pi*q)).
*     cg, sg            cos and sin(n*q*psi).
*     sum1, sum2        summation at i.p. and symmmetry point.
*     disth1, distv1    distortion horz/vert at i.p.
*     disth2, distv2    ditto at symmetry point.
 
*---- clear the sums.
      do 10 i = 1, nres
        src(i) = 0.0
        srs(i) = 0.0
   10 continue
 
*---- accumulate the integrals.
      do 90 ipos = irg1, irg2
        call utelem(lcseq, ipos, iflag, elmnam, iocc, ienum)
        if (iq(lcelm+mbpr) .eq. mpelm) then
          call hastrg(angle, sk1l, K_2 L, sk3l, sk4l)
          if (K_2 L .ne. 0.0) then
            call tbset(lhastb, ipos, 1, lhasbf)
            call ucopy(iq(lhasbf+1), bxb, iq(lhasbf-1))
            pbet(1) = K_2 L * sqrt(bxb) * bxb
            pbet(2) = pbet(1)
            pbet(3) = K_2 L * sqrt(bxb) * byb
            pbet(4) = pbet(3)
            pbet(5) = pbet(3)
            cmux = cos(amuxb)
            smux = sin(amuxb)
            cmuy = cos(amuyb)
            smuy = sin(amuyb)
            c2x = cmux * cmux - smux * smux
            s2x = smux * cmux + cmux * smux
            c2y = cmuy * cmuy - smuy * smuy
            s2y = smuy * cmuy + cmuy * smuy
            cg(1) = cmux * c2x - smux * s2x
            cg(2) = cmux
            cg(3) = cmux * c2y - smux * s2y
            cg(4) = cmux * c2y + smux * s2y
            cg(5) = cg(2)
            sg(1) = smux * c2x + cmux * s2x
            sg(2) = smux
            sg(3) = smux * c2y + cmux * s2y
            sg(4) = smux * c2y - cmux * s2y
            sg(5) = sg(2)
            do 20 i = 1, nres
              src(i) = src(i) + pbet(i) * cg(i)
              srs(i) = srs(i) + pbet(i) * sg(i)
   20       continue
          endif
        endif
   90 continue
 
*---- set up (n*pi*q).
      qxn = pi * qx / nsup
      qy2n = 2\pi * qy / nsup
      aa(1) = 3.0 * qxn
      aa(2) = qxn
      aa(3) = qxn + qy2n
      aa(4) = qxn - qy2n
      aa(5) = qxn
      if (symm) then
        w1 = - 1.0
      else
        w1 = - 0.5
      endif
      do 110 i = 1, nres
        cb(i) = cos(aa(i))
        sb(i) = sin(aa(i))
        aa(i) = w1 / sb(i)
  110 continue
      aa(1) = aa(1) / (4.0 * 6.0)
      aa(2) = aa(2) / (4.0 * 2.0)
      aa(3) = aa(3) / (4.0 * 2.0)
      aa(4) = aa(4) / (4.0 * 2.0)
      aa(5) = aa(5) / 4.0
 
*---- get the functions by using the sums.
      do 120 i = 1, nres
        sum1(i) = (src(i) * cb(i) + srs(i) * sb(i)) * aa(i)
        if (symm) then
          sum2(i) = src(i) * aa(i)
        else
          sum2(i) = sum1(i)
        endif
  120 continue
 
*---- actual distortion.
      h = sqrt(ex) * ensigx
      v = ey * ensigy**2 / h
      disth1(1) = h * sum1(1) * 3.0
      disth1(2) = h * sum1(2)
      disth1(3) = v * sum1(3)
      disth1(4) = v * sum1(4)
      disth1(5) = v * sum1(5)
      distv1(1) = 0.0
      distv1(2) = 0.0
      distv1(3) = h * sum1(3) * 2.0
      distv1(4) = h * sum1(4) * 2.0
      distv1(5) = 0.0
      disth2(1) = h * sum2(1) * 3.0
      disth2(2) = h * sum2(2)
      disth2(3) = v * sum2(3)
      disth2(4) = v * sum2(4)
      disth2(5) = v * sum2(5)
      distv2(1) = 0.0
      distv2(2) = 0.0
      distv2(3) = h * sum2(3) * 2.0
      distv2(4) = h * sum2(4) * 2.0
      distv2(5) = 0.0
 
*---- sum up.
      dh1 = sqrt(disth1(1)**2 + disth1(2)**2 + disth1(3)**2 +
     +           disth1(4)**2 + disth1(5)**2)
      dh2 = sqrt(disth2(1)**2 + disth2(2)**2 + disth2(3)**2 +
     +           disth2(4)**2 + disth2(5)**2)
      dv1 = sqrt(distv1(1)**2 + distv1(2)**2 + distv1(3)**2 +
     +           distv1(4)**2 + distv1(5)**2)
      dv2 = sqrt(distv2(1)**2 + distv2(2)**2 + distv2(3)**2 +
     +           distv2(4)**2 + distv2(5)**2)
 
*---- output.
      if (iprint .gt. 0) then
        write (iqpr2, 910)
        write (iqpr2, 930)
        write (iqpr2, 940) (sum1(i), i = 1, nres)
        write (iqpr2, 950) (disth1(i), i = 1, nres)
        write (iqpr2, 960) (distv1(i), i = 1, nres)
        if (symm) then
          write (iqpr2, 920)
          write (iqpr2, 930)
          write (iqpr2, 940) (sum2(i), i = 1, nres)
          write (iqpr2, 950) (disth2(i), i = 1, nres)
          write (iqpr2, 960) (distv2(i), i = 1, nres)
        endif
      endif
 
  910 format(' '/' '/' resonance coefficients at interaction point:')
  920 format(' '/' resonance coefficients at symmetry point:')
  930 format(' resonances',t38,'3000',12x,'2100',12x,'1020',12x,
     +                         '1002',12x,'1011')
  940 format(' coefficients',t26,1p,5e16.6)
  950 format(' horizontal  ',t26,1p,5e16.6)
  960 format(' vertical    ',t26,1p,5e16.6)
 
      end
\fi
 
 
\section{Fourth-Order Resonances}
***** section to be filled in *****
Find fourth-order resonance coefficients
\iffalse
+dk ha4ana
      subroutine ha4ana
*----------------------------------------------------------------------*
* purpose:                                                             *
*   analyse the fourth order effects of sextupoles,                    *
*   fourth order meaning octupole effects = quarter integer effects.   *
*----------------------------------------------------------------------*
+ca implicit
+ca beam
+ca hacomm
+ca range
+ca zunit
 
      parameter         (beat4 = 1.553774d0, fact4 = 1.446735d0)
 
      dimension         ans1(2),  ans2(2),  ans3(2),  g(2)
      dimension         g2200(2), g0022(2), g1111(2), g4000(2),
     +                  g0040(2), g2020(2), g2002(2)
      logical first
 
*---- fourth order effects of sextupoles.
      write (iqpr2, 910)
      ex0 = ex * ensigx**2
      ey0 = ey * ensigy**2
 
*==== q shift effects.
*---- g2200.
      write (iqpr2, 920)
*     if (.not. skew) then
        call ha4sum(3, 0, 0, 0, 0, 3, 0, 0, 0, ans1)
        call ha4sum(2, 1, 0, 0, 1, 2, 0, 0, 0, ans2)
        g2200(1) = - 9.0*ans1(1) - 3.0*ans2(1)
        g2200(2) = - 9.0*ans1(2) - 3.0*ans2(2)
*     else
*       call ha4sum(2, 0, 1, 0, 0, 2, 0, 1, 0, ans1)
*       call ha4sum(0, 2, 1, 0, 2, 0, 0, 1, 0, ans2)
*       call ha4sum(1, 1, 1, 0, 1, 1, 0, 1, 0, ans3)
*       g2200(1) = - (ans1(1) + ans2(1) + ans3(1))
*       g2200(2) = - (ans1(2) + ans2(2) + ans3(2))
*     endif
      dqde = 2.0 * g2200(1)
      dqx4 = dqde * ex0
      write (iqpr2, 930) g2200, dqde, dqx4
 
*---- g00220.
*     if (.not. skew) then
        call ha4sum(1, 0, 2, 0, 0, 1, 0, 2, 0, ans1)
        call ha4sum(1, 0, 0, 2, 0, 1, 2, 0, 0, ans2)
        call ha4sum(1, 0, 1, 1, 0, 1, 1, 1, 0, ans3)
        g0022(1) = - (ans1(1) + ans2(1) + ans3(1))
        g0022(2) = - (ans1(2) + ans2(2) + ans3(2))
*     else
*       call ha4sum(0, 0, 3, 0, 0, 0, 0, 3, 0, ans1)
*       call ha4sum(0, 0, 2, 1, 0, 0, 1, 2, 0, ans2)
*       g0022(1) = - 9.0*ans1(1) - 3.0ans2(1)
*       g0022(2) = - 9.0*ans1(2) - 3.*ans2(2)
*     endif
      dqde = 2.0 * g0022(1)
      dqdy = dqde * ey0
      write (iqpr2, 940) g0022, dqde, dqdy
 
*---- g11110.
*     if (.not. skew) then
        call ha4sum(1, 0, 2, 0, 0, 1, 0, 2, 0, ans1)
        call ha4sum(0, 1, 2, 0, 1, 0, 0, 2, 0, ans2)
        call ha4sum(2, 1, 0, 0, 0, 1, 1, 1, 0, ans3)
*     else
*       call ha4sum(2, 0, 0, 1, 0, 2, 1, 0, 0, ans1)
*       call ha4sum(2, 0, 1, 0, 0, 2, 0, 1, 0, ans2)
*       call ha4sum(0, 0, 2, 1, 1, 1, 0, 1, 0, ans3)
*     endif
 
*---- third term is 2 * (ans3 + conjg(ans3)).
      g1111(1) = - 4.0 * (ans1(1) + ans2(1) + ans3(1))
      g1111(2) = - 4.0 * (ans1(2) + ans2(2))
      dqxdey = g1111(1)
      dqydex = dqxdey
      dqx4 = dqxdey * ey0
      dqy4 = dqydex * ex0
      write (iqpr2, 950) g1111, dqxdey, dqydex, dqx4, dqy4
 
*==== resonance effects.
      first = .true.
 
*---- g4000p.
   70 continue
      write (iqpr2, 960)
      np = int(4.0 * qx / nsup)
      if (.not. first) np = np + 1
*     if (.not. skew) then
        call ha4sum(2, 1, 0, 0, 3, 0, 0, 0, np, g4000)
        g(1) = 3.0 * g4000(1)
        g(2) = 3.0 * g4000(2)
*     else
*       call ha4sum(2, 0, 0, 1, 2, 0, 1, 0, np, g4000)
*       g(1) = g4000(1)
*       g(2) = g4000(2)
*     endif
      j0 = 4
      k0 = 0
      l0 = 0
      m0 = 0
      assign 80 to igo
      go to 120
 
*---- g0040p.
   80 continue
      np = int(4.0 * qy / nsup)
      if (.not. first) np = np + 1
*     if (.not. skew) then
        call ha4sum(0, 1, 2, 0, 1, 0, 2, 0, np, g0040)
        g(1) = g0040(1)
        g(2) = g0040(2)
*     else
*       call ha4sum(0, 0, 2, 1, 0, 0, 3, 0, np, g0040)
*       g(1) = 3.0 * g0040(1)
*       g(2) = 3.0 * g0040(2)
*     endif
      j0 = 0
      l0 = 4
      assign 90 to igo
      go to 120
 
*---- g2020.
   90 continue
      np = int((2.0 * qx + 2.0 * qy) / nsup)
      if (.not. first) np = np + 1
*     if (.not. skew) then
        call ha4sum(2, 1, 0, 0, 1, 0, 2, 0, np, ans1)
        call ha4sum(0, 1, 2, 0, 3, 0, 0, 0, np, ans2)
        call ha4sum(1, 0, 1, 1, 1, 0, 2, 0, np, ans3)
        g2020(1) = ans1(1) + 3.0*ans2(1) + 2.0*ans3(1)
        g2020(2) = ans1(2) + 3.0*ans2(2) + 2.0*ans3(2)
*     else
*       call ha4sum(1, 1, 1, 0, 2, 0, 1, 0, np, ans1)
*       call ha4sum(0, 0, 2, 1, 2, 0, 1, 0, np, ans2)
*       call ha4sum(2, 0, 0, 1, 0, 0, 3, 0, np, ans3)
*       g2020(1) = 2.0*ans1(1) + ans2(1) + 3.0*ans3(1)
*       g2020(2) = 2.0*ans1(2) + ans2(2) + 3.0*ans3(2)
*     endif
      j0 = 2
      l0 = 2
      g(1) = g2020(1)
      g(2) = g2020(2)
      assign 150 to igo
 
*==== resonance widths.
  120 continue
      n1 = j0 + k0
      n2 = l0 + m0
      N_1 = n1
      N_2 = n2
      g(1) = - g(1)
      g(2) = - g(2)
      gmod = sqrt(g(1)**2 + g(2)**2)
      des = gmod * sqrt(ex0)**n1 * sqrt(ey0)**n2 *
     +      (N_1**2 / ex0 + N_2**2 / ey0)
      dqs = des * beat4**2 / sqrt(N_1**2 + N_2**2)
      dqs20 = dqs * fact4
      ip = nsup * np
      write (iqpr2, 970) j0, k0, l0, m0, ip, g, des, dqs, dqs20
      go to igo, (80, 90, 150)
 
*==== difference resonances.
*---- g2002.
  150 continue
      np = int((2.0 * qx - 2.0 * qy) / nsup)
      if (.not. first) np = np + 1
*     if (.not. skew) then
        call ha4sum(2, 1, 0, 0, 1, 0, 0, 2, np, ans1)
        call ha4sum(0, 1, 0, 2, 3, 0, 0, 0, np, ans2)
        call ha4sum(1, 0, 1, 1, 1, 0, 0, 2, np, ans3)
        g2002(1) = ans1(1) + 3.0*ans2(1) + 2.0*ans3(1)
        g2002(2) = ans1(2) + 3.0*ans2(2) + 2.0*ans3(2)
*     else
*       call ha4sum(1, 1, 0, 1, 2, 0, 0, 1, np, ans1)
*       call ha4sum(0, 0, 1, 2, 2, 0, 0, 1, np, ans2)
*       call ha4sum(2, 0, 1, 0, 0, 0, 0, 3, np, ans3)
*       g2002(1) = 2.0*ans1(1) + ans2(1) + 3.0*ans3(1)
*       g2002(2) = 2.0*ans1(2) + ans2(2) + 3.0*ans3(2)
*     endif
      gmod = sqrt(g2002(1)**2 + g2002(2)**2)
      ip = nsup * np
      write (iqpr2, 980) ip, g2002, gmod
 
*---- set up for resonances above the working point.
      if (.not. first) return
      first = .false.
      go to 70
 
  910 format(' '/' fourth order effects of sextupoles:')
  920 format(' '/' q shift effects')
  930 format(' '/20x,'g95000',6x,'dqx/dex',10x,'dqx'/1p,4e13.5)
  940 format(' '/20x,'g00950',6x,'dqy/dey',10x,'dqy'/1p,4e13.5)
  950 format(' '/20x,'g11110',6x,'dqx/dey',6x,'dqy/dex',10x,'dqx',10x,
     +       'dqy'/1p,6e13.5)
  960 format(' '/' resonance effects')
  970 format(' '/22x,'cos',9x,'sin',10x,'de',10x,'dq',6x,'dq(20)'/
     +       ' g',4i1,',',i3,3x,1p,5e12.4)
  980 format(' '/22x,'cos',9x,'sin',5x,'modulus'/
     +       ' g2002',',',i3,3x,1p,3e12.4)
 
      end
+dk ha4sum
      subroutine ha4sum(jp, kp, lp, mp, jpp, kpp, lpp, mpp, ip, sm)
*----------------------------------------------------------------------*
* purpose:                                                             *
*   evaluate quarter integer resonances for use in ha4ana.             *
* input:                                                               *
*   jp,kp,lp,mp         indices for first factor.                      *
*   jpp,kpp,lpp,mpp     indices for second factor.                     *
*   ip                  value of p in n1*qx + n2*qy = p.               *
* output:                                                              *
*   sm(2)     (real)    computed resonance integrals.                  *
*----------------------------------------------------------------------*
+ca implicit
      dimension         sm(2)
+ca memory
+ca bankhead
+ca prcgroup
+ca beam
+ca halbuf
+ca halink
+ca optic0
+ca range
+ca refer
 
+ca pi
      parameter         (1 = 1.0d0, 2 = 2.0d0, 12 = 12.0d0)
      parameter         (2\pi = 2 * pi)
 
      character*(mcnam) elmnam
 
      na1p   = jp + kp
      na2p   = lp + mp
      np     = na1p + na2p
      na1pp  = jpp + kpp
      na2pp  = lpp + mpp
      npp    = na1pp + na2pp
      n1p    = jp - kp
      n2p    = lp - mp
      n1pp   = jpp - kpp
      n2pp   = lpp - mpp
      sup    = nsup
      ccp    = sup * (- 1) ** ((na2p + 1) / 2) / (2**np * pi *
     +         factor(jp) * factor(kp) * factor(lp) * factor(mp))
      ccpp   = sup * (- 1) ** ((na2pp + 1) / 2) / (2**npp * pi *
     +         factor(jpp) * factor(kpp) * factor(lpp) * factor(mpp))
 
      tune = (n1p * qx + n2p * qy) / sup
      cospiq = cos(pi * tune)
      sinpiq = sin(pi * tune)
 
*---- clear simple sums.
      scp = 0.0
      ssp = 0.0
      scpp = 0.0
      sspp = 0.0
 
*---- clear double sums.
      sum1 = 0.0
      sum2 = 0.0
      sum11 = 0.0
      sum12 = 0.0
      sum21 = 0.0
      sum22 = 0.0
 
*---- loop for all elements.
      do 90 ipos = irg1, irg2
        call utelem(lcseq, ipos, iflag, elmnam, iocc, ienum)
        if (iq(lcelm+mbpr) .eq. mpelm) then
 
*---- find sextupole strength.
          call hastrg(angle, sk1l, K_2 L, sk3l, sk4l)
          if (K_2 L .ne. 0.0) then
 
*---- fetch lattice functions at both ends.
            call tbset(lhaltb, ipos - 1, 1, lhalbf)
            call ucopy(iq(lhalbf+1), bx1, iq(lhalbf-1))
            call tbset(lhaltb, ipos, 1, lhalbf)
            call ucopy(iq(lhalbf+1), bx2, iq(lhalbf-1))
 
*---- averaged arc length.
            el = s2 - s1
            thetan = sup * pi * (s2 + s1) / C
 
            b1 = sqrt(bx1)**na1p * sqrt(by1)**na2p
            b2 = sqrt(bx2)**na1p * sqrt(by2)**na2p
            a1 = na1p * ax1 / bx1 + na2p * ay1 / by1
            a2 = na1p * ax2 / bx2 + na2p * ay2 / by2
            betap = K_2 L * ((b2 + b1) / 2 + el * (a2 - a1) / 12)
 
            b1 = sqrt(bx1)**na1pp * sqrt(by1)**na2pp
            b2 = sqrt(bx2)**na1pp * sqrt(by2)**na2pp
            a1 = na1pp * ax1 / bx1 + na2pp * ay1 / by1
            a2 = na1pp * ax2 / bx2 + na2pp * ay2 / by2
            betapp = K_2 L * ((b2 + b1) / 2 + el * (a2 - a1) / 12)
 
*---- averaged phase functions.
            amuxb = (amux2 + amux1)/2 - el*(1/bx2 - 1/bx1)/12
            amuyb = (amuy2 + amuy1)/2 - el*(1/by2 - 1/by1)/12
            phip = n1p * (amuxb - qx * thetan / sup)
     +           + n2p * (amuyb - qy * thetan / sup)
     +           + tune * thetan
            phipp = n1pp * (amuxb - qx * thetan / sup)
     +            + n2pp * (amuyb - qy * thetan / sup)
     +            + (ip - tune) * thetan
 
*---- terms to be integrated.
            tcp = betap * cos(phip)
            tsp = betap * sin(phip)
            tcpp = betapp * cos(phipp)
            tspp = betapp * sin(phipp)
 
*---- machine is symmetric.
            if (symm) then
              sum1 = sum1 + tcp  * (tcpp + 2 * scpp)
     +                    + tcpp * (tcp  + 2 * scp)
              sum2 = sum2 + tsp  * (tcpp + 2 * scpp)
     +                    - tspp * (tcp  + 2 * scp)
 
*---- machine is asymmetric.
            else
              sum1 = sum1 + (tcp * tcpp - tsp * tspp)
              sum2 = sum2 + (tsp * tcpp + tcp * tspp)
              sum11 = sum11 + (tcp * scpp - tsp * sspp)
              sum12 = sum12 + (tcpp * scp - tspp * ssp)
              sum21 = sum21 + (tsp * scpp + tcp * sspp)
              sum22 = sum22 + (tspp * scp + tcpp * ssp)
            endif
 
*---- accumulate partial sums for next pass (sum from 1 to i-1).
            ssp  = ssp  + tsp
            sspp = sspp + tspp
            scp  = scp  + tcp
            scpp = scpp + tcpp
          endif
        endif
   90 continue
 
      if (symm) then
        sm(1) = (2\pi / (sup * sinpiq)) * ccp * ccpp *
     +          (cospiq * sum1 + sinpiq * sum2)
        sm(2) = 0.0
      else
        sm(1) = (pi / (sup * sinpiq)) * ccp * ccpp *
     +    (cospiq * (sum1 + sum11 + sum12) + sinpiq * (sum21 - sum22))
        sm(2) = (pi / (sup * sinpiq)) * ccp * ccpp *
     +    (cospiq * (sum2 + sum22 + sum21) + sinpiq * (sum12 - sum11))
      endif
 
      end
\fi
 
% ====================================================================
 
\begin{thebibliography}{99}
 
\bibitem{BAS80} M.~Bassetti and G.~A.~Erskine.  {\it Closed expression
for the electrical field of a two-dimensional Gaussian charge}.
CERN-ISR-TH/80-06.
 
\bibitem{SLAC75} Karl~L. Brown.  {\it A First-and Second-Order Matrix
Theory for the Design of Beam Transport Systems and Charged Particle
Spectrometers}.  SLAC 75, Revision 3, SLAC, 1972, and SLAC-PUB-3381,
July 1984.
 
\bibitem{SLAC91} K.~L.~Brown, D.~C.~Carey, Ch.~Iselin,and
F.~Rothacker, {\it TRANSPORT --- A Computer Program for Designing
  Charged Particle Beam Transport Systems}.  CERN 73-16, revised as
CERN 80-4, CERN, 1980.
 
\bibitem{CHA79} A. Chao.  Evaluation of beam distribution parameters
in an electron storage ring.  {\it Journal of Applied Physics},
50:595--598, 1979.
 
\bibitem{COU58} E.~D. Courant and H.~S. Snyder.  {\it Theory of the
alternating gradient synchrotron}.  Annals of Physics, 3:1--48, 1958.
 
\bibitem{DON77} M.~Donald, P.~L.~Morton, and H.~Wiedemann.
{\it Chromaticity Correction in Large Storage Rings}.
IEEE Transaction on Nuclear Science, Vol.~NS-24, No. 3, June~1977.
 
\bibitem{DON82} M.~Donald and D.~Schofield.
{\it A User's Guide to the HARMON Program}.
LEP Note 420, CERN, 1982.
 
\bibitem{DON84} M.~Donald, Private Communication.
 
\bibitem{DOU82} D.~R.~Douglas, {\it Lie Algebraic Methods for Particle
Accelerator Theory}.  Doctoral thesis, University of Maryland,
unpublished, 1982.
 
\bibitem{DRA81} A.~Dragt, {\it Lectures on Nonlinear Orbit Dynamics,
1981 Summer School on High Energy Particle Accelerators, Fermi
National Accelerator Laboratory, July 1981}.  American Institute of
Physics, 1982.
 
\bibitem{GRO90} H.~Grote, F.~C.~Iselin, {\it The MAD Program
(Methodical Accelerator Design) Version 8.1, User's Reference Manual},
CERN/SL/90-13 (AP).
 
\bibitem{HEA88} L.~M.~Healy, {\it Concatenation of Lie Algebraic
Maps}.  CERN, LEP Note No. ???.
 
\bibitem{HEA86} L.~M.~Healy, {\it Lie Algebraic Methods for Treating
Lattice Parameter Errors in Particle Accelerators}.  Doctoral thesis,
University of Maryland, unpublished, 1986.
 
\bibitem{JAE81} J.~J\"ager and D.~M\"ohl, {\it Comparison of Methods
to Evaluate the Chromaticity in LEAR}.  CERN PS/DL/LEAR/Note 81-7.
 
\bibitem{ISE85} F.~Ch.~Iselin, {\it Lie Transformations and Transport
Equations for Combined-Function Dipoles}, Particle Accelerators, 1985,
{\bf 17}, 143-155.
 
\bibitem{JOW80} J.~Jowett, .
 
\bibitem{MAI82} H.~Mais and G.~Ripken, {\it Theory of Coupled
Synchro-Betatron Oscillations}, DESY internal Report, DESY M-82-05,
1982.
 
\bibitem{MIL88} J. Milutinovic and S. Ruggiero.\hfil {\it Comparison
of Accelerator Codes for a RHIC Lattice}.  AD/AP/TN-9, BNL, 1988.
 
\bibitem{PEG81} S.~Peggs, {\it Some Aspects of Machine Physics in the
Cornell Electron Storage Ring}, Doctoral thesis, Cornell University,
unpublished, 1981.
 
\bibitem{RIP70} G.~Ripken, {\it Untersuchungen zur Strahlf\"uhrung und
Stabilit\"at der Teilchenbewegung in Beschleunigern und Storage-Ringen
unter strenger Ber\"ucksichtigung einer Kopplung der
Betatronschwingungen.}, DESY internal Report R1-70/4, 1970.
 
\bibitem{TEN71} L.~C. Teng, {\it Concerning n-Dimensional Coupled
Motion}.  FN 229, FNAL, 1971.
 
\end{thebibliography}
 
\printindex
 
% ====================================================================
\end{document}
\end